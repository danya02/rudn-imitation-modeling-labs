\documentclass[11pt]{article}

    \usepackage[breakable]{tcolorbox}
    \usepackage{parskip} % Stop auto-indenting (to mimic markdown behaviour)
    

    % Basic figure setup, for now with no caption control since it's done
    % automatically by Pandoc (which extracts ![](path) syntax from Markdown).
    \usepackage{graphicx}
    % Maintain compatibility with old templates. Remove in nbconvert 6.0
    \let\Oldincludegraphics\includegraphics
    % Ensure that by default, figures have no caption (until we provide a
    % proper Figure object with a Caption API and a way to capture that
    % in the conversion process - todo).
    \usepackage{caption}
    \DeclareCaptionFormat{nocaption}{}
    \captionsetup{format=nocaption,aboveskip=0pt,belowskip=0pt}

    \usepackage{float}
    \floatplacement{figure}{H} % forces figures to be placed at the correct location
    \usepackage{xcolor} % Allow colors to be defined
    \usepackage{enumerate} % Needed for markdown enumerations to work
    \usepackage{geometry} % Used to adjust the document margins
    \usepackage{amsmath} % Equations
    \usepackage{amssymb} % Equations
    \usepackage[russian]{babel}
    \usepackage{textcomp} % defines textquotesingle
    % Hack from http://tex.stackexchange.com/a/47451/13684:
    \AtBeginDocument{%
        \def\PYZsq{\textquotesingle}% Upright quotes in Pygmentized code
    }
    \usepackage{upquote} % Upright quotes for verbatim code
    \usepackage{eurosym} % defines \euro

    \usepackage{iftex}
    \ifPDFTeX
        \usepackage[T1]{fontenc}
        \IfFileExists{alphabeta.sty}{
              \usepackage{alphabeta}
          }{
              \usepackage[mathletters]{ucs}
              \usepackage[utf8x]{inputenc}
          }
    \else
        \usepackage{fontspec}
        \usepackage{unicode-math}
    \fi

    \usepackage{fancyvrb} % verbatim replacement that allows latex
    \usepackage{grffile} % extends the file name processing of package graphics
                         % to support a larger range
    \makeatletter % fix for old versions of grffile with XeLaTeX
    \@ifpackagelater{grffile}{2019/11/01}
    {
      % Do nothing on new versions
    }
    {
      \def\Gread@@xetex#1{%
        \IfFileExists{"\Gin@base".bb}%
        {\Gread@eps{\Gin@base.bb}}%
        {\Gread@@xetex@aux#1}%
      }
    }
    \makeatother
    \usepackage[Export]{adjustbox} % Used to constrain images to a maximum size
    \adjustboxset{max size={0.9\linewidth}{0.9\paperheight}}

    % The hyperref package gives us a pdf with properly built
    % internal navigation ('pdf bookmarks' for the table of contents,
    % internal cross-reference links, web links for URLs, etc.)
    \usepackage{hyperref}
    % The default LaTeX title has an obnoxious amount of whitespace. By default,
    % titling removes some of it. It also provides customization options.
    \usepackage{titling}
    \usepackage{longtable} % longtable support required by pandoc >1.10
    \usepackage{booktabs}  % table support for pandoc > 1.12.2
    \usepackage{array}     % table support for pandoc >= 2.11.3
    \usepackage{calc}      % table minipage width calculation for pandoc >= 2.11.1
    \usepackage[inline]{enumitem} % IRkernel/repr support (it uses the enumerate* environment)
    \usepackage[normalem]{ulem} % ulem is needed to support strikethroughs (\sout)
                                % normalem makes italics be italics, not underlines
    \usepackage{mathrsfs}
    

    
    % Colors for the hyperref package
    \definecolor{urlcolor}{rgb}{0,.145,.698}
    \definecolor{linkcolor}{rgb}{.71,0.21,0.01}
    \definecolor{citecolor}{rgb}{.12,.54,.11}

    % ANSI colors
    \definecolor{ansi-black}{HTML}{3E424D}
    \definecolor{ansi-black-intense}{HTML}{282C36}
    \definecolor{ansi-red}{HTML}{E75C58}
    \definecolor{ansi-red-intense}{HTML}{B22B31}
    \definecolor{ansi-green}{HTML}{00A250}
    \definecolor{ansi-green-intense}{HTML}{007427}
    \definecolor{ansi-yellow}{HTML}{DDB62B}
    \definecolor{ansi-yellow-intense}{HTML}{B27D12}
    \definecolor{ansi-blue}{HTML}{208FFB}
    \definecolor{ansi-blue-intense}{HTML}{0065CA}
    \definecolor{ansi-magenta}{HTML}{D160C4}
    \definecolor{ansi-magenta-intense}{HTML}{A03196}
    \definecolor{ansi-cyan}{HTML}{60C6C8}
    \definecolor{ansi-cyan-intense}{HTML}{258F8F}
    \definecolor{ansi-white}{HTML}{C5C1B4}
    \definecolor{ansi-white-intense}{HTML}{A1A6B2}
    \definecolor{ansi-default-inverse-fg}{HTML}{FFFFFF}
    \definecolor{ansi-default-inverse-bg}{HTML}{000000}

    % common color for the border for error outputs.
    \definecolor{outerrorbackground}{HTML}{FFDFDF}

    % commands and environments needed by pandoc snippets
    % extracted from the output of `pandoc -s`
    \providecommand{\tightlist}{%
      \setlength{\itemsep}{0pt}\setlength{\parskip}{0pt}}
    \DefineVerbatimEnvironment{Highlighting}{Verbatim}{commandchars=\\\{\}}
    % Add ',fontsize=\small' for more characters per line
    \newenvironment{Shaded}{}{}
    \newcommand{\KeywordTok}[1]{\textcolor[rgb]{0.00,0.44,0.13}{\textbf{{#1}}}}
    \newcommand{\DataTypeTok}[1]{\textcolor[rgb]{0.56,0.13,0.00}{{#1}}}
    \newcommand{\DecValTok}[1]{\textcolor[rgb]{0.25,0.63,0.44}{{#1}}}
    \newcommand{\BaseNTok}[1]{\textcolor[rgb]{0.25,0.63,0.44}{{#1}}}
    \newcommand{\FloatTok}[1]{\textcolor[rgb]{0.25,0.63,0.44}{{#1}}}
    \newcommand{\CharTok}[1]{\textcolor[rgb]{0.25,0.44,0.63}{{#1}}}
    \newcommand{\StringTok}[1]{\textcolor[rgb]{0.25,0.44,0.63}{{#1}}}
    \newcommand{\CommentTok}[1]{\textcolor[rgb]{0.38,0.63,0.69}{\textit{{#1}}}}
    \newcommand{\OtherTok}[1]{\textcolor[rgb]{0.00,0.44,0.13}{{#1}}}
    \newcommand{\AlertTok}[1]{\textcolor[rgb]{1.00,0.00,0.00}{\textbf{{#1}}}}
    \newcommand{\FunctionTok}[1]{\textcolor[rgb]{0.02,0.16,0.49}{{#1}}}
    \newcommand{\RegionMarkerTok}[1]{{#1}}
    \newcommand{\ErrorTok}[1]{\textcolor[rgb]{1.00,0.00,0.00}{\textbf{{#1}}}}
    \newcommand{\NormalTok}[1]{{#1}}

    % Additional commands for more recent versions of Pandoc
    \newcommand{\ConstantTok}[1]{\textcolor[rgb]{0.53,0.00,0.00}{{#1}}}
    \newcommand{\SpecialCharTok}[1]{\textcolor[rgb]{0.25,0.44,0.63}{{#1}}}
    \newcommand{\VerbatimStringTok}[1]{\textcolor[rgb]{0.25,0.44,0.63}{{#1}}}
    \newcommand{\SpecialStringTok}[1]{\textcolor[rgb]{0.73,0.40,0.53}{{#1}}}
    \newcommand{\ImportTok}[1]{{#1}}
    \newcommand{\DocumentationTok}[1]{\textcolor[rgb]{0.73,0.13,0.13}{\textit{{#1}}}}
    \newcommand{\AnnotationTok}[1]{\textcolor[rgb]{0.38,0.63,0.69}{\textbf{\textit{{#1}}}}}
    \newcommand{\CommentVarTok}[1]{\textcolor[rgb]{0.38,0.63,0.69}{\textbf{\textit{{#1}}}}}
    \newcommand{\VariableTok}[1]{\textcolor[rgb]{0.10,0.09,0.49}{{#1}}}
    \newcommand{\ControlFlowTok}[1]{\textcolor[rgb]{0.00,0.44,0.13}{\textbf{{#1}}}}
    \newcommand{\OperatorTok}[1]{\textcolor[rgb]{0.40,0.40,0.40}{{#1}}}
    \newcommand{\BuiltInTok}[1]{{#1}}
    \newcommand{\ExtensionTok}[1]{{#1}}
    \newcommand{\PreprocessorTok}[1]{\textcolor[rgb]{0.74,0.48,0.00}{{#1}}}
    \newcommand{\AttributeTok}[1]{\textcolor[rgb]{0.49,0.56,0.16}{{#1}}}
    \newcommand{\InformationTok}[1]{\textcolor[rgb]{0.38,0.63,0.69}{\textbf{\textit{{#1}}}}}
    \newcommand{\WarningTok}[1]{\textcolor[rgb]{0.38,0.63,0.69}{\textbf{\textit{{#1}}}}}


    % Define a nice break command that doesn't care if a line doesn't already
    % exist.
    \def\br{\hspace*{\fill} \\* }
    % Math Jax compatibility definitions
    \def\gt{>}
    \def\lt{<}
    \let\Oldtex\TeX
    \let\Oldlatex\LaTeX
    \renewcommand{\TeX}{\textrm{\Oldtex}}
    \renewcommand{\LaTeX}{\textrm{\Oldlatex}}
    % Document parameters
    % Document title
    \title{Лабораторная работа 1}
    \author{Генералов Даниил, НПИбд-01-21, 1032212280}
    
    
    
    
    
% Pygments definitions
\makeatletter
\def\PY@reset{\let\PY@it=\relax \let\PY@bf=\relax%
    \let\PY@ul=\relax \let\PY@tc=\relax%
    \let\PY@bc=\relax \let\PY@ff=\relax}
\def\PY@tok#1{\csname PY@tok@#1\endcsname}
\def\PY@toks#1+{\ifx\relax#1\empty\else%
    \PY@tok{#1}\expandafter\PY@toks\fi}
\def\PY@do#1{\PY@bc{\PY@tc{\PY@ul{%
    \PY@it{\PY@bf{\PY@ff{#1}}}}}}}
\def\PY#1#2{\PY@reset\PY@toks#1+\relax+\PY@do{#2}}

\@namedef{PY@tok@w}{\def\PY@tc##1{\textcolor[rgb]{0.73,0.73,0.73}{##1}}}
\@namedef{PY@tok@c}{\let\PY@it=\textit\def\PY@tc##1{\textcolor[rgb]{0.25,0.50,0.50}{##1}}}
\@namedef{PY@tok@cp}{\def\PY@tc##1{\textcolor[rgb]{0.74,0.48,0.00}{##1}}}
\@namedef{PY@tok@k}{\let\PY@bf=\textbf\def\PY@tc##1{\textcolor[rgb]{0.00,0.50,0.00}{##1}}}
\@namedef{PY@tok@kp}{\def\PY@tc##1{\textcolor[rgb]{0.00,0.50,0.00}{##1}}}
\@namedef{PY@tok@kt}{\def\PY@tc##1{\textcolor[rgb]{0.69,0.00,0.25}{##1}}}
\@namedef{PY@tok@o}{\def\PY@tc##1{\textcolor[rgb]{0.40,0.40,0.40}{##1}}}
\@namedef{PY@tok@ow}{\let\PY@bf=\textbf\def\PY@tc##1{\textcolor[rgb]{0.67,0.13,1.00}{##1}}}
\@namedef{PY@tok@nb}{\def\PY@tc##1{\textcolor[rgb]{0.00,0.50,0.00}{##1}}}
\@namedef{PY@tok@nf}{\def\PY@tc##1{\textcolor[rgb]{0.00,0.00,1.00}{##1}}}
\@namedef{PY@tok@nc}{\let\PY@bf=\textbf\def\PY@tc##1{\textcolor[rgb]{0.00,0.00,1.00}{##1}}}
\@namedef{PY@tok@nn}{\let\PY@bf=\textbf\def\PY@tc##1{\textcolor[rgb]{0.00,0.00,1.00}{##1}}}
\@namedef{PY@tok@ne}{\let\PY@bf=\textbf\def\PY@tc##1{\textcolor[rgb]{0.82,0.25,0.23}{##1}}}
\@namedef{PY@tok@nv}{\def\PY@tc##1{\textcolor[rgb]{0.10,0.09,0.49}{##1}}}
\@namedef{PY@tok@no}{\def\PY@tc##1{\textcolor[rgb]{0.53,0.00,0.00}{##1}}}
\@namedef{PY@tok@nl}{\def\PY@tc##1{\textcolor[rgb]{0.63,0.63,0.00}{##1}}}
\@namedef{PY@tok@ni}{\let\PY@bf=\textbf\def\PY@tc##1{\textcolor[rgb]{0.60,0.60,0.60}{##1}}}
\@namedef{PY@tok@na}{\def\PY@tc##1{\textcolor[rgb]{0.49,0.56,0.16}{##1}}}
\@namedef{PY@tok@nt}{\let\PY@bf=\textbf\def\PY@tc##1{\textcolor[rgb]{0.00,0.50,0.00}{##1}}}
\@namedef{PY@tok@nd}{\def\PY@tc##1{\textcolor[rgb]{0.67,0.13,1.00}{##1}}}
\@namedef{PY@tok@s}{\def\PY@tc##1{\textcolor[rgb]{0.73,0.13,0.13}{##1}}}
\@namedef{PY@tok@sd}{\let\PY@it=\textit\def\PY@tc##1{\textcolor[rgb]{0.73,0.13,0.13}{##1}}}
\@namedef{PY@tok@si}{\let\PY@bf=\textbf\def\PY@tc##1{\textcolor[rgb]{0.73,0.40,0.53}{##1}}}
\@namedef{PY@tok@se}{\let\PY@bf=\textbf\def\PY@tc##1{\textcolor[rgb]{0.73,0.40,0.13}{##1}}}
\@namedef{PY@tok@sr}{\def\PY@tc##1{\textcolor[rgb]{0.73,0.40,0.53}{##1}}}
\@namedef{PY@tok@ss}{\def\PY@tc##1{\textcolor[rgb]{0.10,0.09,0.49}{##1}}}
\@namedef{PY@tok@sx}{\def\PY@tc##1{\textcolor[rgb]{0.00,0.50,0.00}{##1}}}
\@namedef{PY@tok@m}{\def\PY@tc##1{\textcolor[rgb]{0.40,0.40,0.40}{##1}}}
\@namedef{PY@tok@gh}{\let\PY@bf=\textbf\def\PY@tc##1{\textcolor[rgb]{0.00,0.00,0.50}{##1}}}
\@namedef{PY@tok@gu}{\let\PY@bf=\textbf\def\PY@tc##1{\textcolor[rgb]{0.50,0.00,0.50}{##1}}}
\@namedef{PY@tok@gd}{\def\PY@tc##1{\textcolor[rgb]{0.63,0.00,0.00}{##1}}}
\@namedef{PY@tok@gi}{\def\PY@tc##1{\textcolor[rgb]{0.00,0.63,0.00}{##1}}}
\@namedef{PY@tok@gr}{\def\PY@tc##1{\textcolor[rgb]{1.00,0.00,0.00}{##1}}}
\@namedef{PY@tok@ge}{\let\PY@it=\textit}
\@namedef{PY@tok@gs}{\let\PY@bf=\textbf}
\@namedef{PY@tok@gp}{\let\PY@bf=\textbf\def\PY@tc##1{\textcolor[rgb]{0.00,0.00,0.50}{##1}}}
\@namedef{PY@tok@go}{\def\PY@tc##1{\textcolor[rgb]{0.53,0.53,0.53}{##1}}}
\@namedef{PY@tok@gt}{\def\PY@tc##1{\textcolor[rgb]{0.00,0.27,0.87}{##1}}}
\@namedef{PY@tok@err}{\def\PY@bc##1{{\setlength{\fboxsep}{\string -\fboxrule}\fcolorbox[rgb]{1.00,0.00,0.00}{1,1,1}{\strut ##1}}}}
\@namedef{PY@tok@kc}{\let\PY@bf=\textbf\def\PY@tc##1{\textcolor[rgb]{0.00,0.50,0.00}{##1}}}
\@namedef{PY@tok@kd}{\let\PY@bf=\textbf\def\PY@tc##1{\textcolor[rgb]{0.00,0.50,0.00}{##1}}}
\@namedef{PY@tok@kn}{\let\PY@bf=\textbf\def\PY@tc##1{\textcolor[rgb]{0.00,0.50,0.00}{##1}}}
\@namedef{PY@tok@kr}{\let\PY@bf=\textbf\def\PY@tc##1{\textcolor[rgb]{0.00,0.50,0.00}{##1}}}
\@namedef{PY@tok@bp}{\def\PY@tc##1{\textcolor[rgb]{0.00,0.50,0.00}{##1}}}
\@namedef{PY@tok@fm}{\def\PY@tc##1{\textcolor[rgb]{0.00,0.00,1.00}{##1}}}
\@namedef{PY@tok@vc}{\def\PY@tc##1{\textcolor[rgb]{0.10,0.09,0.49}{##1}}}
\@namedef{PY@tok@vg}{\def\PY@tc##1{\textcolor[rgb]{0.10,0.09,0.49}{##1}}}
\@namedef{PY@tok@vi}{\def\PY@tc##1{\textcolor[rgb]{0.10,0.09,0.49}{##1}}}
\@namedef{PY@tok@vm}{\def\PY@tc##1{\textcolor[rgb]{0.10,0.09,0.49}{##1}}}
\@namedef{PY@tok@sa}{\def\PY@tc##1{\textcolor[rgb]{0.73,0.13,0.13}{##1}}}
\@namedef{PY@tok@sb}{\def\PY@tc##1{\textcolor[rgb]{0.73,0.13,0.13}{##1}}}
\@namedef{PY@tok@sc}{\def\PY@tc##1{\textcolor[rgb]{0.73,0.13,0.13}{##1}}}
\@namedef{PY@tok@dl}{\def\PY@tc##1{\textcolor[rgb]{0.73,0.13,0.13}{##1}}}
\@namedef{PY@tok@s2}{\def\PY@tc##1{\textcolor[rgb]{0.73,0.13,0.13}{##1}}}
\@namedef{PY@tok@sh}{\def\PY@tc##1{\textcolor[rgb]{0.73,0.13,0.13}{##1}}}
\@namedef{PY@tok@s1}{\def\PY@tc##1{\textcolor[rgb]{0.73,0.13,0.13}{##1}}}
\@namedef{PY@tok@mb}{\def\PY@tc##1{\textcolor[rgb]{0.40,0.40,0.40}{##1}}}
\@namedef{PY@tok@mf}{\def\PY@tc##1{\textcolor[rgb]{0.40,0.40,0.40}{##1}}}
\@namedef{PY@tok@mh}{\def\PY@tc##1{\textcolor[rgb]{0.40,0.40,0.40}{##1}}}
\@namedef{PY@tok@mi}{\def\PY@tc##1{\textcolor[rgb]{0.40,0.40,0.40}{##1}}}
\@namedef{PY@tok@il}{\def\PY@tc##1{\textcolor[rgb]{0.40,0.40,0.40}{##1}}}
\@namedef{PY@tok@mo}{\def\PY@tc##1{\textcolor[rgb]{0.40,0.40,0.40}{##1}}}
\@namedef{PY@tok@ch}{\let\PY@it=\textit\def\PY@tc##1{\textcolor[rgb]{0.25,0.50,0.50}{##1}}}
\@namedef{PY@tok@cm}{\let\PY@it=\textit\def\PY@tc##1{\textcolor[rgb]{0.25,0.50,0.50}{##1}}}
\@namedef{PY@tok@cpf}{\let\PY@it=\textit\def\PY@tc##1{\textcolor[rgb]{0.25,0.50,0.50}{##1}}}
\@namedef{PY@tok@c1}{\let\PY@it=\textit\def\PY@tc##1{\textcolor[rgb]{0.25,0.50,0.50}{##1}}}
\@namedef{PY@tok@cs}{\let\PY@it=\textit\def\PY@tc##1{\textcolor[rgb]{0.25,0.50,0.50}{##1}}}

\def\PYZbs{\char`\\}
\def\PYZus{\char`\_}
\def\PYZob{\char`\{}
\def\PYZcb{\char`\}}
\def\PYZca{\char`\^}
\def\PYZam{\char`\&}
\def\PYZlt{\char`\<}
\def\PYZgt{\char`\>}
\def\PYZsh{\char`\#}
\def\PYZpc{\char`\%}
\def\PYZdl{\char`\$}
\def\PYZhy{\char`\-}
\def\PYZsq{\char`\'}
\def\PYZdq{\char`\"}
\def\PYZti{\char`\~}
% for compatibility with earlier versions
\def\PYZat{@}
\def\PYZlb{[}
\def\PYZrb{]}
\makeatother


    % For linebreaks inside Verbatim environment from package fancyvrb.
    \makeatletter
        \newbox\Wrappedcontinuationbox
        \newbox\Wrappedvisiblespacebox
        \newcommand*\Wrappedvisiblespace {\textcolor{red}{\textvisiblespace}}
        \newcommand*\Wrappedcontinuationsymbol {\textcolor{red}{\llap{\tiny$\m@th\hookrightarrow$}}}
        \newcommand*\Wrappedcontinuationindent {3ex }
        \newcommand*\Wrappedafterbreak {\kern\Wrappedcontinuationindent\copy\Wrappedcontinuationbox}
        % Take advantage of the already applied Pygments mark-up to insert
        % potential linebreaks for TeX processing.
        %        {, <, #, %, $, ' and ": go to next line.
        %        _, }, ^, &, >, - and ~: stay at end of broken line.
        % Use of \textquotesingle for straight quote.
        \newcommand*\Wrappedbreaksatspecials {%
            \def\PYGZus{\discretionary{\char`\_}{\Wrappedafterbreak}{\char`\_}}%
            \def\PYGZob{\discretionary{}{\Wrappedafterbreak\char`\{}{\char`\{}}%
            \def\PYGZcb{\discretionary{\char`\}}{\Wrappedafterbreak}{\char`\}}}%
            \def\PYGZca{\discretionary{\char`\^}{\Wrappedafterbreak}{\char`\^}}%
            \def\PYGZam{\discretionary{\char`\&}{\Wrappedafterbreak}{\char`\&}}%
            \def\PYGZlt{\discretionary{}{\Wrappedafterbreak\char`\<}{\char`\<}}%
            \def\PYGZgt{\discretionary{\char`\>}{\Wrappedafterbreak}{\char`\>}}%
            \def\PYGZsh{\discretionary{}{\Wrappedafterbreak\char`\#}{\char`\#}}%
            \def\PYGZpc{\discretionary{}{\Wrappedafterbreak\char`\%}{\char`\%}}%
            \def\PYGZdl{\discretionary{}{\Wrappedafterbreak\char`\$}{\char`\$}}%
            \def\PYGZhy{\discretionary{\char`\-}{\Wrappedafterbreak}{\char`\-}}%
            \def\PYGZsq{\discretionary{}{\Wrappedafterbreak\textquotesingle}{\textquotesingle}}%
            \def\PYGZdq{\discretionary{}{\Wrappedafterbreak\char`\"}{\char`\"}}%
            \def\PYGZti{\discretionary{\char`\~}{\Wrappedafterbreak}{\char`\~}}%
        }
        % Some characters . , ; ? ! / are not pygmentized.
        % This macro makes them "active" and they will insert potential linebreaks
        \newcommand*\Wrappedbreaksatpunct {%
            \lccode`\~`\.\lowercase{\def~}{\discretionary{\hbox{\char`\.}}{\Wrappedafterbreak}{\hbox{\char`\.}}}%
            \lccode`\~`\,\lowercase{\def~}{\discretionary{\hbox{\char`\,}}{\Wrappedafterbreak}{\hbox{\char`\,}}}%
            \lccode`\~`\;\lowercase{\def~}{\discretionary{\hbox{\char`\;}}{\Wrappedafterbreak}{\hbox{\char`\;}}}%
            \lccode`\~`\:\lowercase{\def~}{\discretionary{\hbox{\char`\:}}{\Wrappedafterbreak}{\hbox{\char`\:}}}%
            \lccode`\~`\?\lowercase{\def~}{\discretionary{\hbox{\char`\?}}{\Wrappedafterbreak}{\hbox{\char`\?}}}%
            \lccode`\~`\!\lowercase{\def~}{\discretionary{\hbox{\char`\!}}{\Wrappedafterbreak}{\hbox{\char`\!}}}%
            \lccode`\~`\/\lowercase{\def~}{\discretionary{\hbox{\char`\/}}{\Wrappedafterbreak}{\hbox{\char`\/}}}%
            \catcode`\.\active
            \catcode`\,\active
            \catcode`\;\active
            \catcode`\:\active
            \catcode`\?\active
            \catcode`\!\active
            \catcode`\/\active
            \lccode`\~`\~
        }
    \makeatother

    \let\OriginalVerbatim=\Verbatim
    \makeatletter
    \renewcommand{\Verbatim}[1][1]{%
        %\parskip\z@skip
        \sbox\Wrappedcontinuationbox {\Wrappedcontinuationsymbol}%
        \sbox\Wrappedvisiblespacebox {\FV@SetupFont\Wrappedvisiblespace}%
        \def\FancyVerbFormatLine ##1{\hsize\linewidth
            \vtop{\raggedright\hyphenpenalty\z@\exhyphenpenalty\z@
                \doublehyphendemerits\z@\finalhyphendemerits\z@
                \strut ##1\strut}%
        }%
        % If the linebreak is at a space, the latter will be displayed as visible
        % space at end of first line, and a continuation symbol starts next line.
        % Stretch/shrink are however usually zero for typewriter font.
        \def\FV@Space {%
            \nobreak\hskip\z@ plus\fontdimen3\font minus\fontdimen4\font
            \discretionary{\copy\Wrappedvisiblespacebox}{\Wrappedafterbreak}
            {\kern\fontdimen2\font}%
        }%

        % Allow breaks at special characters using \PYG... macros.
        \Wrappedbreaksatspecials
        % Breaks at punctuation characters . , ; ? ! and / need catcode=\active
        \OriginalVerbatim[#1,codes*=\Wrappedbreaksatpunct]%
    }
    \makeatother

    % Exact colors from NB
    \definecolor{incolor}{HTML}{303F9F}
    \definecolor{outcolor}{HTML}{D84315}
    \definecolor{cellborder}{HTML}{CFCFCF}
    \definecolor{cellbackground}{HTML}{F7F7F7}

    % prompt
    \makeatletter
    \newcommand{\boxspacing}{\kern\kvtcb@left@rule\kern\kvtcb@boxsep}
    \makeatother
    \newcommand{\prompt}[4]{
        {\ttfamily\llap{{\color{#2}[#3]:\hspace{3pt}#4}}\vspace{-\baselineskip}}
    }
    

    
    % Prevent overflowing lines due to hard-to-break entities
    \sloppy
    % Setup hyperref package
    \hypersetup{
      breaklinks=true,  % so long urls are correctly broken across lines
      colorlinks=true,
      urlcolor=urlcolor,
      linkcolor=linkcolor,
      citecolor=citecolor,
      }
    % Slightly bigger margins than the latex defaults
    
    \geometry{verbose,tmargin=1in,bmargin=1in,lmargin=1in,rmargin=1in}
        

\begin{document}
    
    \maketitle
    
    


    В данной лабораторной работе требуется выполнить симуляции методом
Монте-Карло.

Этот отчет представлен как экспорт Jupyter-блокнота
(https://jupyter.org), использующего ядро evcxr
(https://github.com/evcxr/evcxr).

    \begin{tcolorbox}[breakable, size=fbox, boxrule=1pt, pad at break*=1mm,colback=cellbackground, colframe=cellborder]
\prompt{In}{incolor}{2}{\boxspacing}
\begin{Verbatim}[commandchars=\\\{\}]
\PY{c+c1}{// Эти команды проверяют функционирование ядра evcxr,}
\PY{c+c1}{// а также включают информацию о времени выполнения ячейки}
\PY{c+c1}{// и кеширование артифактов компиляции.}
:\PY{n+nc}{timing}\PY{+w}{ }\PY{l+m+mi}{1}
:\PY{n+nc}{sccache}\PY{+w}{ }\PY{l+m+mi}{1}
\end{Verbatim}
\end{tcolorbox}

            \begin{tcolorbox}[breakable, size=fbox, boxrule=.5pt, pad at break*=1mm, opacityfill=0]
\prompt{Out}{outcolor}{2}{\boxspacing}
\begin{Verbatim}[commandchars=\\\{\}]
Timing: true
sccache: true

\end{Verbatim}
\end{tcolorbox}
        
    \begin{tcolorbox}[breakable, size=fbox, boxrule=1pt, pad at break*=1mm,colback=cellbackground, colframe=cellborder]
\prompt{In}{incolor}{3}{\boxspacing}
\begin{Verbatim}[commandchars=\\\{\}]
:\PY{n+nc}{timing}\PY{+w}{ }\PY{l+m+mi}{1}
:\PY{n+nc}{sccache}\PY{+w}{ }\PY{l+m+mi}{1}

\PY{c+c1}{// Здесь мы описываем все зависимости для выполнения ячеек этого блокнота.}
\PY{c+c1}{// Выполнение этой ячейки скачает и соберет все зависимости заранее.}
:\PY{n+nc}{dep}\PY{+w}{ }\PY{n}{rand}\PY{+w}{ }\PY{o}{=}\PY{+w}{ }\PY{l+s}{\PYZdq{}}\PY{l+s}{0.8.5}\PY{l+s}{\PYZdq{}}
:\PY{n+nc}{dep}\PY{+w}{ }\PY{n}{itertools}\PY{+w}{ }\PY{o}{=}\PY{+w}{ }\PY{l+s}{\PYZdq{}}\PY{l+s}{0.10.5}\PY{l+s}{\PYZdq{}}
:\PY{n+nc}{dep}\PY{+w}{ }\PY{n}{rand\PYZus{}distr}\PY{+w}{ }\PY{o}{=}\PY{+w}{ }\PY{l+s}{\PYZdq{}}\PY{l+s}{0.4.3}\PY{l+s}{\PYZdq{}}
:\PY{n+nc}{dep}\PY{+w}{ }\PY{n}{plotters}\PY{+w}{ }\PY{o}{=}\PY{+w}{ }\PY{p}{\PYZob{}}\PY{+w}{ }\PY{n}{version}\PY{o}{=}\PY{l+s}{\PYZdq{}}\PY{l+s}{0.3.4}\PY{l+s}{\PYZdq{}}\PY{p}{,}\PY{+w}{ }\PY{n}{default\PYZus{}features}\PY{+w}{ }\PY{o}{=}\PY{+w}{ }\PY{k+kc}{false}\PY{p}{,}\PY{+w}{ }\PY{n}{features}\PY{+w}{ }\PY{o}{=}\PY{+w}{ }\PY{p}{[}\PY{l+s}{\PYZdq{}}\PY{l+s}{evcxr}\PY{l+s}{\PYZdq{}}\PY{p}{,}\PY{+w}{ }\PY{l+s}{\PYZdq{}}\PY{l+s}{all\PYZus{}series}\PY{l+s}{\PYZdq{}}\PY{p}{]}\PY{+w}{ }\PY{p}{\PYZcb{}}
\end{Verbatim}
\end{tcolorbox}

            \begin{tcolorbox}[breakable, size=fbox, boxrule=.5pt, pad at break*=1mm, opacityfill=0]
\prompt{Out}{outcolor}{3}{\boxspacing}
\begin{Verbatim}[commandchars=\\\{\}]
Timing: false
sccache: true

\end{Verbatim}
\end{tcolorbox}
        
    \begin{tcolorbox}[breakable, size=fbox, boxrule=1pt, pad at break*=1mm,colback=cellbackground, colframe=cellborder]
\prompt{In}{incolor}{4}{\boxspacing}
\begin{Verbatim}[commandchars=\\\{\}]
:\PY{n+nc}{timing}\PY{+w}{ }\PY{l+m+mi}{1}
:\PY{n+nc}{sccache}\PY{+w}{ }\PY{l+m+mi}{1}
:\PY{n+nc}{dep}\PY{+w}{ }\PY{n}{rand}\PY{+w}{ }\PY{o}{=}\PY{+w}{ }\PY{l+s}{\PYZdq{}}\PY{l+s}{0.8.5}\PY{l+s}{\PYZdq{}}

\PY{l+s+sd}{/// Мы описываем интерфейс, который будут реализовывать случайные опыты.}
\PY{l+s+sd}{/// Он состоит из одного типа и нескольких функций.}
\PY{k}{trait}\PY{+w}{ }\PY{n}{Experiment}\PY{+w}{ }\PY{p}{\PYZob{}}
\PY{+w}{    }\PY{l+s+sd}{/// Тип, описывающий исход выполнения одного опыта.}
\PY{+w}{    }\PY{k}{type} \PY{n+nc}{Outcome}: \PY{n+nb}{Ord}\PY{o}{+}\PY{n+nb}{Eq}\PY{o}{+}\PY{n}{std}::\PY{n}{hash}::\PY{n}{Hash}\PY{o}{+}\PY{n+nb}{Sized}\PY{p}{;}
\PY{+w}{    }
\PY{+w}{    }\PY{l+s+sd}{/// Метод, возвращающий список исходов, которые считаются желаемыми в опыте.}
\PY{+w}{    }\PY{k}{fn} \PY{n+nf}{desired\PYZus{}outcomes}\PY{p}{(}\PY{o}{\PYZam{}}\PY{n+nb+bp}{self}\PY{p}{)}\PY{+w}{ }\PYZhy{}\PYZgt{} \PY{n+nb}{Vec}\PY{o}{\PYZlt{}}\PY{n+nb+bp}{Self}::\PY{n}{Outcome}\PY{o}{\PYZgt{}}\PY{p}{;}
\PY{+w}{    }
\PY{+w}{    }\PY{l+s+sd}{/// Метод, который принимает экземпляр генератора случайных чисел}
\PY{+w}{    }\PY{l+s+sd}{/// и выдает результат выполнения одного опыта.}
\PY{+w}{    }\PY{k}{fn} \PY{n+nf}{try\PYZus{}it}\PY{o}{\PYZlt{}}\PY{n}{T}: \PY{n+nc}{rand}::\PY{n}{Rng}\PY{o}{\PYZgt{}}\PY{p}{(}\PY{o}{\PYZam{}}\PY{n+nb+bp}{self}\PY{p}{,}\PY{+w}{ }\PY{n}{rng}: \PY{k+kp}{\PYZam{}}\PY{n+nc}{mut}\PY{+w}{ }\PY{n}{T}\PY{p}{)}\PY{+w}{ }\PYZhy{}\PYZgt{} \PY{n+nc}{Self}::\PY{n}{Outcome}\PY{p}{;}

\PY{+w}{    }\PY{l+s+sd}{/// Метод, который проводит много опытов и составляет}
\PY{+w}{    }\PY{l+s+sd}{/// гистограмму исходов с количеством того. сколько раз они встречались.}
\PY{+w}{    }\PY{k}{fn} \PY{n+nf}{collect\PYZus{}stats}\PY{p}{(}\PY{o}{\PYZam{}}\PY{n+nb+bp}{self}\PY{p}{,}\PY{+w}{ }\PY{n}{trials}: \PY{k+kt}{usize}\PY{p}{)}\PY{+w}{ }\PYZhy{}\PYZgt{} \PY{n+nc}{std}::\PY{n}{collections}::\PY{n}{HashMap}\PY{o}{\PYZlt{}}\PY{n+nb+bp}{Self}::\PY{n}{Outcome}\PY{p}{,}\PY{+w}{ }\PY{k+kt}{usize}\PY{o}{\PYZgt{}}\PY{+w}{ }\PY{p}{\PYZob{}}
\PY{+w}{        }\PY{k+kd}{let}\PY{+w}{ }\PY{k}{mut}\PY{+w}{ }\PY{n}{outcomes}\PY{+w}{ }\PY{o}{=}\PY{+w}{ }\PY{n}{std}::\PY{n}{collections}::\PY{n}{HashMap}::\PY{n}{new}\PY{p}{(}\PY{p}{)}\PY{p}{;}
\PY{+w}{        }\PY{k+kd}{let}\PY{+w}{ }\PY{k}{mut}\PY{+w}{ }\PY{n}{rng}\PY{+w}{ }\PY{o}{=}\PY{+w}{ }\PY{n}{rand}::\PY{n}{thread\PYZus{}rng}\PY{p}{(}\PY{p}{)}\PY{p}{;}
\PY{+w}{        }\PY{k}{for}\PY{+w}{ }\PY{n}{\PYZus{}}\PY{+w}{ }\PY{k}{in}\PY{+w}{ }\PY{l+m+mi}{0}\PY{o}{..}\PY{n}{trials}\PY{+w}{ }\PY{p}{\PYZob{}}
\PY{+w}{            }\PY{c+c1}{// Для каждого раза опыта, проводим один опыт, и увеличиваем его счетчик в гистограмме.}
\PY{+w}{            }\PY{k+kd}{let}\PY{+w}{ }\PY{n}{outcome}\PY{+w}{ }\PY{o}{=}\PY{+w}{ }\PY{o}{\PYZlt{}}\PY{n+nb+bp}{Self}\PY{+w}{ }\PY{k}{as}\PY{+w}{ }\PY{n}{Experiment}\PY{o}{\PYZgt{}}::\PY{n}{try\PYZus{}it}\PY{p}{(}\PY{o}{\PYZam{}}\PY{n+nb+bp}{self}\PY{p}{,}\PY{+w}{ }\PY{o}{\PYZam{}}\PY{k}{mut}\PY{+w}{ }\PY{n}{rng}\PY{p}{)}\PY{p}{;}
\PY{+w}{            }\PY{n}{outcomes}\PY{p}{.}\PY{n}{entry}\PY{p}{(}\PY{n}{outcome}\PY{p}{)}\PY{p}{.}\PY{n}{and\PYZus{}modify}\PY{p}{(}\PY{o}{|}\PY{n}{i}\PY{o}{|}\PY{+w}{ }\PY{o}{*}\PY{n}{i}\PY{+w}{ }\PY{o}{+}\PY{o}{=}\PY{+w}{ }\PY{l+m+mi}{1}\PY{p}{)}\PY{p}{.}\PY{n}{or\PYZus{}insert}\PY{p}{(}\PY{l+m+mi}{1}\PY{p}{)}\PY{p}{;}
\PY{+w}{        }\PY{p}{\PYZcb{}}
\PY{+w}{        }\PY{n}{outcomes}
\PY{+w}{    }\PY{p}{\PYZcb{}}
\PY{+w}{    }
\PY{+w}{    }\PY{l+s+sd}{/// Метод, который считает вероятность того, что исход опыта будет одним из данных.}
\PY{+w}{    }\PY{k}{fn} \PY{n+nf}{probability\PYZus{}of\PYZus{}outcomes}\PY{p}{(}\PY{o}{\PYZam{}}\PY{n+nb+bp}{self}\PY{p}{,}\PY{+w}{ }\PY{n}{trials}: \PY{k+kt}{usize}\PY{p}{,}\PY{+w}{ }\PY{n}{desired\PYZus{}outcomes}: \PY{k+kp}{\PYZam{}}\PY{p}{[}\PY{n+nb+bp}{Self}::\PY{n}{Outcome}\PY{p}{]}\PY{p}{)}\PY{+w}{ }\PYZhy{}\PYZgt{} \PY{k+kt}{f64} \PY{p}{\PYZob{}}
\PY{+w}{        }\PY{k+kd}{let}\PY{+w}{ }\PY{n}{outcomes}\PY{+w}{ }\PY{o}{=}\PY{+w}{ }\PY{n+nb+bp}{self}\PY{p}{.}\PY{n}{collect\PYZus{}stats}\PY{p}{(}\PY{n}{trials}\PY{p}{)}\PY{p}{;}
\PY{+w}{        }\PY{k+kd}{let}\PY{+w}{ }\PY{k}{mut}\PY{+w}{ }\PY{n}{ok\PYZus{}outcomes}\PY{+w}{ }\PY{o}{=}\PY{+w}{ }\PY{l+m+mf}{0.0}\PY{p}{;}
\PY{+w}{        }\PY{k}{for}\PY{+w}{ }\PY{n}{outcome}\PY{+w}{ }\PY{k}{in}\PY{+w}{ }\PY{n}{desired\PYZus{}outcomes}\PY{+w}{ }\PY{p}{\PYZob{}}
\PY{+w}{            }\PY{n}{ok\PYZus{}outcomes}\PY{+w}{ }\PY{o}{+}\PY{o}{=}\PY{+w}{ }\PY{p}{(}\PY{o}{*}\PY{n}{outcomes}\PY{p}{.}\PY{n}{get}\PY{p}{(}\PY{o}{\PYZam{}}\PY{n}{outcome}\PY{p}{)}\PY{p}{.}\PY{n}{unwrap\PYZus{}or}\PY{p}{(}\PY{o}{\PYZam{}}\PY{l+m+mi}{0}\PY{p}{)}\PY{+w}{ }\PY{k}{as}\PY{+w}{ }\PY{k+kt}{f64}\PY{p}{)}\PY{p}{;}
\PY{+w}{        }\PY{p}{\PYZcb{}}
\PY{+w}{        }\PY{n}{ok\PYZus{}outcomes}\PY{+w}{ }\PY{o}{/}\PY{+w}{ }\PY{p}{(}\PY{n}{trials}\PY{+w}{ }\PY{k}{as}\PY{+w}{ }\PY{k+kt}{f64}\PY{p}{)}
\PY{+w}{    }\PY{p}{\PYZcb{}}
\PY{+w}{    }
\PY{+w}{    }\PY{l+s+sd}{/// Метод, который считает вероятность того, что исход опыта будет одним из желаемых.}
\PY{+w}{    }\PY{k}{fn} \PY{n+nf}{probability\PYZus{}of\PYZus{}desired}\PY{p}{(}\PY{o}{\PYZam{}}\PY{n+nb+bp}{self}\PY{p}{,}\PY{+w}{ }\PY{n}{trials}: \PY{k+kt}{usize}\PY{p}{)}\PY{+w}{ }\PYZhy{}\PYZgt{} \PY{k+kt}{f64} \PY{p}{\PYZob{}}
\PY{+w}{        }\PY{n+nb+bp}{self}\PY{p}{.}\PY{n}{probability\PYZus{}of\PYZus{}outcomes}\PY{p}{(}\PY{n}{trials}\PY{p}{,}\PY{+w}{ }\PY{o}{\PYZam{}}\PY{o}{\PYZlt{}}\PY{n+nb+bp}{Self}\PY{+w}{ }\PY{k}{as}\PY{+w}{ }\PY{n}{Experiment}\PY{o}{\PYZgt{}}::\PY{n}{desired\PYZus{}outcomes}\PY{p}{(}\PY{o}{\PYZam{}}\PY{n+nb+bp}{self}\PY{p}{)}\PY{p}{)}
\PY{+w}{    }\PY{p}{\PYZcb{}}
\PY{+w}{    }
\PY{+w}{    }\PY{l+s+sd}{/// Описание опыта в виде строки.}
\PY{+w}{    }\PY{k}{fn} \PY{n+nf}{description}\PY{p}{(}\PY{o}{\PYZam{}}\PY{n+nb+bp}{self}\PY{p}{)}\PY{+w}{ }\PYZhy{}\PYZgt{} \PY{n+nb}{String}\PY{p}{;}
\PY{p}{\PYZcb{}}

\PY{l+s+sd}{/// Интерфейс для опытов, которые возвращают непрерывные значения,}
\PY{l+s+sd}{/// которые нельзя сравнивать напрямую.}
\PY{k}{trait}\PY{+w}{ }\PY{n}{PartialExperiment}\PY{+w}{ }\PY{p}{\PYZob{}}
\PY{+w}{    }\PY{k}{type} \PY{n+nc}{Outcome}: \PY{n+nb}{PartialOrd}\PY{o}{+}\PY{n+nb}{PartialEq}\PY{o}{+}\PY{n+nb}{Sized}\PY{p}{;}
\PY{+w}{    }
\PY{+w}{    }\PY{k}{fn} \PY{n+nf}{try\PYZus{}it}\PY{o}{\PYZlt{}}\PY{n}{T}: \PY{n+nc}{rand}::\PY{n}{Rng}\PY{o}{\PYZgt{}}\PY{p}{(}\PY{o}{\PYZam{}}\PY{n+nb+bp}{self}\PY{p}{,}\PY{+w}{ }\PY{n}{rng}: \PY{k+kp}{\PYZam{}}\PY{n+nc}{mut}\PY{+w}{ }\PY{n}{T}\PY{p}{)}\PY{+w}{ }\PYZhy{}\PYZgt{} \PY{n+nc}{Self}::\PY{n}{Outcome}\PY{p}{;}
\PY{+w}{    }
\PY{+w}{    }\PY{l+s+sd}{/// Метод, который проводит опыт много раз и возвращает список}
\PY{+w}{    }\PY{l+s+sd}{/// исходов, которые были получены.}
\PY{+w}{    }\PY{k}{fn} \PY{n+nf}{collect\PYZus{}stats}\PY{p}{(}\PY{o}{\PYZam{}}\PY{n+nb+bp}{self}\PY{p}{,}\PY{+w}{ }\PY{n}{trials}: \PY{k+kt}{usize}\PY{p}{)}\PY{+w}{ }\PYZhy{}\PYZgt{} \PY{n+nb}{Vec}\PY{o}{\PYZlt{}}\PY{n+nb+bp}{Self}::\PY{n}{Outcome}\PY{o}{\PYZgt{}}\PY{+w}{ }\PY{p}{\PYZob{}}
\PY{+w}{        }\PY{k+kd}{let}\PY{+w}{ }\PY{k}{mut}\PY{+w}{ }\PY{n}{outcomes}\PY{+w}{ }\PY{o}{=}\PY{+w}{ }\PY{n+nb}{Vec}::\PY{n}{with\PYZus{}capacity}\PY{p}{(}\PY{n}{trials}\PY{p}{)}\PY{p}{;}
\PY{+w}{        }\PY{k+kd}{let}\PY{+w}{ }\PY{k}{mut}\PY{+w}{ }\PY{n}{rng}\PY{+w}{ }\PY{o}{=}\PY{+w}{ }\PY{n}{rand}::\PY{n}{thread\PYZus{}rng}\PY{p}{(}\PY{p}{)}\PY{p}{;}
\PY{+w}{        }\PY{k}{for}\PY{+w}{ }\PY{n}{\PYZus{}}\PY{+w}{ }\PY{k}{in}\PY{+w}{ }\PY{l+m+mi}{0}\PY{o}{..}\PY{n}{trials}\PY{+w}{ }\PY{p}{\PYZob{}}
\PY{+w}{            }\PY{k+kd}{let}\PY{+w}{ }\PY{n}{outcome}\PY{+w}{ }\PY{o}{=}\PY{+w}{ }\PY{o}{\PYZlt{}}\PY{n+nb+bp}{Self}\PY{+w}{ }\PY{k}{as}\PY{+w}{ }\PY{n}{PartialExperiment}\PY{o}{\PYZgt{}}::\PY{n}{try\PYZus{}it}\PY{p}{(}\PY{o}{\PYZam{}}\PY{n+nb+bp}{self}\PY{p}{,}\PY{+w}{ }\PY{o}{\PYZam{}}\PY{k}{mut}\PY{+w}{ }\PY{n}{rng}\PY{p}{)}\PY{p}{;}
\PY{+w}{            }\PY{n}{outcomes}\PY{p}{.}\PY{n}{push}\PY{p}{(}\PY{n}{outcome}\PY{p}{)}
\PY{+w}{        }\PY{p}{\PYZcb{}}
\PY{+w}{        }\PY{n}{outcomes}
\PY{+w}{    }\PY{p}{\PYZcb{}}
\PY{+w}{    }\PY{k}{fn} \PY{n+nf}{description}\PY{p}{(}\PY{o}{\PYZam{}}\PY{n+nb+bp}{self}\PY{p}{)}\PY{+w}{ }\PYZhy{}\PYZgt{} \PY{n+nb}{String}\PY{p}{;}
\PY{p}{\PYZcb{}}

\PY{l+s+sd}{/// Наконец, мы объявляем структуру, которая будет содержать ответы на задания.}
\PY{c+cp}{\PYZsh{}[}\PY{c+cp}{derive(Default, Debug)}\PY{c+cp}{]}
\PY{k}{struct} \PY{n+nc}{Answers}\PY{+w}{ }\PY{p}{\PYZob{}}
\PY{+w}{    }\PY{n}{task\PYZus{}1}: \PY{n+nb}{Option}\PY{o}{\PYZlt{}}\PY{k+kt}{f64}\PY{o}{\PYZgt{}}\PY{p}{,}
\PY{+w}{    }\PY{n}{task\PYZus{}2}: \PY{n+nb}{Option}\PY{o}{\PYZlt{}}\PY{k+kt}{f64}\PY{o}{\PYZgt{}}\PY{p}{,}
\PY{+w}{    }\PY{n}{task\PYZus{}3}: \PY{n+nb}{Option}\PY{o}{\PYZlt{}}\PY{k+kt}{f64}\PY{o}{\PYZgt{}}\PY{p}{,}
\PY{+w}{    }\PY{n}{task\PYZus{}4}: \PY{n+nb}{Option}\PY{o}{\PYZlt{}}\PY{k+kt}{f64}\PY{o}{\PYZgt{}}\PY{p}{,}
\PY{+w}{    }\PY{n}{task\PYZus{}5}: \PY{n+nb}{Option}\PY{o}{\PYZlt{}}\PY{k+kt}{f64}\PY{o}{\PYZgt{}}\PY{p}{,}
\PY{+w}{    }\PY{n}{task\PYZus{}6}: \PY{n+nb}{Option}\PY{o}{\PYZlt{}}\PY{k+kt}{f64}\PY{o}{\PYZgt{}}\PY{p}{,}
\PY{+w}{    }\PY{n}{task\PYZus{}7}: \PY{n+nb}{Option}\PY{o}{\PYZlt{}}\PY{k+kt}{u64}\PY{o}{\PYZgt{}}\PY{p}{,}
\PY{+w}{    }\PY{n}{task\PYZus{}8}: \PY{n+nb}{Option}\PY{o}{\PYZlt{}}\PY{p}{(}\PY{k+kt}{f64}\PY{p}{,}\PY{+w}{ }\PY{k+kt}{f64}\PY{p}{,}\PY{+w}{ }\PY{k+kt}{f64}\PY{p}{)}\PY{o}{\PYZgt{}}\PY{p}{,}
\PY{+w}{    }\PY{n}{task\PYZus{}9}: \PY{n+nb}{Option}\PY{o}{\PYZlt{}}\PY{k+kt}{f64}\PY{o}{\PYZgt{}}\PY{p}{,}
\PY{+w}{    }\PY{n}{task\PYZus{}10}: \PY{n+nb}{Option}\PY{o}{\PYZlt{}}\PY{k+kt}{f64}\PY{o}{\PYZgt{}}\PY{p}{,}
\PY{p}{\PYZcb{}}
\PY{k+kd}{let}\PY{+w}{ }\PY{k}{mut}\PY{+w}{ }\PY{n}{answers}\PY{+w}{ }\PY{o}{=}\PY{+w}{ }\PY{n}{Answers}::\PY{n}{default}\PY{p}{(}\PY{p}{)}\PY{p}{;}
\end{Verbatim}
\end{tcolorbox}

            \begin{tcolorbox}[breakable, size=fbox, boxrule=.5pt, pad at break*=1mm, opacityfill=0]
\prompt{Out}{outcolor}{4}{\boxspacing}
\begin{Verbatim}[commandchars=\\\{\}]
Timing: true
sccache: true

\end{Verbatim}
\end{tcolorbox}
        
    \begin{tcolorbox}[breakable, size=fbox, boxrule=1pt, pad at break*=1mm,colback=cellbackground, colframe=cellborder]
\prompt{In}{incolor}{5}{\boxspacing}
\begin{Verbatim}[commandchars=\\\{\}]
:\PY{n+nc}{timing}\PY{+w}{ }\PY{l+m+mi}{1}
:\PY{n+nc}{sccache}\PY{+w}{ }\PY{l+m+mi}{1}
:\PY{n+nc}{dep}\PY{+w}{ }\PY{n}{rand}\PY{+w}{ }\PY{o}{=}\PY{+w}{ }\PY{l+s}{\PYZdq{}}\PY{l+s}{0.8.5}\PY{l+s}{\PYZdq{}}
:\PY{n+nc}{dep}\PY{+w}{ }\PY{n}{itertools}\PY{+w}{ }\PY{o}{=}\PY{+w}{ }\PY{l+s}{\PYZdq{}}\PY{l+s}{0.10.5}\PY{l+s}{\PYZdq{}}

\PY{k}{use}\PY{+w}{ }\PY{n}{itertools}::\PY{n}{Itertools}\PY{p}{;}

\PY{k}{struct} \PY{n+nc}{Task1}\PY{p}{;}

\PY{l+s+sd}{/// В контексте данной задачи, вектор \PYZhy{}\PYZhy{} это пара двух действительных чисел.}
\PY{k}{type} \PY{n+nc}{Vec2}\PY{+w}{ }\PY{o}{=}\PY{+w}{ }\PY{p}{(}\PY{k+kt}{f64}\PY{p}{,}\PY{+w}{ }\PY{k+kt}{f64}\PY{p}{)}\PY{p}{;}

\PY{l+s+sd}{/// Вспомогательный метод считает длину вектора.}
\PY{k}{fn} \PY{n+nf}{magnitude}\PY{p}{(}\PY{n}{v}: \PY{k+kp}{\PYZam{}}\PY{n+nc}{Vec2}\PY{p}{)}\PY{+w}{ }\PYZhy{}\PYZgt{} \PY{k+kt}{f64} \PY{p}{\PYZob{}}
\PY{+w}{    }\PY{k+kd}{let}\PY{+w}{ }\PY{p}{(}\PY{n}{a}\PY{p}{,}\PY{+w}{ }\PY{n}{b}\PY{p}{)}\PY{+w}{ }\PY{o}{=}\PY{+w}{ }\PY{n}{v}\PY{p}{;}
\PY{+w}{    }\PY{p}{(}\PY{n}{a}\PY{o}{*}\PY{n}{a}\PY{+w}{ }\PY{o}{+}\PY{+w}{ }\PY{n}{b}\PY{o}{*}\PY{n}{b}\PY{p}{)}\PY{p}{.}\PY{n}{sqrt}\PY{p}{(}\PY{p}{)}
\PY{p}{\PYZcb{}}

\PY{k}{impl}\PY{+w}{ }\PY{n}{Experiment}\PY{+w}{ }\PY{k}{for}\PY{+w}{ }\PY{n}{Task1}\PY{+w}{ }\PY{p}{\PYZob{}}
\PY{+w}{    }\PY{k}{type} \PY{n+nc}{Outcome}\PY{+w}{ }\PY{o}{=}\PY{+w}{ }\PY{k+kt}{bool}\PY{p}{;}
\PY{+w}{    }\PY{k}{fn} \PY{n+nf}{description}\PY{p}{(}\PY{o}{\PYZam{}}\PY{n+nb+bp}{self}\PY{p}{)}\PY{+w}{ }\PYZhy{}\PYZgt{} \PY{n+nb}{String} \PY{p}{\PYZob{}}
\PY{+w}{        }\PY{l+s}{\PYZdq{}}\PY{l+s}{Чему равна вероятность того, что случайный треугольник, нарисованный внутри}
\PY{l+s}{квадрата со стороной 1, является тупоугольным?}\PY{l+s}{\PYZdq{}}\PY{p}{.}\PY{n}{to\PYZus{}string}\PY{p}{(}\PY{p}{)}
\PY{+w}{    }\PY{p}{\PYZcb{}}
\PY{+w}{    }\PY{k}{fn} \PY{n+nf}{try\PYZus{}it}\PY{o}{\PYZlt{}}\PY{n}{T}: \PY{n+nc}{rand}::\PY{n}{Rng}\PY{o}{\PYZgt{}}\PY{p}{(}\PY{o}{\PYZam{}}\PY{n+nb+bp}{self}\PY{p}{,}\PY{+w}{ }\PY{n}{rng}: \PY{k+kp}{\PYZam{}}\PY{n+nc}{mut}\PY{+w}{ }\PY{n}{T}\PY{p}{)}\PY{+w}{ }\PYZhy{}\PYZgt{} \PY{n+nc}{Self}::\PY{n}{Outcome}\PY{+w}{ }\PY{p}{\PYZob{}}
\PY{+w}{        }\PY{c+c1}{// Генерируем три пары точек, каждая координата \PYZhy{}\PYZhy{} от 0 до 1.}
\PY{+w}{        }\PY{k+kd}{let}\PY{+w}{ }\PY{n}{pa}: \PY{p}{(}\PY{k+kt}{f64}\PY{p}{,}\PY{+w}{ }\PY{k+kt}{f64}\PY{p}{)}\PY{+w}{ }\PY{o}{=}\PY{+w}{ }\PY{p}{(}\PY{n}{rng}\PY{p}{.}\PY{n}{gen}\PY{p}{(}\PY{p}{)}\PY{p}{,}\PY{+w}{ }\PY{n}{rng}\PY{p}{.}\PY{n}{gen}\PY{p}{(}\PY{p}{)}\PY{p}{)}\PY{p}{;}
\PY{+w}{        }\PY{k+kd}{let}\PY{+w}{ }\PY{n}{pb}: \PY{p}{(}\PY{k+kt}{f64}\PY{p}{,}\PY{+w}{ }\PY{k+kt}{f64}\PY{p}{)}\PY{+w}{ }\PY{o}{=}\PY{+w}{ }\PY{p}{(}\PY{n}{rng}\PY{p}{.}\PY{n}{gen}\PY{p}{(}\PY{p}{)}\PY{p}{,}\PY{+w}{ }\PY{n}{rng}\PY{p}{.}\PY{n}{gen}\PY{p}{(}\PY{p}{)}\PY{p}{)}\PY{p}{;}
\PY{+w}{        }\PY{k+kd}{let}\PY{+w}{ }\PY{n}{pc}: \PY{p}{(}\PY{k+kt}{f64}\PY{p}{,}\PY{+w}{ }\PY{k+kt}{f64}\PY{p}{)}\PY{+w}{ }\PY{o}{=}\PY{+w}{ }\PY{p}{(}\PY{n}{rng}\PY{p}{.}\PY{n}{gen}\PY{p}{(}\PY{p}{)}\PY{p}{,}\PY{+w}{ }\PY{n}{rng}\PY{p}{.}\PY{n}{gen}\PY{p}{(}\PY{p}{)}\PY{p}{)}\PY{p}{;}
\PY{+w}{        }\PY{c+c1}{// Считаем векторы между точками}
\PY{+w}{        }\PY{k+kd}{let}\PY{+w}{ }\PY{n}{va}\PY{+w}{ }\PY{o}{=}\PY{+w}{ }\PY{p}{(}\PY{n}{pa}\PY{p}{.}\PY{l+m+mi}{0}\PY{+w}{ }\PY{o}{\PYZhy{}}\PY{+w}{ }\PY{n}{pb}\PY{p}{.}\PY{l+m+mi}{0}\PY{p}{,}\PY{+w}{ }\PY{n}{pa}\PY{p}{.}\PY{l+m+mi}{1}\PY{+w}{ }\PY{o}{\PYZhy{}}\PY{+w}{ }\PY{n}{pb}\PY{p}{.}\PY{l+m+mi}{1}\PY{p}{)}\PY{p}{;}
\PY{+w}{        }\PY{k+kd}{let}\PY{+w}{ }\PY{n}{vb}\PY{+w}{ }\PY{o}{=}\PY{+w}{ }\PY{p}{(}\PY{n}{pa}\PY{p}{.}\PY{l+m+mi}{0}\PY{+w}{ }\PY{o}{\PYZhy{}}\PY{+w}{ }\PY{n}{pc}\PY{p}{.}\PY{l+m+mi}{0}\PY{p}{,}\PY{+w}{ }\PY{n}{pa}\PY{p}{.}\PY{l+m+mi}{1}\PY{+w}{ }\PY{o}{\PYZhy{}}\PY{+w}{ }\PY{n}{pc}\PY{p}{.}\PY{l+m+mi}{1}\PY{p}{)}\PY{p}{;}
\PY{+w}{        }\PY{k+kd}{let}\PY{+w}{ }\PY{n}{vc}\PY{+w}{ }\PY{o}{=}\PY{+w}{ }\PY{p}{(}\PY{n}{pc}\PY{p}{.}\PY{l+m+mi}{0}\PY{+w}{ }\PY{o}{\PYZhy{}}\PY{+w}{ }\PY{n}{pb}\PY{p}{.}\PY{l+m+mi}{0}\PY{p}{,}\PY{+w}{ }\PY{n}{pc}\PY{p}{.}\PY{l+m+mi}{1}\PY{+w}{ }\PY{o}{\PYZhy{}}\PY{+w}{ }\PY{n}{pb}\PY{p}{.}\PY{l+m+mi}{1}\PY{p}{)}\PY{p}{;}
\PY{+w}{        }\PY{c+c1}{// Считаем длины этих трех векторов}
\PY{+w}{        }\PY{k+kd}{let}\PY{+w}{ }\PY{n}{magnitudes}: \PY{n+nb}{Vec}\PY{o}{\PYZlt{}}\PY{k+kt}{f64}\PY{o}{\PYZgt{}}\PY{+w}{ }\PY{o}{=}\PY{+w}{ }\PY{p}{[}\PY{n}{va}\PY{p}{,}\PY{+w}{ }\PY{n}{vb}\PY{p}{,}\PY{+w}{ }\PY{n}{vc}\PY{p}{]}\PY{p}{.}\PY{n}{iter}\PY{p}{(}\PY{p}{)}\PY{p}{.}\PY{n}{map}\PY{p}{(}\PY{o}{|}\PY{n}{x}\PY{o}{|}\PY{+w}{ }\PY{n}{magnitude}\PY{p}{(}\PY{n}{x}\PY{p}{)}\PY{p}{)}\PY{p}{.}\PY{n}{collect}\PY{p}{(}\PY{p}{)}\PY{p}{;}
\PY{+w}{        }\PY{k+kd}{let}\PY{+w}{ }\PY{n}{ma}\PY{+w}{ }\PY{o}{=}\PY{+w}{ }\PY{n}{magnitudes}\PY{p}{[}\PY{l+m+mi}{0}\PY{p}{]}\PY{p}{;}
\PY{+w}{        }\PY{k+kd}{let}\PY{+w}{ }\PY{n}{mb}\PY{+w}{ }\PY{o}{=}\PY{+w}{ }\PY{n}{magnitudes}\PY{p}{[}\PY{l+m+mi}{1}\PY{p}{]}\PY{p}{;}
\PY{+w}{        }\PY{k+kd}{let}\PY{+w}{ }\PY{n}{mc}\PY{+w}{ }\PY{o}{=}\PY{+w}{ }\PY{n}{magnitudes}\PY{p}{[}\PY{l+m+mi}{2}\PY{p}{]}\PY{p}{;}
\PY{+w}{        }
\PY{+w}{        }\PY{c+c1}{// Перебирая все перестановки этих трех векторов}
\PY{+w}{        }\PY{k}{for}\PY{+w}{ }\PY{n}{m}\PY{+w}{ }\PY{k}{in}\PY{+w}{ }\PY{p}{[}\PY{n}{ma}\PY{p}{,}\PY{+w}{ }\PY{n}{mb}\PY{p}{,}\PY{+w}{ }\PY{n}{mc}\PY{p}{]}\PY{p}{.}\PY{n}{iter}\PY{p}{(}\PY{p}{)}\PY{p}{.}\PY{n}{permutations}\PY{p}{(}\PY{l+m+mi}{3}\PY{p}{)}\PY{+w}{ }\PY{p}{\PYZob{}}
\PY{+w}{            }\PY{c+c1}{// проверяем, квадрат двух катетов меньше ли квадрата гипотенузы}
\PY{+w}{            }\PY{k}{if}\PY{+w}{ }\PY{p}{(}\PY{n}{m}\PY{p}{[}\PY{l+m+mi}{0}\PY{p}{]}\PY{+w}{ }\PY{o}{*}\PY{+w}{ }\PY{n}{m}\PY{p}{[}\PY{l+m+mi}{0}\PY{p}{]}\PY{p}{)}\PY{+w}{ }\PY{o}{+}\PY{+w}{ }\PY{p}{(}\PY{n}{m}\PY{p}{[}\PY{l+m+mi}{1}\PY{p}{]}\PY{+w}{ }\PY{o}{*}\PY{+w}{ }\PY{n}{m}\PY{p}{[}\PY{l+m+mi}{1}\PY{p}{]}\PY{p}{)}\PY{+w}{ }\PY{o}{\PYZlt{}}\PY{+w}{ }\PY{p}{(}\PY{n}{m}\PY{p}{[}\PY{l+m+mi}{2}\PY{p}{]}\PY{+w}{ }\PY{o}{*}\PY{+w}{ }\PY{n}{m}\PY{p}{[}\PY{l+m+mi}{2}\PY{p}{]}\PY{p}{)}\PY{+w}{ }\PY{p}{\PYZob{}}
\PY{+w}{                }\PY{c+c1}{// если да, то такой треугольник тупоугольный.}
\PY{+w}{                }\PY{k}{return}\PY{+w}{ }\PY{k+kc}{true}\PY{p}{;}
\PY{+w}{            }\PY{p}{\PYZcb{}}
\PY{+w}{        }\PY{p}{\PYZcb{}}
\PY{+w}{        }\PY{c+c1}{// Если же такой перестановки сторон не существует,}
\PY{+w}{        }\PY{c+c1}{// то треугольник не тупоугольный.}
\PY{+w}{        }\PY{k+kc}{false}
\PY{+w}{    }\PY{p}{\PYZcb{}}
\PY{+w}{    }
\PY{+w}{    }\PY{k}{fn} \PY{n+nf}{desired\PYZus{}outcomes}\PY{p}{(}\PY{o}{\PYZam{}}\PY{n+nb+bp}{self}\PY{p}{)}\PY{+w}{ }\PYZhy{}\PYZgt{} \PY{n+nb}{Vec}\PY{o}{\PYZlt{}}\PY{n+nb+bp}{Self}::\PY{n}{Outcome}\PY{o}{\PYZgt{}}\PY{+w}{ }\PY{p}{\PYZob{}}\PY{n+nf+fm}{vec!}\PY{p}{[}\PY{k+kc}{true}\PY{p}{]}\PY{p}{\PYZcb{}}
\PY{p}{\PYZcb{}}

\PY{k+kd}{let}\PY{+w}{ }\PY{n}{a}\PY{+w}{ }\PY{o}{=}\PY{+w}{ }\PY{n}{Task1}\PY{p}{\PYZob{}}\PY{p}{\PYZcb{}}\PY{p}{;}
\PY{k+kd}{let}\PY{+w}{ }\PY{n}{prob}\PY{+w}{ }\PY{o}{=}\PY{+w}{ }\PY{n}{a}\PY{p}{.}\PY{n}{probability\PYZus{}of\PYZus{}desired}\PY{p}{(}\PY{l+m+mi}{1\PYZus{}000\PYZus{}000}\PY{p}{)}\PY{p}{;}
\PY{n+nf+fm}{println!}\PY{p}{(}\PY{l+s}{\PYZdq{}}\PY{l+s}{\PYZob{}\PYZcb{} \PYZhy{}\PYZhy{} \PYZob{}\PYZcb{}}\PY{l+s}{\PYZdq{}}\PY{p}{,}\PY{+w}{ }\PY{n}{a}\PY{p}{.}\PY{n}{description}\PY{p}{(}\PY{p}{)}\PY{p}{,}\PY{+w}{ }\PY{n}{prob}\PY{p}{)}\PY{p}{;}
\PY{n}{answers}\PY{p}{.}\PY{n}{task\PYZus{}1}\PY{+w}{ }\PY{o}{=}\PY{+w}{ }\PY{n+nb}{Some}\PY{p}{(}\PY{n}{prob}\PY{p}{)}\PY{p}{;}
\end{Verbatim}
\end{tcolorbox}

    \begin{Verbatim}[commandchars=\\\{\}]
Чему равна вероятность того, что случайный треугольник, нарисованный внутри
квадрата со стороной 1, является тупоугольным? -- 0.725906
    \end{Verbatim}

            \begin{tcolorbox}[breakable, size=fbox, boxrule=.5pt, pad at break*=1mm, opacityfill=0]
\prompt{Out}{outcolor}{5}{\boxspacing}
\begin{Verbatim}[commandchars=\\\{\}]
Timing: false
sccache: true

\end{Verbatim}
\end{tcolorbox}
        
    \begin{tcolorbox}[breakable, size=fbox, boxrule=1pt, pad at break*=1mm,colback=cellbackground, colframe=cellborder]
\prompt{In}{incolor}{6}{\boxspacing}
\begin{Verbatim}[commandchars=\\\{\}]
:\PY{n+nc}{timing}\PY{+w}{ }\PY{l+m+mi}{1}
:\PY{n+nc}{sccache}\PY{+w}{ }\PY{l+m+mi}{1}
:\PY{n+nc}{dep}\PY{+w}{ }\PY{n}{rand}\PY{+w}{ }\PY{o}{=}\PY{+w}{ }\PY{l+s}{\PYZdq{}}\PY{l+s}{0.8.5}\PY{l+s}{\PYZdq{}}

\PY{k}{struct} \PY{n+nc}{Task2}\PY{p}{;}

\PY{k}{impl}\PY{+w}{ }\PY{n}{Experiment}\PY{+w}{ }\PY{k}{for}\PY{+w}{ }\PY{n}{Task2}\PY{+w}{ }\PY{p}{\PYZob{}}
\PY{+w}{    }\PY{k}{type} \PY{n+nc}{Outcome}\PY{+w}{ }\PY{o}{=}\PY{+w}{ }\PY{k+kt}{bool}\PY{p}{;}
\PY{+w}{    }\PY{k}{fn} \PY{n+nf}{description}\PY{p}{(}\PY{o}{\PYZam{}}\PY{n+nb+bp}{self}\PY{p}{)}\PY{+w}{ }\PYZhy{}\PYZgt{} \PY{n+nb}{String} \PY{p}{\PYZob{}}
\PY{+w}{        }\PY{l+s}{\PYZdq{}}\PY{l+s}{Чему равна вероятность того, что случайный треугольник, нарисованный внутри}
\PY{l+s}{прямоугольника, у которого одна сторона в 2 раза длиннее другой, является тупоугольным?}\PY{l+s}{\PYZdq{}}\PY{p}{.}\PY{n}{to\PYZus{}string}\PY{p}{(}\PY{p}{)}
\PY{+w}{    }\PY{p}{\PYZcb{}}
\PY{+w}{    }\PY{k}{fn} \PY{n+nf}{try\PYZus{}it}\PY{o}{\PYZlt{}}\PY{n}{T}: \PY{n+nc}{rand}::\PY{n}{Rng}\PY{o}{\PYZgt{}}\PY{p}{(}\PY{o}{\PYZam{}}\PY{n+nb+bp}{self}\PY{p}{,}\PY{+w}{ }\PY{n}{rng}: \PY{k+kp}{\PYZam{}}\PY{n+nc}{mut}\PY{+w}{ }\PY{n}{T}\PY{p}{)}\PY{+w}{ }\PYZhy{}\PYZgt{} \PY{n+nc}{Self}::\PY{n}{Outcome}\PY{+w}{ }\PY{p}{\PYZob{}}
\PY{+w}{        }\PY{c+c1}{// Эта задача очень похожа на предыдущую, поэтому большинство кода идентично.}
\PY{+w}{        }
\PY{+w}{        }\PY{c+c1}{// Генерируем три пары точек, где первая координата \PYZhy{}\PYZhy{} от 0 до 1, а вторая \PYZhy{}\PYZhy{} от 0 до 2.}
\PY{+w}{        }\PY{k+kd}{let}\PY{+w}{ }\PY{n}{pa}: \PY{p}{(}\PY{k+kt}{f64}\PY{p}{,}\PY{+w}{ }\PY{k+kt}{f64}\PY{p}{)}\PY{+w}{ }\PY{o}{=}\PY{+w}{ }\PY{p}{(}\PY{n}{rng}\PY{p}{.}\PY{n}{gen}\PY{p}{(}\PY{p}{)}\PY{p}{,}\PY{+w}{ }\PY{n}{rng}\PY{p}{.}\PY{n}{gen}::\PY{o}{\PYZlt{}}\PY{k+kt}{f64}\PY{o}{\PYZgt{}}\PY{p}{(}\PY{p}{)}\PY{o}{*}\PY{l+m+mf}{2.0}\PY{p}{)}\PY{p}{;}
\PY{+w}{        }\PY{k+kd}{let}\PY{+w}{ }\PY{n}{pb}: \PY{p}{(}\PY{k+kt}{f64}\PY{p}{,}\PY{+w}{ }\PY{k+kt}{f64}\PY{p}{)}\PY{+w}{ }\PY{o}{=}\PY{+w}{ }\PY{p}{(}\PY{n}{rng}\PY{p}{.}\PY{n}{gen}\PY{p}{(}\PY{p}{)}\PY{p}{,}\PY{+w}{ }\PY{n}{rng}\PY{p}{.}\PY{n}{gen}::\PY{o}{\PYZlt{}}\PY{k+kt}{f64}\PY{o}{\PYZgt{}}\PY{p}{(}\PY{p}{)}\PY{o}{*}\PY{l+m+mf}{2.0}\PY{p}{)}\PY{p}{;}
\PY{+w}{        }\PY{k+kd}{let}\PY{+w}{ }\PY{n}{pc}: \PY{p}{(}\PY{k+kt}{f64}\PY{p}{,}\PY{+w}{ }\PY{k+kt}{f64}\PY{p}{)}\PY{+w}{ }\PY{o}{=}\PY{+w}{ }\PY{p}{(}\PY{n}{rng}\PY{p}{.}\PY{n}{gen}\PY{p}{(}\PY{p}{)}\PY{p}{,}\PY{+w}{ }\PY{n}{rng}\PY{p}{.}\PY{n}{gen}::\PY{o}{\PYZlt{}}\PY{k+kt}{f64}\PY{o}{\PYZgt{}}\PY{p}{(}\PY{p}{)}\PY{o}{*}\PY{l+m+mf}{2.0}\PY{p}{)}\PY{p}{;}
\PY{+w}{        }\PY{c+c1}{// Считаем векторы между точками}
\PY{+w}{        }\PY{k+kd}{let}\PY{+w}{ }\PY{n}{va}\PY{+w}{ }\PY{o}{=}\PY{+w}{ }\PY{p}{(}\PY{n}{pa}\PY{p}{.}\PY{l+m+mi}{0}\PY{+w}{ }\PY{o}{\PYZhy{}}\PY{+w}{ }\PY{n}{pb}\PY{p}{.}\PY{l+m+mi}{0}\PY{p}{,}\PY{+w}{ }\PY{n}{pa}\PY{p}{.}\PY{l+m+mi}{1}\PY{+w}{ }\PY{o}{\PYZhy{}}\PY{+w}{ }\PY{n}{pb}\PY{p}{.}\PY{l+m+mi}{1}\PY{p}{)}\PY{p}{;}
\PY{+w}{        }\PY{k+kd}{let}\PY{+w}{ }\PY{n}{vb}\PY{+w}{ }\PY{o}{=}\PY{+w}{ }\PY{p}{(}\PY{n}{pa}\PY{p}{.}\PY{l+m+mi}{0}\PY{+w}{ }\PY{o}{\PYZhy{}}\PY{+w}{ }\PY{n}{pc}\PY{p}{.}\PY{l+m+mi}{0}\PY{p}{,}\PY{+w}{ }\PY{n}{pa}\PY{p}{.}\PY{l+m+mi}{1}\PY{+w}{ }\PY{o}{\PYZhy{}}\PY{+w}{ }\PY{n}{pc}\PY{p}{.}\PY{l+m+mi}{1}\PY{p}{)}\PY{p}{;}
\PY{+w}{        }\PY{k+kd}{let}\PY{+w}{ }\PY{n}{vc}\PY{+w}{ }\PY{o}{=}\PY{+w}{ }\PY{p}{(}\PY{n}{pc}\PY{p}{.}\PY{l+m+mi}{0}\PY{+w}{ }\PY{o}{\PYZhy{}}\PY{+w}{ }\PY{n}{pb}\PY{p}{.}\PY{l+m+mi}{0}\PY{p}{,}\PY{+w}{ }\PY{n}{pc}\PY{p}{.}\PY{l+m+mi}{1}\PY{+w}{ }\PY{o}{\PYZhy{}}\PY{+w}{ }\PY{n}{pb}\PY{p}{.}\PY{l+m+mi}{1}\PY{p}{)}\PY{p}{;}
\PY{+w}{        }\PY{c+c1}{// Считаем длины этих трех векторов}
\PY{+w}{        }\PY{k+kd}{let}\PY{+w}{ }\PY{n}{magnitudes}: \PY{n+nb}{Vec}\PY{o}{\PYZlt{}}\PY{k+kt}{f64}\PY{o}{\PYZgt{}}\PY{+w}{ }\PY{o}{=}\PY{+w}{ }\PY{p}{[}\PY{n}{va}\PY{p}{,}\PY{+w}{ }\PY{n}{vb}\PY{p}{,}\PY{+w}{ }\PY{n}{vc}\PY{p}{]}\PY{p}{.}\PY{n}{iter}\PY{p}{(}\PY{p}{)}\PY{p}{.}\PY{n}{map}\PY{p}{(}\PY{o}{|}\PY{n}{x}\PY{o}{|}\PY{+w}{ }\PY{n}{magnitude}\PY{p}{(}\PY{n}{x}\PY{p}{)}\PY{p}{)}\PY{p}{.}\PY{n}{collect}\PY{p}{(}\PY{p}{)}\PY{p}{;}
\PY{+w}{        }\PY{k+kd}{let}\PY{+w}{ }\PY{n}{ma}\PY{+w}{ }\PY{o}{=}\PY{+w}{ }\PY{n}{magnitudes}\PY{p}{[}\PY{l+m+mi}{0}\PY{p}{]}\PY{p}{;}
\PY{+w}{        }\PY{k+kd}{let}\PY{+w}{ }\PY{n}{mb}\PY{+w}{ }\PY{o}{=}\PY{+w}{ }\PY{n}{magnitudes}\PY{p}{[}\PY{l+m+mi}{1}\PY{p}{]}\PY{p}{;}
\PY{+w}{        }\PY{k+kd}{let}\PY{+w}{ }\PY{n}{mc}\PY{+w}{ }\PY{o}{=}\PY{+w}{ }\PY{n}{magnitudes}\PY{p}{[}\PY{l+m+mi}{2}\PY{p}{]}\PY{p}{;}
\PY{+w}{        }
\PY{+w}{        }\PY{c+c1}{// Перебирая все перестановки этих трех векторов}
\PY{+w}{        }\PY{k}{for}\PY{+w}{ }\PY{n}{m}\PY{+w}{ }\PY{k}{in}\PY{+w}{ }\PY{p}{[}\PY{n}{ma}\PY{p}{,}\PY{+w}{ }\PY{n}{mb}\PY{p}{,}\PY{+w}{ }\PY{n}{mc}\PY{p}{]}\PY{p}{.}\PY{n}{iter}\PY{p}{(}\PY{p}{)}\PY{p}{.}\PY{n}{permutations}\PY{p}{(}\PY{l+m+mi}{3}\PY{p}{)}\PY{+w}{ }\PY{p}{\PYZob{}}
\PY{+w}{            }\PY{c+c1}{// проверяем, квадрат двух катетов меньше ли квадрата гипотенузы}
\PY{+w}{            }\PY{k}{if}\PY{+w}{ }\PY{p}{(}\PY{n}{m}\PY{p}{[}\PY{l+m+mi}{0}\PY{p}{]}\PY{+w}{ }\PY{o}{*}\PY{+w}{ }\PY{n}{m}\PY{p}{[}\PY{l+m+mi}{0}\PY{p}{]}\PY{p}{)}\PY{+w}{ }\PY{o}{+}\PY{+w}{ }\PY{p}{(}\PY{n}{m}\PY{p}{[}\PY{l+m+mi}{1}\PY{p}{]}\PY{+w}{ }\PY{o}{*}\PY{+w}{ }\PY{n}{m}\PY{p}{[}\PY{l+m+mi}{1}\PY{p}{]}\PY{p}{)}\PY{+w}{ }\PY{o}{\PYZlt{}}\PY{+w}{ }\PY{p}{(}\PY{n}{m}\PY{p}{[}\PY{l+m+mi}{2}\PY{p}{]}\PY{+w}{ }\PY{o}{*}\PY{+w}{ }\PY{n}{m}\PY{p}{[}\PY{l+m+mi}{2}\PY{p}{]}\PY{p}{)}\PY{+w}{ }\PY{p}{\PYZob{}}
\PY{+w}{                }\PY{c+c1}{// если да, то такой треугольник тупоугольный.}
\PY{+w}{                }\PY{k}{return}\PY{+w}{ }\PY{k+kc}{true}\PY{p}{;}
\PY{+w}{            }\PY{p}{\PYZcb{}}
\PY{+w}{        }\PY{p}{\PYZcb{}}
\PY{+w}{        }\PY{c+c1}{// Если же такой перестановки сторон не существует,}
\PY{+w}{        }\PY{c+c1}{// то треугольник не тупоугольный.}
\PY{+w}{        }\PY{k+kc}{false}
\PY{+w}{    }\PY{p}{\PYZcb{}}
\PY{+w}{    }
\PY{+w}{    }\PY{k}{fn} \PY{n+nf}{desired\PYZus{}outcomes}\PY{p}{(}\PY{o}{\PYZam{}}\PY{n+nb+bp}{self}\PY{p}{)}\PY{+w}{ }\PYZhy{}\PYZgt{} \PY{n+nb}{Vec}\PY{o}{\PYZlt{}}\PY{n+nb+bp}{Self}::\PY{n}{Outcome}\PY{o}{\PYZgt{}}\PY{+w}{ }\PY{p}{\PYZob{}}\PY{n+nf+fm}{vec!}\PY{p}{[}\PY{k+kc}{true}\PY{p}{]}\PY{p}{\PYZcb{}}
\PY{p}{\PYZcb{}}

\PY{k+kd}{let}\PY{+w}{ }\PY{n}{a}\PY{+w}{ }\PY{o}{=}\PY{+w}{ }\PY{n}{Task2}\PY{p}{\PYZob{}}\PY{p}{\PYZcb{}}\PY{p}{;}
\PY{k+kd}{let}\PY{+w}{ }\PY{n}{prob}\PY{+w}{ }\PY{o}{=}\PY{+w}{ }\PY{n}{a}\PY{p}{.}\PY{n}{probability\PYZus{}of\PYZus{}desired}\PY{p}{(}\PY{l+m+mi}{1\PYZus{}000\PYZus{}000}\PY{p}{)}\PY{p}{;}
\PY{n+nf+fm}{println!}\PY{p}{(}\PY{l+s}{\PYZdq{}}\PY{l+s}{\PYZob{}\PYZcb{} \PYZhy{}\PYZhy{} \PYZob{}\PYZcb{}}\PY{l+s}{\PYZdq{}}\PY{p}{,}\PY{+w}{ }\PY{n}{a}\PY{p}{.}\PY{n}{description}\PY{p}{(}\PY{p}{)}\PY{p}{,}\PY{+w}{ }\PY{n}{prob}\PY{p}{)}\PY{p}{;}
\PY{n}{answers}\PY{p}{.}\PY{n}{task\PYZus{}2}\PY{+w}{ }\PY{o}{=}\PY{+w}{ }\PY{n+nb}{Some}\PY{p}{(}\PY{n}{prob}\PY{p}{)}\PY{p}{;}
\end{Verbatim}
\end{tcolorbox}

    \begin{Verbatim}[commandchars=\\\{\}]
Чему равна вероятность того, что случайный треугольник, нарисованный внутри
прямоугольника, у которого одна сторона в 2 раза длиннее другой, является
тупоугольным? -- 0.7984
    \end{Verbatim}

            \begin{tcolorbox}[breakable, size=fbox, boxrule=.5pt, pad at break*=1mm, opacityfill=0]
\prompt{Out}{outcolor}{6}{\boxspacing}
\begin{Verbatim}[commandchars=\\\{\}]
Timing: true
sccache: true

\end{Verbatim}
\end{tcolorbox}
        
    \begin{tcolorbox}[breakable, size=fbox, boxrule=1pt, pad at break*=1mm,colback=cellbackground, colframe=cellborder]
\prompt{In}{incolor}{7}{\boxspacing}
\begin{Verbatim}[commandchars=\\\{\}]
:\PY{n+nc}{timing}\PY{+w}{ }\PY{l+m+mi}{1}
:\PY{n+nc}{sccache}\PY{+w}{ }\PY{l+m+mi}{1}
:\PY{n+nc}{dep}\PY{+w}{ }\PY{n}{rand}\PY{+w}{ }\PY{o}{=}\PY{+w}{ }\PY{l+s}{\PYZdq{}}\PY{l+s}{0.8.5}\PY{l+s}{\PYZdq{}}

\PY{k}{struct} \PY{n+nc}{Task3}\PY{p}{;}

\PY{k}{impl}\PY{+w}{ }\PY{n}{Experiment}\PY{+w}{ }\PY{k}{for}\PY{+w}{ }\PY{n}{Task3}\PY{+w}{ }\PY{p}{\PYZob{}}
\PY{+w}{    }\PY{k}{type} \PY{n+nc}{Outcome}\PY{+w}{ }\PY{o}{=}\PY{+w}{ }\PY{k+kt}{bool}\PY{p}{;}
\PY{+w}{    }\PY{k}{fn} \PY{n+nf}{description}\PY{p}{(}\PY{o}{\PYZam{}}\PY{n+nb+bp}{self}\PY{p}{)}\PY{+w}{ }\PYZhy{}\PYZgt{} \PY{n+nb}{String} \PY{p}{\PYZob{}}
\PY{+w}{        }\PY{l+s}{\PYZdq{}}\PY{l+s}{Пусть a, b и c – независимые случайные величины, распределенные равномерно на [0,1].}
\PY{l+s}{Рассмотрим квадратное уравнение a*x\PYZca{}2+b*x+c=0. Чему равна вероятность того, что его решения –}
\PY{l+s}{действительные числа?}\PY{l+s}{\PYZdq{}}\PY{p}{.}\PY{n}{to\PYZus{}string}\PY{p}{(}\PY{p}{)}
\PY{+w}{    }\PY{p}{\PYZcb{}}
\PY{+w}{    }\PY{k}{fn} \PY{n+nf}{try\PYZus{}it}\PY{o}{\PYZlt{}}\PY{n}{T}: \PY{n+nc}{rand}::\PY{n}{Rng}\PY{o}{\PYZgt{}}\PY{p}{(}\PY{o}{\PYZam{}}\PY{n+nb+bp}{self}\PY{p}{,}\PY{+w}{ }\PY{n}{rng}: \PY{k+kp}{\PYZam{}}\PY{n+nc}{mut}\PY{+w}{ }\PY{n}{T}\PY{p}{)}\PY{+w}{ }\PYZhy{}\PYZgt{} \PY{n+nc}{Self}::\PY{n}{Outcome}\PY{+w}{ }\PY{p}{\PYZob{}}
\PY{+w}{        }\PY{c+c1}{// Генерируем три числа от 0 до 1}
\PY{+w}{        }\PY{k+kd}{let}\PY{+w}{ }\PY{n}{a}: \PY{k+kt}{f64} \PY{o}{=}\PY{+w}{ }\PY{n}{rng}\PY{p}{.}\PY{n}{gen}\PY{p}{(}\PY{p}{)}\PY{p}{;}
\PY{+w}{        }\PY{k+kd}{let}\PY{+w}{ }\PY{n}{b}: \PY{k+kt}{f64} \PY{o}{=}\PY{+w}{ }\PY{n}{rng}\PY{p}{.}\PY{n}{gen}\PY{p}{(}\PY{p}{)}\PY{p}{;}
\PY{+w}{        }\PY{k+kd}{let}\PY{+w}{ }\PY{n}{c}: \PY{k+kt}{f64} \PY{o}{=}\PY{+w}{ }\PY{n}{rng}\PY{p}{.}\PY{n}{gen}\PY{p}{(}\PY{p}{)}\PY{p}{;}
\PY{+w}{        }\PY{c+c1}{// Смотрим на значение дискриминанта квадратного уравнения \PYZhy{}\PYZhy{} (b\PYZca{}2) \PYZhy{} (4ac).}
\PY{+w}{        }\PY{k+kd}{let}\PY{+w}{ }\PY{n}{discriminant\PYZus{}sq}\PY{+w}{ }\PY{o}{=}\PY{+w}{ }\PY{p}{(}\PY{n}{b}\PY{+w}{ }\PY{o}{*}\PY{+w}{ }\PY{n}{b}\PY{p}{)}\PY{+w}{ }\PY{o}{\PYZhy{}}\PY{+w}{ }\PY{p}{(}\PY{l+m+mf}{4.0}\PY{+w}{ }\PY{o}{*}\PY{+w}{ }\PY{n}{a}\PY{+w}{ }\PY{o}{*}\PY{+w}{ }\PY{n}{c}\PY{p}{)}\PY{p}{;}
\PY{+w}{        }\PY{c+c1}{// Если оно больше или равно нулю, то у уравнения есть действительные решения.}
\PY{+w}{        }\PY{n}{discriminant\PYZus{}sq}\PY{+w}{ }\PY{o}{\PYZgt{}}\PY{o}{=}\PY{+w}{ }\PY{l+m+mf}{0.0}
\PY{+w}{    }\PY{p}{\PYZcb{}}
\PY{+w}{    }
\PY{+w}{    }\PY{k}{fn} \PY{n+nf}{desired\PYZus{}outcomes}\PY{p}{(}\PY{o}{\PYZam{}}\PY{n+nb+bp}{self}\PY{p}{)}\PY{+w}{ }\PYZhy{}\PYZgt{} \PY{n+nb}{Vec}\PY{o}{\PYZlt{}}\PY{n+nb+bp}{Self}::\PY{n}{Outcome}\PY{o}{\PYZgt{}}\PY{+w}{ }\PY{p}{\PYZob{}}\PY{n+nf+fm}{vec!}\PY{p}{[}\PY{k+kc}{true}\PY{p}{]}\PY{p}{\PYZcb{}}
\PY{p}{\PYZcb{}}

\PY{k+kd}{let}\PY{+w}{ }\PY{n}{a}\PY{+w}{ }\PY{o}{=}\PY{+w}{ }\PY{n}{Task3}\PY{p}{\PYZob{}}\PY{p}{\PYZcb{}}\PY{p}{;}
\PY{k+kd}{let}\PY{+w}{ }\PY{n}{prob}\PY{+w}{ }\PY{o}{=}\PY{+w}{ }\PY{n}{a}\PY{p}{.}\PY{n}{probability\PYZus{}of\PYZus{}desired}\PY{p}{(}\PY{l+m+mi}{1\PYZus{}000\PYZus{}000}\PY{p}{)}\PY{p}{;}
\PY{n+nf+fm}{println!}\PY{p}{(}\PY{l+s}{\PYZdq{}}\PY{l+s}{\PYZob{}\PYZcb{} \PYZhy{}\PYZhy{} \PYZob{}\PYZcb{}}\PY{l+s}{\PYZdq{}}\PY{p}{,}\PY{+w}{ }\PY{n}{a}\PY{p}{.}\PY{n}{description}\PY{p}{(}\PY{p}{)}\PY{p}{,}\PY{+w}{ }\PY{n}{prob}\PY{p}{)}\PY{p}{;}
\PY{n}{answers}\PY{p}{.}\PY{n}{task\PYZus{}3}\PY{+w}{ }\PY{o}{=}\PY{+w}{ }\PY{n+nb}{Some}\PY{p}{(}\PY{n}{prob}\PY{p}{)}\PY{p}{;}
\end{Verbatim}
\end{tcolorbox}

    \begin{Verbatim}[commandchars=\\\{\}]
Пусть a, b и c – независимые случайные величины, распределенные равномерно на
[0,1].
Рассмотрим квадратное уравнение a*x\^{}2+b*x+c=0. Чему равна вероятность того, что
его решения –
действительные числа? -- 0.254331
    \end{Verbatim}

            \begin{tcolorbox}[breakable, size=fbox, boxrule=.5pt, pad at break*=1mm, opacityfill=0]
\prompt{Out}{outcolor}{7}{\boxspacing}
\begin{Verbatim}[commandchars=\\\{\}]
Timing: false
sccache: true

\end{Verbatim}
\end{tcolorbox}
        
    \begin{tcolorbox}[breakable, size=fbox, boxrule=1pt, pad at break*=1mm,colback=cellbackground, colframe=cellborder]
\prompt{In}{incolor}{8}{\boxspacing}
\begin{Verbatim}[commandchars=\\\{\}]
:\PY{n+nc}{timing}\PY{+w}{ }\PY{l+m+mi}{1}
:\PY{n+nc}{sccache}\PY{+w}{ }\PY{l+m+mi}{1}
:\PY{n+nc}{dep}\PY{+w}{ }\PY{n}{rand}\PY{+w}{ }\PY{o}{=}\PY{+w}{ }\PY{l+s}{\PYZdq{}}\PY{l+s}{0.8.5}\PY{l+s}{\PYZdq{}}

\PY{k}{use}\PY{+w}{ }\PY{n}{rand}::\PY{n}{prelude}::\PY{p}{\PYZob{}}\PY{n}{SliceRandom}\PY{p}{,}\PY{+w}{ }\PY{n}{IteratorRandom}\PY{p}{\PYZcb{}}\PY{p}{;}
\PY{k}{struct} \PY{n+nc}{Task4}\PY{p}{(}\PY{k+kt}{usize}\PY{p}{)}\PY{p}{;}

\PY{k}{impl}\PY{+w}{ }\PY{n}{Experiment}\PY{+w}{ }\PY{k}{for}\PY{+w}{ }\PY{n}{Task4}\PY{+w}{ }\PY{p}{\PYZob{}}
\PY{+w}{    }\PY{k}{type} \PY{n+nc}{Outcome}\PY{+w}{ }\PY{o}{=}\PY{+w}{ }\PY{k+kt}{bool}\PY{p}{;}
\PY{+w}{    }\PY{k}{fn} \PY{n+nf}{description}\PY{p}{(}\PY{o}{\PYZam{}}\PY{n+nb+bp}{self}\PY{p}{)}\PY{+w}{ }\PYZhy{}\PYZgt{} \PY{n+nb}{String} \PY{p}{\PYZob{}}
\PY{+w}{        }\PY{n+nf+fm}{format!}\PY{p}{(}\PY{l+s}{\PYZdq{}}\PY{l+s}{В самолете \PYZob{}\PYZcb{} мест, и все билеты проданы пассажирам. Первым в самолет заходит}
\PY{l+s}{рассеянный учёный и, не посмотрев на билет, занимает первое попавшееся место. Далее пассажиры}
\PY{l+s}{входят по одному. Если вошедший видит, что его место свободно, он занимает свое место. Если же}
\PY{l+s}{место занято, то вошедший занимает первое попавшееся свободное место. Найдите вероятность того,}
\PY{l+s}{что пассажир, вошедший последним, займет место согласно своему билету.}\PY{l+s}{\PYZdq{}}\PY{p}{,}\PY{+w}{ }\PY{n+nb+bp}{self}\PY{p}{.}\PY{l+m+mi}{0}\PY{p}{)}
\PY{+w}{    }\PY{p}{\PYZcb{}}
\PY{+w}{    }\PY{k}{fn} \PY{n+nf}{try\PYZus{}it}\PY{o}{\PYZlt{}}\PY{n}{T}: \PY{n+nc}{rand}::\PY{n}{Rng}\PY{o}{\PYZgt{}}\PY{p}{(}\PY{o}{\PYZam{}}\PY{n+nb+bp}{self}\PY{p}{,}\PY{+w}{ }\PY{n}{rng}: \PY{k+kp}{\PYZam{}}\PY{n+nc}{mut}\PY{+w}{ }\PY{n}{T}\PY{p}{)}\PY{+w}{ }\PYZhy{}\PYZgt{} \PY{n+nc}{Self}::\PY{n}{Outcome}\PY{+w}{ }\PY{p}{\PYZob{}}
\PY{+w}{        }\PY{c+c1}{// Решение существует, только если мест не меньше 2.}
\PY{+w}{        }\PY{n+nf+fm}{assert!}\PY{p}{(}\PY{n+nb+bp}{self}\PY{p}{.}\PY{l+m+mi}{0}\PY{+w}{ }\PY{o}{\PYZgt{}}\PY{o}{=}\PY{+w}{ }\PY{l+m+mi}{2}\PY{p}{)}\PY{p}{;}
\PY{+w}{        }\PY{c+c1}{// Сначала создаем список мест, список пустых мест и список пассажиров.}
\PY{+w}{        }\PY{c+c1}{// Пассажиры идут в случайном порядке.}
\PY{+w}{        }\PY{k+kd}{let}\PY{+w}{ }\PY{k}{mut}\PY{+w}{ }\PY{n}{seats}: \PY{n+nb}{Vec}\PY{o}{\PYZlt{}}\PY{n+nb}{Option}\PY{o}{\PYZlt{}}\PY{k+kt}{usize}\PY{o}{\PYZgt{}}\PY{o}{\PYZgt{}}\PY{+w}{ }\PY{o}{=}\PY{+w}{ }\PY{n+nf+fm}{vec!}\PY{p}{[}\PY{n+nb}{None}\PY{p}{;}\PY{+w}{ }\PY{n+nb+bp}{self}\PY{p}{.}\PY{l+m+mi}{0}\PY{p}{]}\PY{p}{;}
\PY{+w}{        }\PY{k+kd}{let}\PY{+w}{ }\PY{k}{mut}\PY{+w}{ }\PY{n}{empty\PYZus{}seats}: \PY{n+nb}{Vec}\PY{o}{\PYZlt{}}\PY{k+kt}{usize}\PY{o}{\PYZgt{}}\PY{+w}{ }\PY{o}{=}\PY{+w}{ }\PY{p}{(}\PY{l+m+mi}{0}\PY{o}{..}\PY{n+nb+bp}{self}\PY{p}{.}\PY{l+m+mi}{0}\PY{p}{)}\PY{p}{.}\PY{n}{collect}\PY{p}{(}\PY{p}{)}\PY{p}{;}
\PY{+w}{        }\PY{k+kd}{let}\PY{+w}{ }\PY{k}{mut}\PY{+w}{ }\PY{n}{passengers}: \PY{n+nb}{Vec}\PY{o}{\PYZlt{}}\PY{k+kt}{usize}\PY{o}{\PYZgt{}}\PY{+w}{ }\PY{o}{=}\PY{+w}{ }\PY{p}{(}\PY{l+m+mi}{0}\PY{o}{..}\PY{n+nb+bp}{self}\PY{p}{.}\PY{l+m+mi}{0}\PY{p}{)}\PY{p}{.}\PY{n}{collect}\PY{p}{(}\PY{p}{)}\PY{p}{;}
\PY{+w}{        }\PY{n}{passengers}\PY{p}{.}\PY{n}{shuffle}\PY{p}{(}\PY{n}{rng}\PY{p}{)}\PY{p}{;}
\PY{+w}{        }
\PY{+w}{        }\PY{c+c1}{// Сначала первый пассажир занимает свое место случайно.}
\PY{+w}{        }\PY{k+kd}{let}\PY{+w}{ }\PY{n}{first\PYZus{}passenger}\PY{+w}{ }\PY{o}{=}\PY{+w}{ }\PY{n}{passengers}\PY{p}{.}\PY{n}{pop}\PY{p}{(}\PY{p}{)}\PY{p}{;}
\PY{+w}{        }\PY{k+kd}{let}\PY{+w}{ }\PY{n}{first\PYZus{}seat\PYZus{}idx}\PY{+w}{ }\PY{o}{=}\PY{+w}{ }\PY{o}{*}\PY{n}{empty\PYZus{}seats}\PY{p}{.}\PY{n}{iter}\PY{p}{(}\PY{p}{)}\PY{p}{.}\PY{n}{choose}\PY{p}{(}\PY{n}{rng}\PY{p}{)}\PY{p}{.}\PY{n}{unwrap}\PY{p}{(}\PY{p}{)}\PY{p}{;}
\PY{+w}{        }\PY{n}{seats}\PY{p}{[}\PY{n}{first\PYZus{}seat\PYZus{}idx}\PY{p}{]}\PY{+w}{ }\PY{o}{=}\PY{+w}{ }\PY{n}{first\PYZus{}passenger}\PY{p}{;}
\PY{+w}{        }\PY{n}{empty\PYZus{}seats}\PY{p}{.}\PY{n}{retain}\PY{p}{(}\PY{o}{|}\PY{n}{i}\PY{o}{|}\PY{+w}{ }\PY{o}{*}\PY{n}{i}\PY{+w}{ }\PY{o}{!}\PY{o}{=}\PY{+w}{ }\PY{n}{first\PYZus{}seat\PYZus{}idx}\PY{p}{)}\PY{p}{;}
\PY{+w}{        }
\PY{+w}{        }\PY{c+c1}{// Затем, все пассажиры до последнего занимают место.}
\PY{+w}{        }\PY{k}{while}\PY{+w}{ }\PY{n}{empty\PYZus{}seats}\PY{p}{.}\PY{n}{len}\PY{p}{(}\PY{p}{)}\PY{+w}{ }\PY{o}{!}\PY{o}{=}\PY{+w}{ }\PY{l+m+mi}{1}\PY{+w}{ }\PY{p}{\PYZob{}}
\PY{+w}{            }\PY{k+kd}{let}\PY{+w}{ }\PY{n}{next\PYZus{}passenger}\PY{+w}{ }\PY{o}{=}\PY{+w}{ }\PY{n}{passengers}\PY{p}{.}\PY{n}{pop}\PY{p}{(}\PY{p}{)}\PY{p}{.}\PY{n}{unwrap}\PY{p}{(}\PY{p}{)}\PY{p}{;}
\PY{+w}{            }\PY{c+c1}{// Если место этого пассажира не занято, то пассажир занимает свое место.}
\PY{+w}{            }\PY{k}{if}\PY{+w}{ }\PY{n}{seats}\PY{p}{[}\PY{n}{next\PYZus{}passenger}\PY{p}{]}\PY{p}{.}\PY{n}{is\PYZus{}none}\PY{p}{(}\PY{p}{)}\PY{+w}{ }\PY{p}{\PYZob{}}
\PY{+w}{                }\PY{n}{seats}\PY{p}{[}\PY{n}{next\PYZus{}passenger}\PY{p}{]}\PY{+w}{ }\PY{o}{=}\PY{+w}{ }\PY{n+nb}{Some}\PY{p}{(}\PY{n}{next\PYZus{}passenger}\PY{p}{)}\PY{p}{;}
\PY{+w}{                }\PY{n}{empty\PYZus{}seats}\PY{p}{.}\PY{n}{retain}\PY{p}{(}\PY{o}{|}\PY{n}{i}\PY{o}{|}\PY{+w}{ }\PY{o}{*}\PY{n}{i}\PY{+w}{ }\PY{o}{!}\PY{o}{=}\PY{+w}{ }\PY{n}{next\PYZus{}passenger}\PY{p}{)}\PY{p}{;}
\PY{+w}{            }\PY{p}{\PYZcb{}}\PY{+w}{ }\PY{k}{else}\PY{+w}{ }\PY{p}{\PYZob{}}\PY{+w}{ }\PY{c+c1}{// иначе он занимает случайное место}
\PY{+w}{                }\PY{c+c1}{// Он выбирает одно из мест, которые остались пустыми.}
\PY{+w}{                }\PY{k+kd}{let}\PY{+w}{ }\PY{n}{random\PYZus{}empty\PYZus{}seat}\PY{+w}{ }\PY{o}{=}\PY{+w}{ }\PY{o}{*}\PY{n}{empty\PYZus{}seats}\PY{p}{.}\PY{n}{iter}\PY{p}{(}\PY{p}{)}\PY{p}{.}\PY{n}{choose}\PY{p}{(}\PY{n}{rng}\PY{p}{)}\PY{p}{.}\PY{n}{unwrap}\PY{p}{(}\PY{p}{)}\PY{p}{;}
\PY{+w}{                }\PY{n}{seats}\PY{p}{[}\PY{n}{random\PYZus{}empty\PYZus{}seat}\PY{p}{]}\PY{+w}{ }\PY{o}{=}\PY{+w}{ }\PY{n+nb}{Some}\PY{p}{(}\PY{n}{next\PYZus{}passenger}\PY{p}{)}\PY{p}{;}
\PY{+w}{                }\PY{n}{empty\PYZus{}seats}\PY{p}{.}\PY{n}{retain}\PY{p}{(}\PY{o}{|}\PY{n}{i}\PY{o}{|}\PY{+w}{ }\PY{o}{*}\PY{n}{i}\PY{+w}{ }\PY{o}{!}\PY{o}{=}\PY{+w}{ }\PY{n}{random\PYZus{}empty\PYZus{}seat}\PY{p}{)}\PY{p}{;}
\PY{+w}{            }\PY{p}{\PYZcb{}}
\PY{+w}{        }\PY{p}{\PYZcb{}}
\PY{+w}{        }
\PY{+w}{        }\PY{c+c1}{// Наконец, мы смотрим \PYZhy{}\PYZhy{} место последнего пассажира занято ли?}
\PY{+w}{        }\PY{k+kd}{let}\PY{+w}{ }\PY{n}{last\PYZus{}passenger}\PY{+w}{ }\PY{o}{=}\PY{+w}{ }\PY{n}{passengers}\PY{p}{.}\PY{n}{pop}\PY{p}{(}\PY{p}{)}\PY{p}{.}\PY{n}{unwrap}\PY{p}{(}\PY{p}{)}\PY{p}{;}
\PY{+w}{        }\PY{n}{seats}\PY{p}{[}\PY{n}{last\PYZus{}passenger}\PY{p}{]}\PY{p}{.}\PY{n}{is\PYZus{}none}\PY{p}{(}\PY{p}{)}
\PY{+w}{    }\PY{p}{\PYZcb{}}
\PY{+w}{    }
\PY{+w}{    }\PY{k}{fn} \PY{n+nf}{desired\PYZus{}outcomes}\PY{p}{(}\PY{o}{\PYZam{}}\PY{n+nb+bp}{self}\PY{p}{)}\PY{+w}{ }\PYZhy{}\PYZgt{} \PY{n+nb}{Vec}\PY{o}{\PYZlt{}}\PY{n+nb+bp}{Self}::\PY{n}{Outcome}\PY{o}{\PYZgt{}}\PY{+w}{ }\PY{p}{\PYZob{}}\PY{n+nf+fm}{vec!}\PY{p}{[}\PY{k+kc}{true}\PY{p}{]}\PY{p}{\PYZcb{}}
\PY{p}{\PYZcb{}}

\PY{k+kd}{let}\PY{+w}{ }\PY{n}{a}\PY{+w}{ }\PY{o}{=}\PY{+w}{ }\PY{n}{Task4}\PY{p}{(}\PY{l+m+mi}{220}\PY{p}{)}\PY{p}{;}
\PY{k+kd}{let}\PY{+w}{ }\PY{n}{prob}\PY{+w}{ }\PY{o}{=}\PY{+w}{ }\PY{n}{a}\PY{p}{.}\PY{n}{probability\PYZus{}of\PYZus{}desired}\PY{p}{(}\PY{l+m+mi}{1\PYZus{}000\PYZus{}000}\PY{p}{)}\PY{p}{;}
\PY{n+nf+fm}{println!}\PY{p}{(}\PY{l+s}{\PYZdq{}}\PY{l+s}{\PYZob{}\PYZcb{} \PYZhy{}\PYZhy{} \PYZob{}\PYZcb{}}\PY{l+s}{\PYZdq{}}\PY{p}{,}\PY{+w}{ }\PY{n}{a}\PY{p}{.}\PY{n}{description}\PY{p}{(}\PY{p}{)}\PY{p}{,}\PY{+w}{ }\PY{n}{prob}\PY{p}{)}\PY{p}{;}
\PY{n}{answers}\PY{p}{.}\PY{n}{task\PYZus{}4}\PY{+w}{ }\PY{o}{=}\PY{+w}{ }\PY{n+nb}{Some}\PY{p}{(}\PY{n}{prob}\PY{p}{)}\PY{p}{;}
\end{Verbatim}
\end{tcolorbox}

    \begin{Verbatim}[commandchars=\\\{\}]
В самолете 220 мест, и все билеты проданы пассажирам. Первым в самолет заходит
рассеянный учёный и, не посмотрев на билет, занимает первое попавшееся место.
Далее пассажиры
входят по одному. Если вошедший видит, что его место свободно, он занимает свое
место. Если же
место занято, то вошедший занимает первое попавшееся свободное место. Найдите
вероятность того,
что пассажир, вошедший последним, займет место согласно своему билету. --
0.49964
    \end{Verbatim}

            \begin{tcolorbox}[breakable, size=fbox, boxrule=.5pt, pad at break*=1mm, opacityfill=0]
\prompt{Out}{outcolor}{8}{\boxspacing}
\begin{Verbatim}[commandchars=\\\{\}]
Timing: true
sccache: true

\end{Verbatim}
\end{tcolorbox}
        
    \begin{tcolorbox}[breakable, size=fbox, boxrule=1pt, pad at break*=1mm,colback=cellbackground, colframe=cellborder]
\prompt{In}{incolor}{9}{\boxspacing}
\begin{Verbatim}[commandchars=\\\{\}]
:\PY{n+nc}{timing}\PY{+w}{ }\PY{l+m+mi}{1}
:\PY{n+nc}{sccache}\PY{+w}{ }\PY{l+m+mi}{1}
:\PY{n+nc}{dep}\PY{+w}{ }\PY{n}{rand}\PY{+w}{ }\PY{o}{=}\PY{+w}{ }\PY{l+s}{\PYZdq{}}\PY{l+s}{0.8.5}\PY{l+s}{\PYZdq{}}

\PY{k}{struct} \PY{n+nc}{Task5}\PY{p}{(}\PY{k+kt}{f64}\PY{p}{)}\PY{p}{;}

\PY{k}{impl}\PY{+w}{ }\PY{n}{PartialExperiment}\PY{+w}{ }\PY{k}{for}\PY{+w}{ }\PY{n}{Task5}\PY{+w}{ }\PY{p}{\PYZob{}}
\PY{+w}{    }\PY{k}{type} \PY{n+nc}{Outcome}\PY{+w}{ }\PY{o}{=}\PY{+w}{ }\PY{k+kt}{f64}\PY{p}{;}
\PY{+w}{    }\PY{k}{fn} \PY{n+nf}{description}\PY{p}{(}\PY{o}{\PYZam{}}\PY{n+nb+bp}{self}\PY{p}{)}\PY{+w}{ }\PYZhy{}\PYZgt{} \PY{n+nb}{String} \PY{p}{\PYZob{}}
\PY{+w}{        }\PY{n+nf+fm}{format!}\PY{p}{(}\PY{l+s}{\PYZdq{}}\PY{l+s}{Диаметр круга имеет равномерное распределение на [0,\PYZob{}\PYZcb{}]. Чему равна средняя площадь}
\PY{l+s}{круга?}\PY{l+s}{\PYZdq{}}\PY{p}{,}\PY{+w}{ }\PY{n+nb+bp}{self}\PY{p}{.}\PY{l+m+mi}{0}\PY{p}{)}
\PY{+w}{    }\PY{p}{\PYZcb{}}
\PY{+w}{    }\PY{k}{fn} \PY{n+nf}{try\PYZus{}it}\PY{o}{\PYZlt{}}\PY{n}{T}: \PY{n+nc}{rand}::\PY{n}{Rng}\PY{o}{\PYZgt{}}\PY{p}{(}\PY{o}{\PYZam{}}\PY{n+nb+bp}{self}\PY{p}{,}\PY{+w}{ }\PY{n}{rng}: \PY{k+kp}{\PYZam{}}\PY{n+nc}{mut}\PY{+w}{ }\PY{n}{T}\PY{p}{)}\PY{+w}{ }\PYZhy{}\PYZgt{} \PY{n+nc}{Self}::\PY{n}{Outcome}\PY{+w}{ }\PY{p}{\PYZob{}}
\PY{+w}{        }\PY{c+c1}{// Мы выбираем диаметр, умножая случайное число от 0 до 1 на максимальный диаметр.}
\PY{+w}{        }\PY{c+c1}{// Это дает случайное число от 0 до максимума.}
\PY{+w}{        }\PY{k+kd}{let}\PY{+w}{ }\PY{n}{diameter}\PY{+w}{ }\PY{o}{=}\PY{+w}{ }\PY{n}{rng}\PY{p}{.}\PY{n}{gen}::\PY{o}{\PYZlt{}}\PY{k+kt}{f64}\PY{o}{\PYZgt{}}\PY{p}{(}\PY{p}{)}\PY{+w}{ }\PY{o}{*}\PY{+w}{ }\PY{n+nb+bp}{self}\PY{p}{.}\PY{l+m+mi}{0}\PY{p}{;}
\PY{+w}{        }\PY{k+kd}{let}\PY{+w}{ }\PY{n}{radius}\PY{+w}{ }\PY{o}{=}\PY{+w}{ }\PY{n}{diameter}\PY{+w}{ }\PY{o}{/}\PY{+w}{ }\PY{l+m+mf}{2.0}\PY{p}{;}
\PY{+w}{        }\PY{c+c1}{// Площадь круга: \PYZbs{}pi R\PYZca{}2  }
\PY{+w}{        }\PY{k+kd}{let}\PY{+w}{ }\PY{n}{area}\PY{+w}{ }\PY{o}{=}\PY{+w}{ }\PY{n}{std}::\PY{k+kt}{f64}::\PY{n}{consts}::\PY{n}{PI}\PY{+w}{ }\PY{o}{*}\PY{+w}{ }\PY{n}{radius}\PY{+w}{ }\PY{o}{*}\PY{+w}{ }\PY{n}{radius}\PY{p}{;}
\PY{+w}{        }\PY{n}{area}
\PY{+w}{    }\PY{p}{\PYZcb{}}
\PY{p}{\PYZcb{}}

\PY{k+kd}{let}\PY{+w}{ }\PY{n}{a}\PY{+w}{ }\PY{o}{=}\PY{+w}{ }\PY{n}{Task5}\PY{p}{(}\PY{l+m+mf}{5.0}\PY{p}{)}\PY{p}{;}
\PY{k+kd}{let}\PY{+w}{ }\PY{n}{output}\PY{+w}{ }\PY{o}{=}\PY{+w}{ }\PY{p}{\PYZob{}}
\PY{+w}{    }\PY{c+c1}{// Собираем результаты опыта}
\PY{+w}{    }\PY{k+kd}{let}\PY{+w}{ }\PY{n}{stats}\PY{+w}{ }\PY{o}{=}\PY{+w}{ }\PY{n}{a}\PY{p}{.}\PY{n}{collect\PYZus{}stats}\PY{p}{(}\PY{l+m+mi}{100\PYZus{}000}\PY{p}{)}\PY{p}{;}
\PY{+w}{    }\PY{c+c1}{// и считаем среднее значение}
\PY{+w}{    }\PY{k+kd}{let}\PY{+w}{ }\PY{n}{avg}: \PY{k+kt}{f64} \PY{o}{=}\PY{+w}{ }\PY{n}{stats}\PY{p}{.}\PY{n}{iter}\PY{p}{(}\PY{p}{)}\PY{p}{.}\PY{n}{sum}::\PY{o}{\PYZlt{}}\PY{k+kt}{f64}\PY{o}{\PYZgt{}}\PY{p}{(}\PY{p}{)}\PY{+w}{ }\PY{o}{/}\PY{+w}{ }\PY{p}{(}\PY{n}{stats}\PY{p}{.}\PY{n}{len}\PY{p}{(}\PY{p}{)}\PY{+w}{ }\PY{k}{as}\PY{+w}{ }\PY{k+kt}{f64}\PY{p}{)}\PY{p}{;}
\PY{+w}{    }\PY{n}{avg}
\PY{p}{\PYZcb{}}\PY{p}{;}
\PY{n+nf+fm}{println!}\PY{p}{(}\PY{l+s}{\PYZdq{}}\PY{l+s}{\PYZob{}\PYZcb{} \PYZhy{}\PYZhy{} \PYZob{}\PYZcb{}}\PY{l+s}{\PYZdq{}}\PY{p}{,}\PY{+w}{ }\PY{n}{a}\PY{p}{.}\PY{n}{description}\PY{p}{(}\PY{p}{)}\PY{p}{,}\PY{+w}{ }\PY{n}{output}\PY{p}{)}\PY{p}{;}
\PY{n}{answers}\PY{p}{.}\PY{n}{task\PYZus{}5}\PY{+w}{ }\PY{o}{=}\PY{+w}{ }\PY{n+nb}{Some}\PY{p}{(}\PY{n}{output}\PY{p}{)}\PY{p}{;}
\end{Verbatim}
\end{tcolorbox}

    \begin{Verbatim}[commandchars=\\\{\}]
Диаметр круга имеет равномерное распределение на [0,5]. Чему равна средняя
площадь
круга? -- 6.537402833236494
    \end{Verbatim}

            \begin{tcolorbox}[breakable, size=fbox, boxrule=.5pt, pad at break*=1mm, opacityfill=0]
\prompt{Out}{outcolor}{9}{\boxspacing}
\begin{Verbatim}[commandchars=\\\{\}]
Timing: false
sccache: true

\end{Verbatim}
\end{tcolorbox}
        
    \begin{tcolorbox}[breakable, size=fbox, boxrule=1pt, pad at break*=1mm,colback=cellbackground, colframe=cellborder]
\prompt{In}{incolor}{10}{\boxspacing}
\begin{Verbatim}[commandchars=\\\{\}]
:\PY{n+nc}{timing}\PY{+w}{ }\PY{l+m+mi}{1}
:\PY{n+nc}{sccache}\PY{+w}{ }\PY{l+m+mi}{1}
:\PY{n+nc}{dep}\PY{+w}{ }\PY{n}{rand}\PY{+w}{ }\PY{o}{=}\PY{+w}{ }\PY{l+s}{\PYZdq{}}\PY{l+s}{0.8.5}\PY{l+s}{\PYZdq{}}

\PY{l+s+sd}{/// Для этой задачи нужно хранить количество шагов,}
\PY{l+s+sd}{/// а также нижнюю границу интервала случайных чисел.}
\PY{k}{struct} \PY{n+nc}{Task6}\PY{+w}{ }\PY{p}{\PYZob{}}
\PY{+w}{    }\PY{k}{pub}\PY{+w}{ }\PY{n}{remaining\PYZus{}n}: \PY{k+kt}{usize}\PY{p}{,}
\PY{+w}{    }\PY{k}{pub}\PY{+w}{ }\PY{n}{last\PYZus{}answer}: \PY{k+kt}{f64}\PY{p}{,}
\PY{p}{\PYZcb{}}

\PY{k}{impl}\PY{+w}{ }\PY{n}{Task6}\PY{+w}{ }\PY{p}{\PYZob{}}
\PY{+w}{    }\PY{l+s+sd}{/// Сначала нижняя граница равна нулю. }
\PY{+w}{    }\PY{k}{pub}\PY{+w}{ }\PY{k}{fn} \PY{n+nf}{new}\PY{p}{(}\PY{n}{n}: \PY{k+kt}{usize}\PY{p}{)}\PY{+w}{ }\PYZhy{}\PYZgt{} \PY{n+nc}{Self}\PY{+w}{ }\PY{p}{\PYZob{}}
\PY{+w}{        }\PY{n}{Task6}\PY{+w}{ }\PY{p}{\PYZob{}}
\PY{+w}{            }\PY{n}{remaining\PYZus{}n}: \PY{n+nc}{n}\PY{p}{,}
\PY{+w}{            }\PY{n}{last\PYZus{}answer}: \PY{l+m+mf}{0.0}\PY{k}{f64}\PY{p}{,}
\PY{+w}{        }\PY{p}{\PYZcb{}}
\PY{+w}{    }\PY{p}{\PYZcb{}}
\PY{+w}{    }
\PY{+w}{    }\PY{l+s+sd}{/// На каждом шаге мы меняем границу диапазона}
\PY{+w}{    }\PY{l+s+sd}{/// и возвращаем эту границу в результате этой функции.}
\PY{+w}{    }\PY{k}{pub}\PY{+w}{ }\PY{k}{fn} \PY{n+nf}{run\PYZus{}single\PYZus{}step}\PY{o}{\PYZlt{}}\PY{n}{T}: \PY{n+nc}{rand}::\PY{n}{Rng}\PY{o}{\PYZgt{}}\PY{p}{(}\PY{o}{\PYZam{}}\PY{k}{mut}\PY{+w}{ }\PY{n+nb+bp}{self}\PY{p}{,}\PY{+w}{ }\PY{n}{rng}: \PY{k+kp}{\PYZam{}}\PY{n+nc}{mut}\PY{+w}{ }\PY{n}{T}\PY{p}{)}\PY{+w}{ }\PYZhy{}\PYZgt{} \PY{n+nb}{Option}\PY{o}{\PYZlt{}}\PY{k+kt}{f64}\PY{o}{\PYZgt{}}\PY{+w}{ }\PY{p}{\PYZob{}}
\PY{+w}{        }\PY{k}{if}\PY{+w}{ }\PY{n+nb+bp}{self}\PY{p}{.}\PY{n}{remaining\PYZus{}n}\PY{+w}{ }\PY{o}{=}\PY{o}{=}\PY{+w}{ }\PY{l+m+mi}{0}\PY{+w}{ }\PY{p}{\PYZob{}}\PY{k}{return}\PY{+w}{ }\PY{n+nb}{None}\PY{p}{;}\PY{p}{\PYZcb{}}
\PY{+w}{        }\PY{c+c1}{// Мы создаем случайное число}
\PY{+w}{        }\PY{k+kd}{let}\PY{+w}{ }\PY{n}{random}: \PY{k+kt}{f64} \PY{o}{=}\PY{+w}{ }\PY{n}{rng}\PY{p}{.}\PY{n}{gen}\PY{p}{(}\PY{p}{)}\PY{p}{;}
\PY{+w}{        }\PY{c+c1}{// в диапазоне от нижней границы до 1}
\PY{+w}{        }\PY{k+kd}{let}\PY{+w}{ }\PY{n}{range\PYZus{}size}\PY{+w}{ }\PY{o}{=}\PY{+w}{ }\PY{l+m+mf}{1.0}\PY{+w}{ }\PY{o}{\PYZhy{}}\PY{+w}{ }\PY{n+nb+bp}{self}\PY{p}{.}\PY{n}{last\PYZus{}answer}\PY{p}{;}
\PY{+w}{        }\PY{c+c1}{// где минимальное число \PYZhy{}\PYZhy{} это нижняя граница}
\PY{+w}{        }\PY{k+kd}{let}\PY{+w}{ }\PY{n}{random}\PY{+w}{ }\PY{o}{=}\PY{+w}{ }\PY{n+nb+bp}{self}\PY{p}{.}\PY{n}{last\PYZus{}answer}\PY{+w}{ }\PY{o}{+}\PY{+w}{ }\PY{p}{(}\PY{n}{random}\PY{+w}{ }\PY{o}{*}\PY{+w}{ }\PY{n}{range\PYZus{}size}\PY{p}{)}\PY{p}{;}
\PY{+w}{        }\PY{n+nb+bp}{self}\PY{p}{.}\PY{n}{last\PYZus{}answer}\PY{+w}{ }\PY{o}{=}\PY{+w}{ }\PY{n}{random}\PY{p}{;}
\PY{+w}{        }\PY{n+nb+bp}{self}\PY{p}{.}\PY{n}{remaining\PYZus{}n}\PY{+w}{ }\PY{o}{\PYZhy{}}\PY{o}{=}\PY{+w}{ }\PY{l+m+mi}{1}\PY{p}{;}
\PY{+w}{        }\PY{n+nb}{Some}\PY{p}{(}\PY{n}{random}\PY{p}{)}
\PY{+w}{    }\PY{p}{\PYZcb{}}
\PY{+w}{    }
\PY{+w}{    }\PY{l+s+sd}{/// Чтобы взять все шаги, мы перемножаем все нижние границы.}
\PY{+w}{    }\PY{k}{pub}\PY{+w}{ }\PY{k}{fn} \PY{n+nf}{run\PYZus{}until\PYZus{}completion}\PY{o}{\PYZlt{}}\PY{n}{T}: \PY{n+nc}{rand}::\PY{n}{Rng}\PY{o}{\PYZgt{}}\PY{p}{(}\PY{o}{\PYZam{}}\PY{k}{mut}\PY{+w}{ }\PY{n+nb+bp}{self}\PY{p}{,}\PY{+w}{ }\PY{n}{rng}: \PY{k+kp}{\PYZam{}}\PY{n+nc}{mut}\PY{+w}{ }\PY{n}{T}\PY{p}{)}\PY{+w}{ }\PYZhy{}\PYZgt{} \PY{k+kt}{f64} \PY{p}{\PYZob{}}
\PY{+w}{        }\PY{c+c1}{// Начинаем с единицы \PYZhy{}\PYZhy{} нулевого элемента относительно умножения}
\PY{+w}{        }\PY{k+kd}{let}\PY{+w}{ }\PY{k}{mut}\PY{+w}{ }\PY{n}{multiplication}\PY{+w}{ }\PY{o}{=}\PY{+w}{ }\PY{l+m+mf}{1.0}\PY{k}{f64}\PY{p}{;}
\PY{+w}{        }\PY{c+c1}{// На каждом шаге, умножаем это значение на нижнуюю границу}
\PY{+w}{        }\PY{k}{while}\PY{+w}{ }\PY{k+kd}{let}\PY{+w}{ }\PY{n+nb}{Some}\PY{p}{(}\PY{n}{value}\PY{p}{)}\PY{+w}{ }\PY{o}{=}\PY{+w}{ }\PY{n+nb+bp}{self}\PY{p}{.}\PY{n}{run\PYZus{}single\PYZus{}step}\PY{p}{(}\PY{n}{rng}\PY{p}{)}\PY{+w}{ }\PY{p}{\PYZob{}}
\PY{+w}{            }\PY{n}{multiplication}\PY{+w}{ }\PY{o}{*}\PY{o}{=}\PY{+w}{ }\PY{n}{value}\PY{p}{;}
\PY{+w}{        }\PY{p}{\PYZcb{}}
\PY{+w}{        }\PY{n}{multiplication}
\PY{+w}{    }\PY{p}{\PYZcb{}}
\PY{p}{\PYZcb{}}

\PY{l+s+sd}{/// Считаем среднее значение ответа}
\PY{k+kd}{let}\PY{+w}{ }\PY{n}{avg}\PY{+w}{ }\PY{o}{=}\PY{+w}{ }\PY{p}{\PYZob{}}
\PY{+w}{    }\PY{k+kd}{let}\PY{+w}{ }\PY{k}{mut}\PY{+w}{ }\PY{n}{rng}\PY{+w}{ }\PY{o}{=}\PY{+w}{ }\PY{n}{rand}::\PY{n}{thread\PYZus{}rng}\PY{p}{(}\PY{p}{)}\PY{p}{;}
\PY{+w}{    }\PY{k+kd}{let}\PY{+w}{ }\PY{k}{mut}\PY{+w}{ }\PY{n}{sum\PYZus{}of\PYZus{}answers}\PY{+w}{ }\PY{o}{=}\PY{+w}{ }\PY{l+m+mf}{0.0}\PY{p}{;}
\PY{+w}{    }\PY{k+kd}{let}\PY{+w}{ }\PY{k}{mut}\PY{+w}{ }\PY{n}{count}\PY{+w}{ }\PY{o}{=}\PY{+w}{ }\PY{l+m+mi}{0}\PY{p}{;}
\PY{+w}{    }\PY{k}{for}\PY{+w}{ }\PY{n}{\PYZus{}}\PY{+w}{ }\PY{k}{in}\PY{+w}{ }\PY{l+m+mi}{0}\PY{o}{..}\PY{l+m+mi}{10\PYZus{}000}\PY{+w}{ }\PY{p}{\PYZob{}}
\PY{+w}{        }\PY{k+kd}{let}\PY{+w}{ }\PY{k}{mut}\PY{+w}{ }\PY{n}{experiment}\PY{+w}{ }\PY{o}{=}\PY{+w}{ }\PY{n}{Task6}::\PY{n}{new}\PY{p}{(}\PY{l+m+mi}{100\PYZus{}000}\PY{p}{)}\PY{p}{;}
\PY{+w}{        }\PY{n}{sum\PYZus{}of\PYZus{}answers}\PY{+w}{ }\PY{o}{+}\PY{o}{=}\PY{+w}{ }\PY{n}{experiment}\PY{p}{.}\PY{n}{run\PYZus{}until\PYZus{}completion}\PY{p}{(}\PY{o}{\PYZam{}}\PY{k}{mut}\PY{+w}{ }\PY{n}{rng}\PY{p}{)}\PY{p}{;}
\PY{+w}{        }\PY{n}{count}\PY{+w}{ }\PY{o}{+}\PY{o}{=}\PY{+w}{ }\PY{l+m+mi}{1}\PY{p}{;}
\PY{+w}{    }\PY{p}{\PYZcb{}}
\PY{+w}{    }\PY{n}{sum\PYZus{}of\PYZus{}answers}\PY{+w}{ }\PY{o}{/}\PY{+w}{ }\PY{p}{(}\PY{n}{count}\PY{+w}{ }\PY{k}{as}\PY{+w}{ }\PY{k+kt}{f64}\PY{p}{)}
\PY{p}{\PYZcb{}}\PY{p}{;}
\PY{n+nf+fm}{println!}\PY{p}{(}\PY{l+s}{\PYZdq{}}\PY{l+s}{Пусть x(1) выбирается наугад из интервала (0,1). Далее x(2) выбирается наугад из интервала}
\PY{l+s}{(x(1),1). Далее x(3) выбирается наугад из интервала (x(2),1) и так далее, т.е. x(n+1) выбирается наугад}
\PY{l+s}{из интервала (x(n),1). Чему равно среднее значение произведения x(1)x(2)...x(n) если n велико? \PYZhy{}\PYZhy{} \PYZob{}\PYZcb{}}\PY{l+s}{\PYZdq{}}\PY{p}{,}\PY{+w}{ }\PY{n}{avg}\PY{p}{)}\PY{p}{;}
\PY{n}{answers}\PY{p}{.}\PY{n}{task\PYZus{}6}\PY{+w}{ }\PY{o}{=}\PY{+w}{ }\PY{n+nb}{Some}\PY{p}{(}\PY{n}{avg}\PY{p}{)}\PY{p}{;}
\end{Verbatim}
\end{tcolorbox}

    \begin{Verbatim}[commandchars=\\\{\}]
Пусть x(1) выбирается наугад из интервала (0,1). Далее x(2) выбирается наугад из
интервала
(x(1),1). Далее x(3) выбирается наугад из интервала (x(2),1) и так далее, т.е.
x(n+1) выбирается наугад
из интервала (x(n),1). Чему равно среднее значение произведения x(1)x(2){\ldots}x(n)
если n велико? -- 0.3688154219891535
    \end{Verbatim}

            \begin{tcolorbox}[breakable, size=fbox, boxrule=.5pt, pad at break*=1mm, opacityfill=0]
\prompt{Out}{outcolor}{10}{\boxspacing}
\begin{Verbatim}[commandchars=\\\{\}]
Timing: true
sccache: true

\end{Verbatim}
\end{tcolorbox}
        
    \begin{tcolorbox}[breakable, size=fbox, boxrule=1pt, pad at break*=1mm,colback=cellbackground, colframe=cellborder]
\prompt{In}{incolor}{11}{\boxspacing}
\begin{Verbatim}[commandchars=\\\{\}]
:\PY{n+nc}{timing}\PY{+w}{ }\PY{l+m+mi}{1}
:\PY{n+nc}{sccache}\PY{+w}{ }\PY{l+m+mi}{1}
:\PY{n+nc}{dep}\PY{+w}{ }\PY{n}{rand}\PY{+w}{ }\PY{o}{=}\PY{+w}{ }\PY{l+s}{\PYZdq{}}\PY{l+s}{0.8.5}\PY{l+s}{\PYZdq{}}
:\PY{n+nc}{dep}\PY{+w}{ }\PY{n}{rand\PYZus{}distr}\PY{+w}{ }\PY{o}{=}\PY{+w}{ }\PY{l+s}{\PYZdq{}}\PY{l+s}{0.4.3}\PY{l+s}{\PYZdq{}}
:\PY{n+nc}{dep}\PY{+w}{ }\PY{n}{plotters}\PY{+w}{ }\PY{o}{=}\PY{+w}{ }\PY{p}{\PYZob{}}\PY{+w}{ }\PY{n}{version}\PY{o}{=}\PY{l+s}{\PYZdq{}}\PY{l+s}{0.3.4}\PY{l+s}{\PYZdq{}}\PY{p}{,}\PY{+w}{ }\PY{n}{default\PYZus{}features}\PY{+w}{ }\PY{o}{=}\PY{+w}{ }\PY{k+kc}{false}\PY{p}{,}\PY{+w}{ }\PY{n}{features}\PY{+w}{ }\PY{o}{=}\PY{+w}{ }\PY{p}{[}\PY{l+s}{\PYZdq{}}\PY{l+s}{evcxr}\PY{l+s}{\PYZdq{}}\PY{p}{,}\PY{+w}{ }\PY{l+s}{\PYZdq{}}\PY{l+s}{all\PYZus{}series}\PY{l+s}{\PYZdq{}}\PY{p}{]}\PY{+w}{ }\PY{p}{\PYZcb{}}

\PY{k}{use}\PY{+w}{ }\PY{n}{rand\PYZus{}distr}::\PY{n}{Distribution}\PY{p}{;}

\PY{l+s+sd}{/// Для этого задания мы должны хранить:}
\PY{k}{struct} \PY{n+nc}{Task7}\PY{+w}{ }\PY{p}{\PYZob{}}
\PY{+w}{    }\PY{l+s+sd}{/// количество людей в городе}
\PY{+w}{    }\PY{k}{pub}\PY{+w}{ }\PY{n}{citizen\PYZus{}count}: \PY{k+kt}{u64}\PY{p}{,}
\PY{+w}{    }\PY{l+s+sd}{/// вероятность того, что один человек будет на поезде}
\PY{+w}{    }\PY{k}{pub}\PY{+w}{ }\PY{n}{citizen\PYZus{}train\PYZus{}probability}: \PY{k+kt}{f64}\PY{p}{,}
\PY{+w}{    }\PY{l+s+sd}{/// текущий размер поезда}
\PY{+w}{    }\PY{k}{pub}\PY{+w}{ }\PY{n}{current\PYZus{}train\PYZus{}capacity}: \PY{k+kt}{u64}\PY{p}{,}
\PY{p}{\PYZcb{}}

\PY{k}{impl}\PY{+w}{ }\PY{n}{Task7}\PY{+w}{ }\PY{p}{\PYZob{}}
\PY{+w}{        }\PY{k}{pub}\PY{+w}{ }\PY{k}{fn} \PY{n+nf}{new}\PY{p}{(}\PY{p}{)}\PY{+w}{ }\PYZhy{}\PYZgt{} \PY{n+nc}{Self}\PY{+w}{ }\PY{p}{\PYZob{}}
\PY{+w}{        }\PY{n}{Task7}\PY{+w}{ }\PY{p}{\PYZob{}}
\PY{+w}{            }\PY{n}{citizen\PYZus{}count}: \PY{l+m+mi}{2500}\PY{p}{,}
\PY{+w}{            }\PY{n}{citizen\PYZus{}train\PYZus{}probability}: \PY{l+m+mf}{6.0}\PY{+w}{ }\PY{o}{/}\PY{+w}{ }\PY{l+m+mf}{30.0}\PY{p}{,}
\PY{+w}{            }\PY{c+c1}{// Сначала пусть размер поезда равен нулю. }
\PY{+w}{            }\PY{n}{current\PYZus{}train\PYZus{}capacity}: \PY{l+m+mi}{0}\PY{p}{,}
\PY{+w}{        }\PY{p}{\PYZcb{}}
\PY{+w}{    }\PY{p}{\PYZcb{}}
\PY{+w}{    }
\PY{+w}{    }\PY{l+s+sd}{/// Для определенного размера мы считаем вероятность переполнения поезда}
\PY{+w}{    }\PY{k}{pub}\PY{+w}{ }\PY{k}{fn} \PY{n+nf}{overfill\PYZus{}probability}\PY{o}{\PYZlt{}}\PY{n}{T}: \PY{n+nc}{rand}::\PY{n}{Rng}\PY{o}{\PYZgt{}}\PY{p}{(}\PY{o}{\PYZam{}}\PY{n+nb+bp}{self}\PY{p}{,}\PY{+w}{ }\PY{n}{rng}: \PY{k+kp}{\PYZam{}}\PY{n+nc}{mut}\PY{+w}{ }\PY{n}{T}\PY{p}{,}\PY{+w}{ }\PY{n}{samples}: \PY{k+kt}{usize}\PY{p}{)}\PY{+w}{ }\PYZhy{}\PYZgt{} \PY{k+kt}{f64} \PY{p}{\PYZob{}}
\PY{+w}{        }\PY{c+c1}{// Количество людей на поезде в один день следует биномиальному распределению}
\PY{+w}{        }\PY{k+kd}{let}\PY{+w}{ }\PY{n}{train\PYZus{}users\PYZus{}distribution}\PY{+w}{ }\PY{o}{=}\PY{+w}{ }\PY{n}{rand\PYZus{}distr}::\PY{n}{Binomial}::\PY{n}{new}\PY{p}{(}
\PY{+w}{            }\PY{n+nb+bp}{self}\PY{p}{.}\PY{n}{citizen\PYZus{}count}\PY{p}{,}\PY{+w}{ }\PY{n+nb+bp}{self}\PY{p}{.}\PY{n}{citizen\PYZus{}train\PYZus{}probability}
\PY{+w}{        }\PY{p}{)}\PY{p}{.}\PY{n}{unwrap}\PY{p}{(}\PY{p}{)}\PY{p}{;}
\PY{+w}{        }\PY{c+c1}{// Считаем количество раз, когда поезд переполнен}
\PY{+w}{        }\PY{k+kd}{let}\PY{+w}{ }\PY{k}{mut}\PY{+w}{ }\PY{n}{overfilled}\PY{+w}{ }\PY{o}{=}\PY{+w}{ }\PY{l+m+mi}{0}\PY{p}{;}
\PY{+w}{        }\PY{k}{for}\PY{+w}{ }\PY{n}{\PYZus{}}\PY{+w}{ }\PY{k}{in}\PY{+w}{ }\PY{l+m+mi}{0}\PY{o}{..}\PY{n}{samples}\PY{+w}{ }\PY{p}{\PYZob{}}
\PY{+w}{            }\PY{c+c1}{// Если измерение из распределения оказывается больше размера поезда,}
\PY{+w}{            }\PY{c+c1}{// считаем это как переполнение поезда}
\PY{+w}{            }\PY{k+kd}{let}\PY{+w}{ }\PY{n}{train\PYZus{}users}\PY{+w}{ }\PY{o}{=}\PY{+w}{ }\PY{n}{train\PYZus{}users\PYZus{}distribution}\PY{p}{.}\PY{n}{sample}\PY{p}{(}\PY{n}{rng}\PY{p}{)}\PY{p}{;}
\PY{+w}{            }\PY{k}{if}\PY{+w}{ }\PY{n}{train\PYZus{}users}\PY{+w}{ }\PY{o}{\PYZgt{}}\PY{+w}{ }\PY{n+nb+bp}{self}\PY{p}{.}\PY{n}{current\PYZus{}train\PYZus{}capacity}\PY{+w}{ }\PY{p}{\PYZob{}}
\PY{+w}{                }\PY{n}{overfilled}\PY{+w}{ }\PY{o}{+}\PY{o}{=}\PY{+w}{ }\PY{l+m+mi}{1}
\PY{+w}{            }\PY{p}{\PYZcb{}}
\PY{+w}{        }\PY{p}{\PYZcb{}}
\PY{+w}{        }\PY{p}{(}\PY{n}{overfilled}\PY{+w}{ }\PY{k}{as}\PY{+w}{ }\PY{k+kt}{f64}\PY{p}{)}\PY{+w}{ }\PY{o}{/}\PY{+w}{ }\PY{p}{(}\PY{n}{samples}\PY{+w}{ }\PY{k}{as}\PY{+w}{ }\PY{k+kt}{f64}\PY{p}{)}
\PY{+w}{    }\PY{p}{\PYZcb{}}
\PY{+w}{    }
\PY{+w}{    }\PY{l+s+sd}{/// На каждом шаге мы оцениваем разницу между вероятностями}
\PY{+w}{    }\PY{l+s+sd}{/// переполнения, которая наблюдалась и которая должна быть на самом деле,}
\PY{+w}{    }\PY{l+s+sd}{/// и изменяем размер поезда, чтобы совместить их.}
\PY{+w}{    }\PY{k}{pub}\PY{+w}{ }\PY{k}{fn} \PY{n+nf}{adjust\PYZus{}train\PYZus{}capacity}\PY{p}{(}\PY{o}{\PYZam{}}\PY{k}{mut}\PY{+w}{ }\PY{n+nb+bp}{self}\PY{p}{,}\PY{+w}{ }\PY{n}{desired}: \PY{k+kt}{f64}\PY{p}{,}\PY{+w}{ }\PY{n}{got}: \PY{k+kt}{f64}\PY{p}{)}\PY{+w}{ }\PY{p}{\PYZob{}}
\PY{+w}{        }\PY{c+c1}{// Погрешность \PYZhy{}\PYZhy{} это разница между двумя вероятностями}
\PY{+w}{        }\PY{k+kd}{let}\PY{+w}{ }\PY{n}{error}\PY{+w}{ }\PY{o}{=}\PY{+w}{ }\PY{n}{got}\PY{+w}{ }\PY{o}{\PYZhy{}}\PY{+w}{ }\PY{n}{desired}\PY{p}{;}
\PY{+w}{        }\PY{c+c1}{// Если погрешность отрицательная, значит поезд слишком большой}
\PY{+w}{        }\PY{k}{if}\PY{+w}{ }\PY{n}{error}\PY{+w}{ }\PY{o}{\PYZlt{}}\PY{+w}{ }\PY{l+m+mf}{0.0}\PY{+w}{ }\PY{p}{\PYZob{}}
\PY{+w}{            }\PY{n+nb+bp}{self}\PY{p}{.}\PY{n}{current\PYZus{}train\PYZus{}capacity}\PY{+w}{ }\PY{o}{\PYZhy{}}\PY{o}{=}\PY{+w}{ }\PY{l+m+mi}{1}\PY{p}{;}
\PY{+w}{        }\PY{p}{\PYZcb{}}\PY{+w}{ }\PY{k}{else}\PY{+w}{ }\PY{p}{\PYZob{}}\PY{+w}{ }\PY{c+c1}{// Иначе поезд слишком маленький}
\PY{+w}{            }\PY{n+nb+bp}{self}\PY{p}{.}\PY{n}{current\PYZus{}train\PYZus{}capacity}\PY{+w}{ }\PY{o}{+}\PY{o}{=}\PY{+w}{ }\PY{l+m+mi}{1}\PY{p}{;}
\PY{+w}{        }\PY{p}{\PYZcb{}}
\PY{+w}{        }
\PY{+w}{        }\PY{c+c1}{// (Для простоты я здесь использую шаг в одну единицу размера \PYZhy{}\PYZhy{} }
\PY{+w}{        }\PY{c+c1}{// для оптимальности этот шаг должен быть пропорциональным погрешности.)}
\PY{+w}{    }\PY{p}{\PYZcb{}}
\PY{p}{\PYZcb{}}

\PY{c+c1}{// Чтобы проверить, что ответ сошелся, мы рисуем график размера поезда от шагов.}
\PY{k}{extern}\PY{+w}{ }\PY{k}{crate}\PY{+w}{ }\PY{n}{plotters}\PY{p}{;}
\PY{k}{use}\PY{+w}{ }\PY{n}{plotters}::\PY{n}{prelude}::\PY{o}{*}\PY{p}{;}

\PY{k+kd}{let}\PY{+w}{ }\PY{n}{figure}\PY{+w}{ }\PY{o}{=}\PY{+w}{ }\PY{p}{\PYZob{}}
\PY{+w}{    }\PY{c+c1}{// Требуется достичь одной неудачи из 100 дней.}
\PY{+w}{    }\PY{k+kd}{let}\PY{+w}{ }\PY{n}{one\PYZus{}fail\PYZus{}per\PYZus{}days}\PY{+w}{ }\PY{o}{=}\PY{+w}{ }\PY{l+m+mf}{100.0}\PY{k}{f64}\PY{p}{;}
\PY{+w}{    }
\PY{+w}{    }\PY{k+kd}{let}\PY{+w}{ }\PY{k}{mut}\PY{+w}{ }\PY{n}{task}\PY{+w}{ }\PY{o}{=}\PY{+w}{ }\PY{n}{Task7}::\PY{n}{new}\PY{p}{(}\PY{p}{)}\PY{p}{;}
\PY{+w}{    }\PY{k+kd}{let}\PY{+w}{ }\PY{k}{mut}\PY{+w}{ }\PY{n}{rng}\PY{+w}{ }\PY{o}{=}\PY{+w}{ }\PY{n}{rand}::\PY{n}{thread\PYZus{}rng}\PY{p}{(}\PY{p}{)}\PY{p}{;}
\PY{+w}{    }\PY{k+kd}{let}\PY{+w}{ }\PY{n}{figure}\PY{+w}{ }\PY{o}{=}\PY{+w}{ }\PY{n}{evcxr\PYZus{}figure}\PY{p}{(}\PY{p}{(}\PY{l+m+mi}{640}\PY{p}{,}\PY{+w}{ }\PY{l+m+mi}{480}\PY{p}{)}\PY{p}{,}\PY{+w}{ }\PY{o}{|}\PY{n}{root}\PY{o}{|}\PY{p}{\PYZob{}}
\PY{+w}{        }\PY{c+c1}{// Инициализируем область для рисования графика}
\PY{+w}{        }\PY{n}{root}\PY{p}{.}\PY{n}{fill}\PY{p}{(}\PY{o}{\PYZam{}}\PY{n}{WHITE}\PY{p}{)}\PY{o}{?}\PY{p}{;}
\PY{+w}{        }\PY{k+kd}{let}\PY{+w}{ }\PY{n}{root}\PY{+w}{ }\PY{o}{=}\PY{+w}{ }\PY{n}{root}\PY{p}{.}\PY{n}{margin}\PY{p}{(}\PY{l+m+mi}{10}\PY{p}{,}\PY{+w}{ }\PY{l+m+mi}{10}\PY{p}{,}\PY{+w}{ }\PY{l+m+mi}{10}\PY{p}{,}\PY{+w}{ }\PY{l+m+mi}{10}\PY{p}{)}\PY{p}{;}
\PY{+w}{    }
\PY{+w}{        }\PY{c+c1}{// График имеет декартовы координаты (0\PYZhy{}\PYZhy{}2000)x(0\PYZhy{}\PYZhy{}2500)}
\PY{+w}{        }\PY{k+kd}{let}\PY{+w}{ }\PY{k}{mut}\PY{+w}{ }\PY{n}{chart}\PY{+w}{ }\PY{o}{=}\PY{+w}{ }\PY{n}{ChartBuilder}::\PY{n}{on}\PY{p}{(}\PY{o}{\PYZam{}}\PY{n}{root}\PY{p}{)}
\PY{+w}{        }\PY{p}{.}\PY{n}{x\PYZus{}label\PYZus{}area\PYZus{}size}\PY{p}{(}\PY{l+m+mi}{20}\PY{p}{)}
\PY{+w}{        }\PY{p}{.}\PY{n}{y\PYZus{}label\PYZus{}area\PYZus{}size}\PY{p}{(}\PY{l+m+mi}{40}\PY{p}{)}
\PY{+w}{        }\PY{p}{.}\PY{n}{build\PYZus{}cartesian\PYZus{}2d}\PY{p}{(}\PY{l+m+mi}{0}\PY{k}{f32}\PY{o}{..}\PY{l+m+mi}{2000}\PY{k}{f32}\PY{p}{,}\PY{+w}{ }\PY{l+m+mi}{0}\PY{k}{f32}\PY{o}{..}\PY{l+m+mi}{2500}\PY{k}{f32}\PY{p}{)}\PY{o}{?}\PY{p}{;}
\PY{+w}{        }
\PY{+w}{        }\PY{c+c1}{// У графика есть сетка с 10 делениями по Y\PYZhy{}оси}
\PY{+w}{         }\PY{n}{chart}\PY{p}{.}\PY{n}{configure\PYZus{}mesh}\PY{p}{(}\PY{p}{)}
\PY{+w}{        }\PY{p}{.}\PY{n}{y\PYZus{}labels}\PY{p}{(}\PY{l+m+mi}{10}\PY{p}{)}
\PY{+w}{        }\PY{p}{.}\PY{n}{light\PYZus{}line\PYZus{}style}\PY{p}{(}\PY{o}{\PYZam{}}\PY{n}{TRANSPARENT}\PY{p}{)}
\PY{+w}{        }\PY{p}{.}\PY{n}{disable\PYZus{}x\PYZus{}mesh}\PY{p}{(}\PY{p}{)}
\PY{+w}{        }\PY{p}{.}\PY{n}{draw}\PY{p}{(}\PY{p}{)}\PY{o}{?}\PY{p}{;}
\PY{+w}{        }
\PY{+w}{        }\PY{k+kd}{let}\PY{+w}{ }\PY{k}{mut}\PY{+w}{ }\PY{n}{points}\PY{+w}{ }\PY{o}{=}\PY{+w}{ }\PY{n+nf+fm}{vec!}\PY{p}{[}\PY{p}{]}\PY{p}{;}
\PY{+w}{        }
\PY{+w}{        }\PY{k}{for}\PY{+w}{ }\PY{n}{i}\PY{+w}{ }\PY{k}{in}\PY{+w}{ }\PY{l+m+mi}{0}\PY{o}{..}\PY{l+m+mi}{2000}\PY{+w}{ }\PY{p}{\PYZob{}}
\PY{+w}{            }\PY{c+c1}{// Записываем текущий размер поезда}
\PY{+w}{            }\PY{n}{points}\PY{p}{.}\PY{n}{push}\PY{p}{(}\PY{+w}{ }\PY{p}{(}\PY{n}{i}\PY{+w}{ }\PY{k}{as}\PY{+w}{ }\PY{k+kt}{f32}\PY{p}{,}\PY{+w}{ }\PY{n}{task}\PY{p}{.}\PY{n}{current\PYZus{}train\PYZus{}capacity}\PY{+w}{ }\PY{k}{as}\PY{+w}{ }\PY{k+kt}{f32}\PY{p}{)}\PY{+w}{ }\PY{p}{)}\PY{p}{;}
\PY{+w}{            }\PY{c+c1}{// Считаем вероятность переполнения поезда}
\PY{+w}{            }\PY{k+kd}{let}\PY{+w}{ }\PY{n}{overfill\PYZus{}prob}\PY{+w}{ }\PY{o}{=}\PY{+w}{ }\PY{n}{task}\PY{p}{.}\PY{n}{overfill\PYZus{}probability}\PY{p}{(}\PY{o}{\PYZam{}}\PY{k}{mut}\PY{+w}{ }\PY{n}{rng}\PY{p}{,}\PY{+w}{ }\PY{l+m+mi}{1000}\PY{p}{)}\PY{p}{;}
\PY{+w}{            }\PY{c+c1}{// Изменяем объем поезда относительно вероятности неудачи}
\PY{+w}{            }\PY{n}{task}\PY{p}{.}\PY{n}{adjust\PYZus{}train\PYZus{}capacity}\PY{p}{(}\PY{l+m+mf}{1.0}\PY{+w}{ }\PY{o}{/}\PY{+w}{ }\PY{n}{one\PYZus{}fail\PYZus{}per\PYZus{}days}\PY{p}{,}\PY{+w}{ }\PY{n}{overfill\PYZus{}prob}\PY{p}{)}\PY{p}{;}
\PY{+w}{        }\PY{p}{\PYZcb{}}
\PY{+w}{        }\PY{c+c1}{// Рисуем точки как красную линию}
\PY{+w}{        }\PY{n}{chart}\PY{p}{.}\PY{n}{draw\PYZus{}series}\PY{p}{(}\PY{n}{LineSeries}::\PY{n}{new}\PY{p}{(}
\PY{+w}{            }\PY{n}{points}\PY{p}{,}
\PY{+w}{            }\PY{o}{\PYZam{}}\PY{n}{RED}\PY{p}{,}
\PY{+w}{        }\PY{p}{)}\PY{p}{)}\PY{o}{?}\PY{p}{;}
\PY{+w}{        }
\PY{+w}{        }\PY{c+c1}{// Выводим текстовый ответ}
\PY{+w}{        }\PY{n+nf+fm}{println!}\PY{p}{(}\PY{l+s}{\PYZdq{}}\PY{l+s}{В поселке \PYZob{}\PYZcb{} жителей. Каждый из них примерно \PYZob{}\PYZcb{} раз в месяц ездит на поезде в город (т.е.}
\PY{l+s}{с вероятностью \PYZob{}\PYZcb{}), выбирая дни поездок по случайным мотивам независимо от остальных. Поезд}
\PY{l+s}{ходит один раз в сутки. Какой наименьшей вместимостью должен обладать поезд, чтобы он}
\PY{l+s}{переполнялся в среднем не чаще одного раза в \PYZob{}\PYZcb{} дней (т.е. с вероятностью \PYZob{}\PYZcb{})? \PYZhy{}\PYZhy{} \PYZob{}\PYZcb{}}\PY{l+s}{\PYZdq{}}\PY{p}{,}
\PY{+w}{        }\PY{n}{task}\PY{p}{.}\PY{n}{citizen\PYZus{}count}\PY{p}{,}
\PY{+w}{            }\PY{n}{task}\PY{p}{.}\PY{n}{citizen\PYZus{}train\PYZus{}probability}\PY{o}{/}\PY{+w}{ }\PY{p}{(}\PY{l+m+mf}{1.0}\PY{+w}{ }\PY{o}{/}\PY{+w}{ }\PY{l+m+mf}{30.0}\PY{p}{)}\PY{p}{,}
\PY{+w}{            }\PY{n}{task}\PY{p}{.}\PY{n}{citizen\PYZus{}train\PYZus{}probability}\PY{p}{,}
\PY{+w}{            }\PY{n}{one\PYZus{}fail\PYZus{}per\PYZus{}days}\PY{p}{,}
\PY{+w}{            }\PY{l+m+mf}{1.0}\PY{+w}{ }\PY{o}{/}\PY{+w}{ }\PY{n}{one\PYZus{}fail\PYZus{}per\PYZus{}days}\PY{p}{,}
\PY{+w}{            }\PY{n}{task}\PY{p}{.}\PY{n}{current\PYZus{}train\PYZus{}capacity}
\PY{+w}{        }\PY{p}{)}\PY{p}{;}
\PY{+w}{        }\PY{n}{answers}\PY{p}{.}\PY{n}{task\PYZus{}7}\PY{+w}{ }\PY{o}{=}\PY{+w}{ }\PY{n+nb}{Some}\PY{p}{(}\PY{n}{task}\PY{p}{.}\PY{n}{current\PYZus{}train\PYZus{}capacity}\PY{p}{)}\PY{p}{;}

\PY{+w}{        }\PY{n+nb}{Ok}\PY{p}{(}\PY{p}{(}\PY{p}{)}\PY{p}{)}
\PY{+w}{    }\PY{p}{\PYZcb{}}\PY{p}{)}\PY{p}{;}
\PY{+w}{    }\PY{n}{figure}
\PY{p}{\PYZcb{}}\PY{p}{;}
\PY{n}{figure}
\end{Verbatim}
\end{tcolorbox}

    \begin{Verbatim}[commandchars=\\\{\}]
В поселке 2500 жителей. Каждый из них примерно 6 раз в месяц ездит на поезде в
город (т.е.
с вероятностью 0.2), выбирая дни поездок по случайным мотивам независимо от
остальных. Поезд
ходит один раз в сутки. Какой наименьшей вместимостью должен обладать поезд,
чтобы он
переполнялся в среднем не чаще одного раза в 100 дней (т.е. с вероятностью
0.01)? -- 546
    \end{Verbatim}

            \begin{tcolorbox}[breakable, size=fbox, boxrule=.5pt, pad at break*=1mm, opacityfill=0]
\prompt{Out}{outcolor}{11}{\boxspacing}
\begin{Verbatim}[commandchars=\\\{\}]
Timing: false
sccache: true

\end{Verbatim}
\end{tcolorbox}
        
    \begin{tcolorbox}[breakable, size=fbox, boxrule=1pt, pad at break*=1mm,colback=cellbackground, colframe=cellborder]
\prompt{In}{incolor}{12}{\boxspacing}
\begin{Verbatim}[commandchars=\\\{\}]
:\PY{n+nc}{timing}\PY{+w}{ }\PY{l+m+mi}{1}
:\PY{n+nc}{sccache}\PY{+w}{ }\PY{l+m+mi}{1}
:\PY{n+nc}{dep}\PY{+w}{ }\PY{n}{rand}\PY{+w}{ }\PY{o}{=}\PY{+w}{ }\PY{l+s}{\PYZdq{}}\PY{l+s}{0.8.5}\PY{l+s}{\PYZdq{}}

\PY{k}{use}\PY{+w}{ }\PY{n}{rand}::\PY{n}{prelude}::\PY{p}{\PYZob{}}\PY{n}{SliceRandom}\PY{p}{,}\PY{+w}{ }\PY{n}{IteratorRandom}\PY{p}{\PYZcb{}}\PY{p}{;}
\PY{k}{struct} \PY{n+nc}{Task8}\PY{p}{(}\PY{k+kt}{usize}\PY{p}{)}\PY{p}{;}

\PY{k}{impl}\PY{+w}{ }\PY{n}{Experiment}\PY{+w}{ }\PY{k}{for}\PY{+w}{ }\PY{n}{Task8}\PY{+w}{ }\PY{p}{\PYZob{}}
\PY{+w}{    }\PY{k}{type} \PY{n+nc}{Outcome}\PY{+w}{ }\PY{o}{=}\PY{+w}{ }\PY{k+kt}{bool}\PY{p}{;}
\PY{+w}{    }\PY{k}{fn} \PY{n+nf}{description}\PY{p}{(}\PY{o}{\PYZam{}}\PY{n+nb+bp}{self}\PY{p}{)}\PY{+w}{ }\PYZhy{}\PYZgt{} \PY{n+nb}{String} \PY{p}{\PYZob{}}
\PY{+w}{        }\PY{n+nf+fm}{format!}\PY{p}{(}\PY{l+s}{\PYZdq{}}\PY{l+s}{В зале кинотеатра \PYZob{}\PYZcb{} мест и все \PYZob{}\PYZcb{} билетов были распроданы. Когда посетители пришли в}
\PY{l+s}{зал, свет в зале не работал (номера мест не видны) и каждому пришлось выбирать себе место наугад.}
\PY{l+s}{Чему равна вероятность того, что все \PYZob{}\PYZcb{} посетителей сели мимо своих мест (указанных в билете)?}\PY{l+s}{\PYZdq{}}\PY{p}{,}\PY{+w}{ }\PY{n+nb+bp}{self}\PY{p}{.}\PY{l+m+mi}{0}\PY{p}{,}\PY{+w}{ }\PY{n+nb+bp}{self}\PY{p}{.}\PY{l+m+mi}{0}\PY{p}{,}\PY{+w}{ }\PY{n+nb+bp}{self}\PY{p}{.}\PY{l+m+mi}{0}\PY{p}{)}
\PY{+w}{    }\PY{p}{\PYZcb{}}
\PY{+w}{    }\PY{k}{fn} \PY{n+nf}{try\PYZus{}it}\PY{o}{\PYZlt{}}\PY{n}{T}: \PY{n+nc}{rand}::\PY{n}{Rng}\PY{o}{\PYZgt{}}\PY{p}{(}\PY{o}{\PYZam{}}\PY{n+nb+bp}{self}\PY{p}{,}\PY{+w}{ }\PY{n}{rng}: \PY{k+kp}{\PYZam{}}\PY{n+nc}{mut}\PY{+w}{ }\PY{n}{T}\PY{p}{)}\PY{+w}{ }\PYZhy{}\PYZgt{} \PY{n+nc}{Self}::\PY{n}{Outcome}\PY{+w}{ }\PY{p}{\PYZob{}}
\PY{+w}{        }\PY{c+c1}{// Составляем список посетителей и перемешиваем его}
\PY{+w}{        }\PY{k+kd}{let}\PY{+w}{ }\PY{k}{mut}\PY{+w}{ }\PY{n}{visitors}: \PY{n+nb}{Vec}\PY{o}{\PYZlt{}}\PY{k+kt}{usize}\PY{o}{\PYZgt{}}\PY{+w}{ }\PY{o}{=}\PY{+w}{ }\PY{p}{(}\PY{l+m+mi}{0}\PY{o}{..}\PY{n+nb+bp}{self}\PY{p}{.}\PY{l+m+mi}{0}\PY{p}{)}\PY{p}{.}\PY{n}{collect}\PY{p}{(}\PY{p}{)}\PY{p}{;}
\PY{+w}{        }\PY{n}{visitors}\PY{p}{.}\PY{n}{shuffle}\PY{p}{(}\PY{n}{rng}\PY{p}{)}\PY{p}{;}
\PY{+w}{        }\PY{k}{for}\PY{+w}{ }\PY{p}{(}\PY{n}{i}\PY{p}{,}\PY{+w}{ }\PY{n}{p}\PY{p}{)}\PY{+w}{ }\PY{k}{in}\PY{+w}{ }\PY{n}{visitors}\PY{p}{.}\PY{n}{iter}\PY{p}{(}\PY{p}{)}\PY{p}{.}\PY{n}{enumerate}\PY{p}{(}\PY{p}{)}\PY{+w}{ }\PY{p}{\PYZob{}}
\PY{+w}{            }\PY{k}{if}\PY{+w}{ }\PY{n}{i}\PY{+w}{ }\PY{o}{=}\PY{o}{=}\PY{+w}{ }\PY{o}{*}\PY{n}{p}\PY{+w}{ }\PY{p}{\PYZob{}}
\PY{+w}{                }\PY{c+c1}{// Если чей\PYZhy{}то номер совпал с их номером билета, то опыт не удался}
\PY{+w}{                }\PY{k}{return}\PY{+w}{ }\PY{k+kc}{false}\PY{p}{;}
\PY{+w}{            }\PY{p}{\PYZcb{}}
\PY{+w}{        }\PY{p}{\PYZcb{}}
\PY{+w}{        }\PY{c+c1}{// Иначе, все сели мимо своих мест, и опыт удался}
\PY{+w}{        }\PY{k+kc}{true}
\PY{+w}{    }\PY{p}{\PYZcb{}}
\PY{+w}{    }
\PY{+w}{    }\PY{k}{fn} \PY{n+nf}{desired\PYZus{}outcomes}\PY{p}{(}\PY{o}{\PYZam{}}\PY{n+nb+bp}{self}\PY{p}{)}\PY{+w}{ }\PYZhy{}\PYZgt{} \PY{n+nb}{Vec}\PY{o}{\PYZlt{}}\PY{n+nb+bp}{Self}::\PY{n}{Outcome}\PY{o}{\PYZgt{}}\PY{+w}{ }\PY{p}{\PYZob{}}\PY{n+nf+fm}{vec!}\PY{p}{[}\PY{k+kc}{true}\PY{p}{]}\PY{p}{\PYZcb{}}
\PY{p}{\PYZcb{}}

\PY{k+kd}{let}\PY{+w}{ }\PY{n}{a}\PY{+w}{ }\PY{o}{=}\PY{+w}{ }\PY{n}{Task8}\PY{p}{(}\PY{l+m+mi}{5}\PY{p}{)}\PY{p}{;}
\PY{k+kd}{let}\PY{+w}{ }\PY{n}{aans}\PY{+w}{ }\PY{o}{=}\PY{+w}{ }\PY{n}{a}\PY{p}{.}\PY{n}{probability\PYZus{}of\PYZus{}desired}\PY{p}{(}\PY{l+m+mi}{1\PYZus{}000\PYZus{}000}\PY{p}{)}\PY{p}{;}
\PY{n+nf+fm}{println!}\PY{p}{(}\PY{l+s}{\PYZdq{}}\PY{l+s}{\PYZob{}\PYZcb{} \PYZhy{}\PYZhy{} \PYZob{}\PYZcb{}}\PY{l+s}{\PYZdq{}}\PY{p}{,}\PY{+w}{ }\PY{n}{a}\PY{p}{.}\PY{n}{description}\PY{p}{(}\PY{p}{)}\PY{p}{,}\PY{+w}{ }\PY{n}{aans}\PY{p}{)}\PY{p}{;}
\PY{k+kd}{let}\PY{+w}{ }\PY{n}{b}\PY{+w}{ }\PY{o}{=}\PY{+w}{ }\PY{n}{Task8}\PY{p}{(}\PY{l+m+mi}{50}\PY{p}{)}\PY{p}{;}
\PY{k+kd}{let}\PY{+w}{ }\PY{n}{bans}\PY{+w}{ }\PY{o}{=}\PY{+w}{ }\PY{n}{b}\PY{p}{.}\PY{n}{probability\PYZus{}of\PYZus{}desired}\PY{p}{(}\PY{l+m+mi}{1\PYZus{}000\PYZus{}000}\PY{p}{)}\PY{p}{;}
\PY{n+nf+fm}{println!}\PY{p}{(}\PY{l+s}{\PYZdq{}}\PY{l+s}{\PYZob{}\PYZcb{} \PYZhy{}\PYZhy{} \PYZob{}\PYZcb{}}\PY{l+s}{\PYZdq{}}\PY{p}{,}\PY{+w}{ }\PY{n}{b}\PY{p}{.}\PY{n}{description}\PY{p}{(}\PY{p}{)}\PY{p}{,}\PY{+w}{ }\PY{n}{bans}\PY{p}{)}\PY{p}{;}
\PY{k+kd}{let}\PY{+w}{ }\PY{n}{c}\PY{+w}{ }\PY{o}{=}\PY{+w}{ }\PY{n}{Task8}\PY{p}{(}\PY{l+m+mi}{100}\PY{p}{)}\PY{p}{;}
\PY{k+kd}{let}\PY{+w}{ }\PY{n}{cans}\PY{+w}{ }\PY{o}{=}\PY{+w}{ }\PY{n}{c}\PY{p}{.}\PY{n}{probability\PYZus{}of\PYZus{}desired}\PY{p}{(}\PY{l+m+mi}{1\PYZus{}000\PYZus{}000}\PY{p}{)}\PY{p}{;}
\PY{n+nf+fm}{println!}\PY{p}{(}\PY{l+s}{\PYZdq{}}\PY{l+s}{\PYZob{}\PYZcb{} \PYZhy{}\PYZhy{} \PYZob{}\PYZcb{}}\PY{l+s}{\PYZdq{}}\PY{p}{,}\PY{+w}{ }\PY{n}{c}\PY{p}{.}\PY{n}{description}\PY{p}{(}\PY{p}{)}\PY{p}{,}\PY{+w}{ }\PY{n}{cans}\PY{p}{)}\PY{p}{;}
\PY{n}{answers}\PY{p}{.}\PY{n}{task\PYZus{}8}\PY{+w}{ }\PY{o}{=}\PY{+w}{ }\PY{n+nb}{Some}\PY{p}{(}\PY{p}{(}\PY{n}{aans}\PY{p}{,}\PY{+w}{ }\PY{n}{bans}\PY{p}{,}\PY{+w}{ }\PY{n}{cans}\PY{p}{)}\PY{p}{)}\PY{p}{;}
\end{Verbatim}
\end{tcolorbox}

    \begin{Verbatim}[commandchars=\\\{\}]
В зале кинотеатра 5 мест и все 5 билетов были распроданы. Когда посетители
пришли в
зал, свет в зале не работал (номера мест не видны) и каждому пришлось выбирать
себе место наугад.
Чему равна вероятность того, что все 5 посетителей сели мимо своих мест
(указанных в билете)? -- 0.366745
В зале кинотеатра 50 мест и все 50 билетов были распроданы. Когда посетители
пришли в
зал, свет в зале не работал (номера мест не видны) и каждому пришлось выбирать
себе место наугад.
Чему равна вероятность того, что все 50 посетителей сели мимо своих мест
(указанных в билете)? -- 0.367519
В зале кинотеатра 100 мест и все 100 билетов были распроданы. Когда посетители
пришли в
зал, свет в зале не работал (номера мест не видны) и каждому пришлось выбирать
себе место наугад.
Чему равна вероятность того, что все 100 посетителей сели мимо своих мест
(указанных в билете)? -- 0.367601
    \end{Verbatim}

            \begin{tcolorbox}[breakable, size=fbox, boxrule=.5pt, pad at break*=1mm, opacityfill=0]
\prompt{Out}{outcolor}{12}{\boxspacing}
\begin{Verbatim}[commandchars=\\\{\}]
Timing: true
sccache: true

\end{Verbatim}
\end{tcolorbox}
        
    \begin{tcolorbox}[breakable, size=fbox, boxrule=1pt, pad at break*=1mm,colback=cellbackground, colframe=cellborder]
\prompt{In}{incolor}{13}{\boxspacing}
\begin{Verbatim}[commandchars=\\\{\}]
:\PY{n+nc}{timing}\PY{+w}{ }\PY{l+m+mi}{1}
:\PY{n+nc}{sccache}\PY{+w}{ }\PY{l+m+mi}{1}
:\PY{n+nc}{dep}\PY{+w}{ }\PY{n}{rand}\PY{+w}{ }\PY{o}{=}\PY{+w}{ }\PY{l+s}{\PYZdq{}}\PY{l+s}{0.8.5}\PY{l+s}{\PYZdq{}}
:\PY{n+nc}{dep}\PY{+w}{ }\PY{n}{itertools}\PY{+w}{ }\PY{o}{=}\PY{+w}{ }\PY{l+s}{\PYZdq{}}\PY{l+s}{0.10.5}\PY{l+s}{\PYZdq{}}

\PY{k}{use}\PY{+w}{ }\PY{n}{rand}::\PY{n}{prelude}::\PY{p}{\PYZob{}}\PY{n}{SliceRandom}\PY{p}{,}\PY{+w}{ }\PY{n}{IteratorRandom}\PY{p}{\PYZcb{}}\PY{p}{;}

\PY{l+s+sd}{/// В этом задании есть 4 группы, из которых набираются члены в группу определенного размера;}
\PY{l+s+sd}{/// здесь мы храним размеры этих групп.}
\PY{k}{struct} \PY{n+nc}{Task9}\PY{p}{\PYZob{}}
\PY{+w}{    }\PY{k}{pub}\PY{+w}{ }\PY{n}{group\PYZus{}size}: \PY{k+kt}{usize}\PY{p}{,}
\PY{+w}{    }\PY{k}{pub}\PY{+w}{ }\PY{n}{cat1\PYZus{}size}: \PY{k+kt}{usize}\PY{p}{,}
\PY{+w}{    }\PY{k}{pub}\PY{+w}{ }\PY{n}{cat2\PYZus{}size}: \PY{k+kt}{usize}\PY{p}{,}
\PY{+w}{    }\PY{k}{pub}\PY{+w}{ }\PY{n}{cat3\PYZus{}size}: \PY{k+kt}{usize}\PY{p}{,}
\PY{+w}{    }\PY{k}{pub}\PY{+w}{ }\PY{n}{cat4\PYZus{}size}: \PY{k+kt}{usize}\PY{p}{,}
\PY{p}{\PYZcb{}}

\PY{k}{impl}\PY{+w}{ }\PY{n}{Task9}\PY{+w}{ }\PY{p}{\PYZob{}}
\PY{+w}{    }\PY{k}{pub}\PY{+w}{ }\PY{k}{fn} \PY{n+nf}{new}\PY{p}{(}\PY{p}{)}\PY{+w}{ }\PYZhy{}\PYZgt{} \PY{n+nc}{Self}\PY{+w}{ }\PY{p}{\PYZob{}}
\PY{+w}{        }\PY{n}{Task9}\PY{+w}{ }\PY{p}{\PYZob{}}
\PY{+w}{            }\PY{n}{group\PYZus{}size}: \PY{l+m+mi}{6}\PY{p}{,}
\PY{+w}{            }\PY{n}{cat1\PYZus{}size}: \PY{l+m+mi}{6}\PY{p}{,}
\PY{+w}{            }\PY{n}{cat2\PYZus{}size}: \PY{l+m+mi}{6}\PY{p}{,}
\PY{+w}{            }\PY{n}{cat3\PYZus{}size}: \PY{l+m+mi}{10}\PY{p}{,}
\PY{+w}{            }\PY{n}{cat4\PYZus{}size}: \PY{l+m+mi}{12}\PY{p}{,}
\PY{+w}{        }\PY{p}{\PYZcb{}}
\PY{+w}{    }\PY{p}{\PYZcb{}}
\PY{p}{\PYZcb{}}

\PY{k}{use}\PY{+w}{ }\PY{n}{core}::\PY{n}{iter}::\PY{n}{repeat}\PY{p}{;}
\PY{k}{use}\PY{+w}{ }\PY{n}{itertools}::\PY{n}{Itertools}\PY{p}{;}
\PY{k}{impl}\PY{+w}{ }\PY{n}{Experiment}\PY{+w}{ }\PY{k}{for}\PY{+w}{ }\PY{n}{Task9}\PY{+w}{ }\PY{p}{\PYZob{}}
\PY{+w}{    }\PY{k}{type} \PY{n+nc}{Outcome}\PY{+w}{ }\PY{o}{=}\PY{+w}{ }\PY{k+kt}{u8}\PY{p}{;}
\PY{+w}{    }\PY{k}{fn} \PY{n+nf}{description}\PY{p}{(}\PY{o}{\PYZam{}}\PY{n+nb+bp}{self}\PY{p}{)}\PY{+w}{ }\PYZhy{}\PYZgt{} \PY{n+nb}{String} \PY{p}{\PYZob{}}
\PY{+w}{        }\PY{n+nf+fm}{format!}\PY{p}{(}\PY{l+s}{\PYZdq{}}\PY{l+s}{На факультете работает \PYZob{}\PYZcb{} профессоров, \PYZob{}\PYZcb{} доцентов, \PYZob{}\PYZcb{} старших преподавателей и \PYZob{}\PYZcb{}}
\PY{l+s}{ассистентов. Из работников факультета случайным образом формируется комитет из \PYZob{}\PYZcb{} участников.}
\PY{l+s}{Чему равна вероятность того, что в комитет войдет по крайней мере один человек каждой должности?}\PY{l+s}{\PYZdq{}}\PY{p}{,}
\PY{+w}{        }\PY{n+nb+bp}{self}\PY{p}{.}\PY{n}{cat1\PYZus{}size}\PY{p}{,}
\PY{+w}{        }\PY{n+nb+bp}{self}\PY{p}{.}\PY{n}{cat2\PYZus{}size}\PY{p}{,}
\PY{+w}{        }\PY{n+nb+bp}{self}\PY{p}{.}\PY{n}{cat3\PYZus{}size}\PY{p}{,}
\PY{+w}{        }\PY{n+nb+bp}{self}\PY{p}{.}\PY{n}{cat4\PYZus{}size}\PY{p}{,}
\PY{+w}{        }\PY{n+nb+bp}{self}\PY{p}{.}\PY{n}{group\PYZus{}size}\PY{p}{,}
\PY{+w}{        }\PY{p}{)}
\PY{+w}{    }\PY{p}{\PYZcb{}}
\PY{+w}{    }\PY{k}{fn} \PY{n+nf}{try\PYZus{}it}\PY{o}{\PYZlt{}}\PY{n}{T}: \PY{n+nc}{rand}::\PY{n}{Rng}\PY{o}{\PYZgt{}}\PY{p}{(}\PY{o}{\PYZam{}}\PY{n+nb+bp}{self}\PY{p}{,}\PY{+w}{ }\PY{n}{rng}: \PY{k+kp}{\PYZam{}}\PY{n+nc}{mut}\PY{+w}{ }\PY{n}{T}\PY{p}{)}\PY{+w}{ }\PYZhy{}\PYZgt{} \PY{n+nc}{Self}::\PY{n}{Outcome}\PY{+w}{ }\PY{p}{\PYZob{}}
\PY{+w}{        }\PY{c+c1}{// Группу можно набрать, только если членов всех остальных групп достаточно}
\PY{+w}{        }\PY{n+nf+fm}{assert!}\PY{p}{(}\PY{n+nb+bp}{self}\PY{p}{.}\PY{n}{cat1\PYZus{}size}\PY{+w}{ }\PY{o}{+}\PY{+w}{ }\PY{n+nb+bp}{self}\PY{p}{.}\PY{n}{cat2\PYZus{}size}\PY{+w}{ }\PY{o}{+}\PY{+w}{ }\PY{n+nb+bp}{self}\PY{p}{.}\PY{n}{cat3\PYZus{}size}\PY{+w}{ }\PY{o}{+}\PY{+w}{ }\PY{n+nb+bp}{self}\PY{p}{.}\PY{n}{cat4\PYZus{}size}\PY{+w}{ }\PY{o}{\PYZgt{}}\PY{o}{=}\PY{+w}{ }\PY{n+nb+bp}{self}\PY{p}{.}\PY{n}{group\PYZus{}size}\PY{p}{)}\PY{p}{;}
\PY{+w}{        }\PY{c+c1}{// Составляем список, состоящий из значений 1, 2, 3 и 4 для каждой из исходных групп.}
\PY{+w}{        }\PY{k+kd}{let}\PY{+w}{ }\PY{k}{mut}\PY{+w}{ }\PY{n}{members}: \PY{n+nb}{Vec}\PY{o}{\PYZlt{}}\PY{k+kt}{u8}\PY{o}{\PYZgt{}}\PY{+w}{ }\PY{o}{=}\PY{+w}{ }\PY{n+nb}{Vec}::\PY{n}{with\PYZus{}capacity}\PY{p}{(}\PY{n+nb+bp}{self}\PY{p}{.}\PY{n}{cat1\PYZus{}size}\PY{+w}{ }\PY{o}{+}\PY{+w}{ }\PY{n+nb+bp}{self}\PY{p}{.}\PY{n}{cat2\PYZus{}size}\PY{+w}{ }\PY{o}{+}\PY{+w}{ }\PY{n+nb+bp}{self}\PY{p}{.}\PY{n}{cat3\PYZus{}size}\PY{+w}{ }\PY{o}{+}\PY{+w}{ }\PY{n+nb+bp}{self}\PY{p}{.}\PY{n}{cat4\PYZus{}size}\PY{p}{)}\PY{p}{;}
\PY{+w}{        }\PY{n}{members}\PY{p}{.}\PY{n}{extend}\PY{p}{(}\PY{n}{repeat}\PY{p}{(}\PY{l+m+mi}{1}\PY{p}{)}\PY{p}{.}\PY{n}{take}\PY{p}{(}\PY{n+nb+bp}{self}\PY{p}{.}\PY{n}{cat1\PYZus{}size}\PY{p}{)}\PY{p}{)}\PY{p}{;}
\PY{+w}{        }\PY{n}{members}\PY{p}{.}\PY{n}{extend}\PY{p}{(}\PY{n}{repeat}\PY{p}{(}\PY{l+m+mi}{2}\PY{p}{)}\PY{p}{.}\PY{n}{take}\PY{p}{(}\PY{n+nb+bp}{self}\PY{p}{.}\PY{n}{cat2\PYZus{}size}\PY{p}{)}\PY{p}{)}\PY{p}{;}
\PY{+w}{        }\PY{n}{members}\PY{p}{.}\PY{n}{extend}\PY{p}{(}\PY{n}{repeat}\PY{p}{(}\PY{l+m+mi}{3}\PY{p}{)}\PY{p}{.}\PY{n}{take}\PY{p}{(}\PY{n+nb+bp}{self}\PY{p}{.}\PY{n}{cat3\PYZus{}size}\PY{p}{)}\PY{p}{)}\PY{p}{;}
\PY{+w}{        }\PY{n}{members}\PY{p}{.}\PY{n}{extend}\PY{p}{(}\PY{n}{repeat}\PY{p}{(}\PY{l+m+mi}{4}\PY{p}{)}\PY{p}{.}\PY{n}{take}\PY{p}{(}\PY{n+nb+bp}{self}\PY{p}{.}\PY{n}{cat4\PYZus{}size}\PY{p}{)}\PY{p}{)}\PY{p}{;}
\PY{+w}{        }\PY{c+c1}{// Перемешиваем этот список}
\PY{+w}{        }\PY{n}{members}\PY{p}{.}\PY{n}{shuffle}\PY{p}{(}\PY{n}{rng}\PY{p}{)}\PY{p}{;}
\PY{+w}{        }\PY{c+c1}{// Берем первые несколько элементов, из них выбираем уникальные и считаем их}
\PY{+w}{        }\PY{n}{members}\PY{p}{.}\PY{n}{iter}\PY{p}{(}\PY{p}{)}\PY{p}{.}\PY{n}{take}\PY{p}{(}\PY{n+nb+bp}{self}\PY{p}{.}\PY{n}{group\PYZus{}size}\PY{p}{)}\PY{p}{.}\PY{n}{unique}\PY{p}{(}\PY{p}{)}\PY{p}{.}\PY{n}{count}\PY{p}{(}\PY{p}{)}\PY{+w}{ }\PY{k}{as}\PY{+w}{ }\PY{k+kt}{u8}
\PY{+w}{    }\PY{p}{\PYZcb{}}
\PY{+w}{    }
\PY{+w}{    }\PY{l+s+sd}{/// Нам нужно, чтобы уникальных элементов было 4.}
\PY{+w}{    }\PY{k}{fn} \PY{n+nf}{desired\PYZus{}outcomes}\PY{p}{(}\PY{o}{\PYZam{}}\PY{n+nb+bp}{self}\PY{p}{)}\PY{+w}{ }\PYZhy{}\PYZgt{} \PY{n+nb}{Vec}\PY{o}{\PYZlt{}}\PY{n+nb+bp}{Self}::\PY{n}{Outcome}\PY{o}{\PYZgt{}}\PY{+w}{ }\PY{p}{\PYZob{}}\PY{n+nf+fm}{vec!}\PY{p}{[}\PY{l+m+mi}{4}\PY{p}{]}\PY{p}{\PYZcb{}}
\PY{p}{\PYZcb{}}

\PY{k+kd}{let}\PY{+w}{ }\PY{n}{a}\PY{+w}{ }\PY{o}{=}\PY{+w}{ }\PY{n}{Task9}::\PY{n}{new}\PY{p}{(}\PY{p}{)}\PY{p}{;}
\PY{k+kd}{let}\PY{+w}{ }\PY{n}{prob}\PY{+w}{ }\PY{o}{=}\PY{+w}{ }\PY{n}{a}\PY{p}{.}\PY{n}{probability\PYZus{}of\PYZus{}desired}\PY{p}{(}\PY{l+m+mi}{1\PYZus{}000\PYZus{}000}\PY{p}{)}\PY{p}{;}
\PY{n+nf+fm}{println!}\PY{p}{(}\PY{l+s}{\PYZdq{}}\PY{l+s}{\PYZob{}\PYZcb{} \PYZhy{}\PYZhy{} \PYZob{}\PYZcb{}}\PY{l+s}{\PYZdq{}}\PY{p}{,}\PY{+w}{ }\PY{n}{a}\PY{p}{.}\PY{n}{description}\PY{p}{(}\PY{p}{)}\PY{p}{,}\PY{+w}{ }\PY{n}{prob}\PY{p}{)}\PY{p}{;}
\PY{n}{answers}\PY{p}{.}\PY{n}{task\PYZus{}9}\PY{+w}{ }\PY{o}{=}\PY{+w}{ }\PY{n+nb}{Some}\PY{p}{(}\PY{n}{prob}\PY{p}{)}\PY{p}{;}
\end{Verbatim}
\end{tcolorbox}

    \begin{Verbatim}[commandchars=\\\{\}]
На факультете работает 6 профессоров, 6 доцентов, 10 старших преподавателей и 12
ассистентов. Из работников факультета случайным образом формируется комитет из 6
участников.
Чему равна вероятность того, что в комитет войдет по крайней мере один человек
каждой должности? -- 0.378413
    \end{Verbatim}

            \begin{tcolorbox}[breakable, size=fbox, boxrule=.5pt, pad at break*=1mm, opacityfill=0]
\prompt{Out}{outcolor}{13}{\boxspacing}
\begin{Verbatim}[commandchars=\\\{\}]
Timing: false
sccache: true

\end{Verbatim}
\end{tcolorbox}
        
    \begin{tcolorbox}[breakable, size=fbox, boxrule=1pt, pad at break*=1mm,colback=cellbackground, colframe=cellborder]
\prompt{In}{incolor}{14}{\boxspacing}
\begin{Verbatim}[commandchars=\\\{\}]
:\PY{n+nc}{timing}\PY{+w}{ }\PY{l+m+mi}{1}
:\PY{n+nc}{sccache}\PY{+w}{ }\PY{l+m+mi}{1}
:\PY{n+nc}{dep}\PY{+w}{ }\PY{n}{rand}\PY{+w}{ }\PY{o}{=}\PY{+w}{ }\PY{l+s}{\PYZdq{}}\PY{l+s}{0.8.5}\PY{l+s}{\PYZdq{}}
:\PY{n+nc}{dep}\PY{+w}{ }\PY{n}{rand\PYZus{}distr}\PY{+w}{ }\PY{o}{=}\PY{+w}{ }\PY{l+s}{\PYZdq{}}\PY{l+s}{0.4.3}\PY{l+s}{\PYZdq{}}


\PY{l+s+sd}{/// Для этой задачи требуется хранить экземпляр распределения,}
\PY{l+s+sd}{/// членами которого являются пары действительных чисел.}
\PY{k}{struct} \PY{n+nc}{Task10}\PY{o}{\PYZlt{}}\PY{n}{T}: \PY{n+nc}{rand\PYZus{}distr}::\PY{n}{Distribution}\PY{o}{\PYZlt{}}\PY{p}{[}\PY{k+kt}{f64}\PY{p}{;}\PY{+w}{ }\PY{l+m+mi}{2}\PY{p}{]}\PY{o}{\PYZgt{}}\PY{o}{\PYZgt{}}\PY{+w}{ }\PY{p}{\PYZob{}}
\PY{+w}{    }\PY{n}{distribution}: \PY{n+nc}{T}\PY{p}{,}
\PY{p}{\PYZcb{}}

\PY{c+c1}{// Эти функции взяты из данного ответа на StackOverflow:}
\PY{c+c1}{// https://stackoverflow.com/a/2049593/5936187}
\PY{c+c1}{// Они связаны с вопросом того, находится ли точка внутри треугольника.}
\PY{k}{fn} \PY{n+nf}{sign}\PY{p}{(}\PY{n}{a}: \PY{p}{[}\PY{k+kt}{f64}\PY{p}{;}\PY{+w}{ }\PY{l+m+mi}{2}\PY{p}{]}\PY{p}{,}\PY{+w}{ }\PY{n}{b}: \PY{p}{[}\PY{k+kt}{f64}\PY{p}{;}\PY{+w}{ }\PY{l+m+mi}{2}\PY{p}{]}\PY{p}{,}\PY{+w}{ }\PY{n}{c}: \PY{p}{[}\PY{k+kt}{f64}\PY{p}{;}\PY{+w}{ }\PY{l+m+mi}{2}\PY{p}{]}\PY{p}{)}\PY{+w}{ }\PYZhy{}\PYZgt{} \PY{k+kt}{f64} \PY{p}{\PYZob{}}
\PY{+w}{    }\PY{p}{(}\PY{n}{a}\PY{p}{[}\PY{l+m+mi}{0}\PY{p}{]}\PY{o}{\PYZhy{}}\PY{n}{c}\PY{p}{[}\PY{l+m+mi}{0}\PY{p}{]}\PY{p}{)}\PY{+w}{ }\PY{o}{*}\PY{+w}{ }\PY{p}{(}\PY{n}{b}\PY{p}{[}\PY{l+m+mi}{1}\PY{p}{]}\PY{o}{\PYZhy{}}\PY{n}{c}\PY{p}{[}\PY{l+m+mi}{1}\PY{p}{]}\PY{p}{)}\PY{+w}{ }\PY{o}{\PYZhy{}}\PY{+w}{ }\PY{p}{(}\PY{n}{b}\PY{p}{[}\PY{l+m+mi}{0}\PY{p}{]}\PY{o}{\PYZhy{}}\PY{n}{c}\PY{p}{[}\PY{l+m+mi}{0}\PY{p}{]}\PY{p}{)}\PY{+w}{ }\PY{o}{*}\PY{+w}{ }\PY{p}{(}\PY{n}{a}\PY{p}{[}\PY{l+m+mi}{1}\PY{p}{]}\PY{o}{\PYZhy{}}\PY{n}{c}\PY{p}{[}\PY{l+m+mi}{1}\PY{p}{]}\PY{p}{)}
\PY{p}{\PYZcb{}}

\PY{k}{fn} \PY{n+nf}{point\PYZus{}in\PYZus{}triangle}\PY{p}{(}\PY{n}{p}: \PY{p}{[}\PY{k+kt}{f64}\PY{p}{;}\PY{l+m+mi}{2}\PY{p}{]}\PY{p}{,}\PY{+w}{ }\PY{n}{a}: \PY{p}{[}\PY{k+kt}{f64}\PY{p}{;}\PY{+w}{ }\PY{l+m+mi}{2}\PY{p}{]}\PY{p}{,}\PY{+w}{ }\PY{n}{b}: \PY{p}{[}\PY{k+kt}{f64}\PY{p}{;}\PY{+w}{ }\PY{l+m+mi}{2}\PY{p}{]}\PY{p}{,}\PY{+w}{ }\PY{n}{c}: \PY{p}{[}\PY{k+kt}{f64}\PY{p}{;}\PY{+w}{ }\PY{l+m+mi}{2}\PY{p}{]}\PY{p}{)}\PY{+w}{ }\PYZhy{}\PYZgt{} \PY{k+kt}{bool} \PY{p}{\PYZob{}}
\PY{+w}{    }\PY{k+kd}{let}\PY{+w}{ }\PY{n}{d1}\PY{+w}{ }\PY{o}{=}\PY{+w}{ }\PY{n}{sign}\PY{p}{(}\PY{n}{p}\PY{p}{,}\PY{+w}{ }\PY{n}{a}\PY{p}{,}\PY{+w}{ }\PY{n}{b}\PY{p}{)}\PY{p}{;}
\PY{+w}{    }\PY{k+kd}{let}\PY{+w}{ }\PY{n}{d2}\PY{+w}{ }\PY{o}{=}\PY{+w}{ }\PY{n}{sign}\PY{p}{(}\PY{n}{p}\PY{p}{,}\PY{+w}{ }\PY{n}{b}\PY{p}{,}\PY{+w}{ }\PY{n}{c}\PY{p}{)}\PY{p}{;}
\PY{+w}{    }\PY{k+kd}{let}\PY{+w}{ }\PY{n}{d3}\PY{+w}{ }\PY{o}{=}\PY{+w}{ }\PY{n}{sign}\PY{p}{(}\PY{n}{p}\PY{p}{,}\PY{+w}{ }\PY{n}{c}\PY{p}{,}\PY{+w}{ }\PY{n}{a}\PY{p}{)}\PY{p}{;}
\PY{+w}{    }\PY{k+kd}{let}\PY{+w}{ }\PY{n}{has\PYZus{}neg}\PY{+w}{ }\PY{o}{=}\PY{+w}{ }\PY{p}{(}\PY{n}{d1}\PY{o}{\PYZlt{}}\PY{l+m+mf}{0.0}\PY{p}{)}\PY{+w}{ }\PY{o}{|}\PY{o}{|}\PY{+w}{ }\PY{p}{(}\PY{n}{d2}\PY{o}{\PYZlt{}}\PY{l+m+mf}{0.0}\PY{p}{)}\PY{+w}{ }\PY{o}{|}\PY{o}{|}\PY{+w}{ }\PY{p}{(}\PY{n}{d3}\PY{o}{\PYZlt{}}\PY{l+m+mf}{0.0}\PY{p}{)}\PY{p}{;}
\PY{+w}{    }\PY{k+kd}{let}\PY{+w}{ }\PY{n}{has\PYZus{}pos}\PY{+w}{ }\PY{o}{=}\PY{+w}{ }\PY{p}{(}\PY{n}{d1}\PY{o}{\PYZgt{}}\PY{l+m+mf}{0.0}\PY{p}{)}\PY{+w}{ }\PY{o}{|}\PY{o}{|}\PY{+w}{ }\PY{p}{(}\PY{n}{d2}\PY{o}{\PYZgt{}}\PY{l+m+mf}{0.0}\PY{p}{)}\PY{+w}{ }\PY{o}{|}\PY{o}{|}\PY{+w}{ }\PY{p}{(}\PY{n}{d3}\PY{o}{\PYZgt{}}\PY{l+m+mf}{0.0}\PY{p}{)}\PY{p}{;}
\PY{+w}{    }\PY{o}{!}\PY{p}{(}\PY{n}{has\PYZus{}neg}\PY{+w}{ }\PY{o}{\PYZam{}}\PY{o}{\PYZam{}}\PY{+w}{ }\PY{n}{has\PYZus{}pos}\PY{p}{)}
\PY{p}{\PYZcb{}}

\PY{k}{impl}\PY{o}{\PYZlt{}}\PY{n}{T}\PY{o}{\PYZgt{}}\PY{+w}{ }\PY{n}{Experiment}\PY{+w}{ }\PY{k}{for}\PY{+w}{ }\PY{n}{Task10}\PY{o}{\PYZlt{}}\PY{n}{T}\PY{o}{\PYZgt{}}
\PY{k}{where}\PY{+w}{ }\PY{n}{T}: \PY{n+nc}{rand\PYZus{}distr}::\PY{n}{Distribution}\PY{o}{\PYZlt{}}\PY{p}{[}\PY{k+kt}{f64}\PY{p}{;}\PY{+w}{ }\PY{l+m+mi}{2}\PY{p}{]}\PY{o}{\PYZgt{}}
\PY{p}{\PYZob{}}
\PY{+w}{    }\PY{k}{type} \PY{n+nc}{Outcome}\PY{+w}{ }\PY{o}{=}\PY{+w}{ }\PY{k+kt}{bool}\PY{p}{;}
\PY{+w}{    }\PY{k}{fn} \PY{n+nf}{description}\PY{p}{(}\PY{o}{\PYZam{}}\PY{n+nb+bp}{self}\PY{p}{)}\PY{+w}{ }\PYZhy{}\PYZgt{} \PY{n+nb}{String} \PY{p}{\PYZob{}}
\PY{+w}{        }\PY{n+nf+fm}{format!}\PY{p}{(}\PY{l+s}{\PYZdq{}}\PY{l+s}{Чему равна вероятность того, что 4 случайно выбранные точки в данном выпуклом}
\PY{l+s}{множестве являются вершинами вогнутого четырехугольника?}\PY{l+s}{\PYZdq{}}\PY{p}{)}
\PY{+w}{    }\PY{p}{\PYZcb{}}
\PY{+w}{    }\PY{k}{fn} \PY{n+nf}{try\PYZus{}it}\PY{o}{\PYZlt{}}\PY{n}{R}: \PY{n+nc}{rand}::\PY{n}{Rng}\PY{o}{\PYZgt{}}\PY{p}{(}\PY{o}{\PYZam{}}\PY{n+nb+bp}{self}\PY{p}{,}\PY{+w}{ }\PY{n}{rng}: \PY{k+kp}{\PYZam{}}\PY{n+nc}{mut}\PY{+w}{ }\PY{n}{R}\PY{p}{)}\PY{+w}{ }\PYZhy{}\PYZgt{} \PY{n+nc}{Self}::\PY{n}{Outcome}\PY{+w}{ }\PY{p}{\PYZob{}}
\PY{+w}{        }\PY{c+c1}{// Сначала мы выбираем 4 точки из данного распределения}
\PY{+w}{        }\PY{k+kd}{let}\PY{+w}{ }\PY{p}{(}\PY{n}{a}\PY{p}{,}\PY{n}{b}\PY{p}{,}\PY{n}{c}\PY{p}{,}\PY{n}{d}\PY{p}{)}\PY{+w}{ }\PY{o}{=}\PY{+w}{ }\PY{p}{(}
\PY{+w}{            }\PY{n+nb+bp}{self}\PY{p}{.}\PY{n}{distribution}\PY{p}{.}\PY{n}{sample}\PY{p}{(}\PY{n}{rng}\PY{p}{)}\PY{p}{,}
\PY{+w}{            }\PY{n+nb+bp}{self}\PY{p}{.}\PY{n}{distribution}\PY{p}{.}\PY{n}{sample}\PY{p}{(}\PY{n}{rng}\PY{p}{)}\PY{p}{,}
\PY{+w}{            }\PY{n+nb+bp}{self}\PY{p}{.}\PY{n}{distribution}\PY{p}{.}\PY{n}{sample}\PY{p}{(}\PY{n}{rng}\PY{p}{)}\PY{p}{,}
\PY{+w}{            }\PY{n+nb+bp}{self}\PY{p}{.}\PY{n}{distribution}\PY{p}{.}\PY{n}{sample}\PY{p}{(}\PY{n}{rng}\PY{p}{)}\PY{p}{,}
\PY{+w}{        }\PY{p}{)}\PY{p}{;}
\PY{+w}{        }\PY{c+c1}{// Чтобы определить, является ли четырехугольник вогнутым, мы проверяем:}
\PY{+w}{        }\PY{c+c1}{// Можно ли выбрать три точки так, что четвертая точка находится посреди них?}
\PY{+w}{        }\PY{c+c1}{// Если да, то этот четырехугольник вогнутый,}
\PY{+w}{        }\PY{c+c1}{// а иначе он равен своей выпуклой оболочке и поэтому выпуклый.}
\PY{+w}{        }\PY{k}{for}\PY{+w}{ }\PY{n}{ordering}\PY{+w}{ }\PY{k}{in}\PY{+w}{ }\PY{p}{[}\PY{n}{a}\PY{p}{,}\PY{n}{b}\PY{p}{,}\PY{n}{c}\PY{p}{,}\PY{n}{d}\PY{p}{]}\PY{p}{.}\PY{n}{iter}\PY{p}{(}\PY{p}{)}\PY{p}{.}\PY{n}{permutations}\PY{p}{(}\PY{l+m+mi}{4}\PY{p}{)}\PY{+w}{ }\PY{p}{\PYZob{}}
\PY{+w}{            }\PY{k}{if}\PY{+w}{ }\PY{n}{point\PYZus{}in\PYZus{}triangle}\PY{p}{(}\PY{o}{*}\PY{n}{ordering}\PY{p}{[}\PY{l+m+mi}{0}\PY{p}{]}\PY{p}{,}\PY{+w}{ }\PY{o}{*}\PY{n}{ordering}\PY{p}{[}\PY{l+m+mi}{1}\PY{p}{]}\PY{p}{,}\PY{+w}{ }\PY{o}{*}\PY{n}{ordering}\PY{p}{[}\PY{l+m+mi}{2}\PY{p}{]}\PY{p}{,}\PY{+w}{ }\PY{o}{*}\PY{n}{ordering}\PY{p}{[}\PY{l+m+mi}{3}\PY{p}{]}\PY{p}{)}\PY{+w}{ }\PY{p}{\PYZob{}}
\PY{+w}{                }\PY{k}{return}\PY{+w}{ }\PY{k+kc}{true}\PY{p}{;}
\PY{+w}{            }\PY{p}{\PYZcb{}}
\PY{+w}{        }\PY{p}{\PYZcb{}}
\PY{+w}{        }\PY{k+kc}{false}
\PY{+w}{    }\PY{p}{\PYZcb{}}
\PY{+w}{    }
\PY{+w}{    }\PY{k}{fn} \PY{n+nf}{desired\PYZus{}outcomes}\PY{p}{(}\PY{o}{\PYZam{}}\PY{n+nb+bp}{self}\PY{p}{)}\PY{+w}{ }\PYZhy{}\PYZgt{} \PY{n+nb}{Vec}\PY{o}{\PYZlt{}}\PY{n+nb+bp}{Self}::\PY{n}{Outcome}\PY{o}{\PYZgt{}}\PY{+w}{ }\PY{p}{\PYZob{}}\PY{n+nf+fm}{vec!}\PY{p}{[}\PY{k+kc}{true}\PY{p}{]}\PY{p}{\PYZcb{}}
\PY{p}{\PYZcb{}}

\PY{k+kd}{let}\PY{+w}{ }\PY{n}{a}\PY{+w}{ }\PY{o}{=}\PY{+w}{ }\PY{n}{Task10}\PY{+w}{ }\PY{p}{\PYZob{}}
\PY{+w}{  }\PY{n}{distribution}: \PY{n+nc}{rand\PYZus{}distr}::\PY{n}{UnitDisc}\PY{p}{,}
\PY{p}{\PYZcb{}}\PY{p}{;}

\PY{k+kd}{let}\PY{+w}{ }\PY{n}{prob}\PY{+w}{ }\PY{o}{=}\PY{+w}{ }\PY{n}{a}\PY{p}{.}\PY{n}{probability\PYZus{}of\PYZus{}desired}\PY{p}{(}\PY{l+m+mi}{1\PYZus{}000\PYZus{}000}\PY{p}{)}\PY{p}{;}
\PY{n+nf+fm}{println!}\PY{p}{(}\PY{l+s}{\PYZdq{}}\PY{l+s}{\PYZob{}\PYZcb{} \PYZhy{}\PYZhy{} \PYZob{}\PYZcb{}}\PY{l+s}{\PYZdq{}}\PY{p}{,}\PY{+w}{ }\PY{n}{a}\PY{p}{.}\PY{n}{description}\PY{p}{(}\PY{p}{)}\PY{p}{,}\PY{+w}{ }\PY{n}{prob}\PY{p}{)}\PY{p}{;}
\PY{n}{answers}\PY{p}{.}\PY{n}{task\PYZus{}10}\PY{+w}{ }\PY{o}{=}\PY{+w}{ }\PY{n+nb}{Some}\PY{p}{(}\PY{n}{prob}\PY{p}{)}\PY{p}{;}
\end{Verbatim}
\end{tcolorbox}

    \begin{Verbatim}[commandchars=\\\{\}]
Чему равна вероятность того, что 4 случайно выбранные точки в данном выпуклом
множестве являются вершинами вогнутого четырехугольника? -- 0.295522
    \end{Verbatim}

            \begin{tcolorbox}[breakable, size=fbox, boxrule=.5pt, pad at break*=1mm, opacityfill=0]
\prompt{Out}{outcolor}{14}{\boxspacing}
\begin{Verbatim}[commandchars=\\\{\}]
Timing: true
sccache: true

\end{Verbatim}
\end{tcolorbox}
        
    \begin{tcolorbox}[breakable, size=fbox, boxrule=1pt, pad at break*=1mm,colback=cellbackground, colframe=cellborder]
\prompt{In}{incolor}{17}{\boxspacing}
\begin{Verbatim}[commandchars=\\\{\}]
\PY{c+c1}{// Посмотреть состояние ответов на задания}
\PY{n+nf+fm}{println!}\PY{p}{(}\PY{l+s}{\PYZdq{}}\PY{l+s}{Задание 1: \PYZob{}:?\PYZcb{}}\PY{l+s}{\PYZdq{}}\PY{p}{,}\PY{+w}{ }\PY{n}{answers}\PY{p}{.}\PY{n}{task\PYZus{}1}\PY{p}{.}\PY{n}{unwrap}\PY{p}{(}\PY{p}{)}\PY{p}{)}\PY{p}{;}
\PY{n+nf+fm}{println!}\PY{p}{(}\PY{l+s}{\PYZdq{}}\PY{l+s}{Задание 2: \PYZob{}:?\PYZcb{}}\PY{l+s}{\PYZdq{}}\PY{p}{,}\PY{+w}{ }\PY{n}{answers}\PY{p}{.}\PY{n}{task\PYZus{}2}\PY{p}{.}\PY{n}{unwrap}\PY{p}{(}\PY{p}{)}\PY{p}{)}\PY{p}{;}
\PY{n+nf+fm}{println!}\PY{p}{(}\PY{l+s}{\PYZdq{}}\PY{l+s}{Задание 3: \PYZob{}:?\PYZcb{}}\PY{l+s}{\PYZdq{}}\PY{p}{,}\PY{+w}{ }\PY{n}{answers}\PY{p}{.}\PY{n}{task\PYZus{}3}\PY{p}{.}\PY{n}{unwrap}\PY{p}{(}\PY{p}{)}\PY{p}{)}\PY{p}{;}
\PY{n+nf+fm}{println!}\PY{p}{(}\PY{l+s}{\PYZdq{}}\PY{l+s}{Задание 4: \PYZob{}:?\PYZcb{}}\PY{l+s}{\PYZdq{}}\PY{p}{,}\PY{+w}{ }\PY{n}{answers}\PY{p}{.}\PY{n}{task\PYZus{}4}\PY{p}{.}\PY{n}{unwrap}\PY{p}{(}\PY{p}{)}\PY{p}{)}\PY{p}{;}
\PY{n+nf+fm}{println!}\PY{p}{(}\PY{l+s}{\PYZdq{}}\PY{l+s}{Задание 5: \PYZob{}:?\PYZcb{}}\PY{l+s}{\PYZdq{}}\PY{p}{,}\PY{+w}{ }\PY{n}{answers}\PY{p}{.}\PY{n}{task\PYZus{}5}\PY{p}{.}\PY{n}{unwrap}\PY{p}{(}\PY{p}{)}\PY{p}{)}\PY{p}{;}
\PY{n+nf+fm}{println!}\PY{p}{(}\PY{l+s}{\PYZdq{}}\PY{l+s}{Задание 6: \PYZob{}:?\PYZcb{}}\PY{l+s}{\PYZdq{}}\PY{p}{,}\PY{+w}{ }\PY{n}{answers}\PY{p}{.}\PY{n}{task\PYZus{}6}\PY{p}{.}\PY{n}{unwrap}\PY{p}{(}\PY{p}{)}\PY{p}{)}\PY{p}{;}
\PY{n+nf+fm}{println!}\PY{p}{(}\PY{l+s}{\PYZdq{}}\PY{l+s}{Задание 7: \PYZob{}:?\PYZcb{}}\PY{l+s}{\PYZdq{}}\PY{p}{,}\PY{+w}{ }\PY{n}{answers}\PY{p}{.}\PY{n}{task\PYZus{}7}\PY{p}{.}\PY{n}{unwrap}\PY{p}{(}\PY{p}{)}\PY{p}{)}\PY{p}{;}
\PY{n+nf+fm}{println!}\PY{p}{(}\PY{l+s}{\PYZdq{}}\PY{l+s}{Задание 8: \PYZob{}:?\PYZcb{}}\PY{l+s}{\PYZdq{}}\PY{p}{,}\PY{+w}{ }\PY{n}{answers}\PY{p}{.}\PY{n}{task\PYZus{}8}\PY{p}{.}\PY{n}{unwrap}\PY{p}{(}\PY{p}{)}\PY{p}{)}\PY{p}{;}
\PY{n+nf+fm}{println!}\PY{p}{(}\PY{l+s}{\PYZdq{}}\PY{l+s}{Задание 9: \PYZob{}:?\PYZcb{}}\PY{l+s}{\PYZdq{}}\PY{p}{,}\PY{+w}{ }\PY{n}{answers}\PY{p}{.}\PY{n}{task\PYZus{}9}\PY{p}{.}\PY{n}{unwrap}\PY{p}{(}\PY{p}{)}\PY{p}{)}\PY{p}{;}
\PY{n+nf+fm}{println!}\PY{p}{(}\PY{l+s}{\PYZdq{}}\PY{l+s}{Задание 10: \PYZob{}:?\PYZcb{}}\PY{l+s}{\PYZdq{}}\PY{p}{,}\PY{+w}{ }\PY{n}{answers}\PY{p}{.}\PY{n}{task\PYZus{}10}\PY{p}{.}\PY{n}{unwrap}\PY{p}{(}\PY{p}{)}\PY{p}{)}\PY{p}{;}
\end{Verbatim}
\end{tcolorbox}

    \begin{Verbatim}[commandchars=\\\{\}]
Задание 1: 0.725906
Задание 2: 0.7984
    \end{Verbatim}

    \begin{Verbatim}[commandchars=\\\{\}]
Задание 3: 0.254331
Задание 4: 0.49964
Задание 5: 6.537402833236494
Задание 6: 0.3688154219891535
Задание 7: 546
Задание 8: (0.366745, 0.367519, 0.367601)
Задание 9: 0.378413
Задание 10: 0.295522
    \end{Verbatim}

    \begin{tcolorbox}[breakable, size=fbox, boxrule=1pt, pad at break*=1mm,colback=cellbackground, colframe=cellborder]
\prompt{In}{incolor}{ }{\boxspacing}
\begin{Verbatim}[commandchars=\\\{\}]

\end{Verbatim}
\end{tcolorbox}


    % Add a bibliography block to the postdoc
    
    
    
\end{document}
